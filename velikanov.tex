% !TEX TS-program = pdflatexmk
\documentclass[press]{vestnik}

\draft{3}

\usepackage{ulem}
\setlength\ULdepth{5pt}

\OJS{1049}

\begin{document}

\newcommand\change[1]{%
  \mathrel{\smash{\mathop{\scriptstyle#1}\limits^{\scriptsize\raisebox{-3pt}{$\to$}}_{\scriptsize\raisebox{3pt}{$\leftarrow$}}}}%
}

\udc{531.39}

\rubric{Механика}

\titlerus{Исследование композитов в виде слоистых ортотропных оболочек}

\addtotocrus{Исследование композитов в виде слоистых ортотропных оболочек}

\titleeng{Investigation of a composites in the form of a layered orthotropic shell}

\addtotoceng{Investigation of a Composites in the Form of a Layered Orthotropic Shell}

\authorrus*[1]{Великанов}{Петр Геннадьевич}
\authoreng*[1]{Velikanov}{Peter G.}
\inforus{канд. физ.-мат. наук, доцент кафедры реактивных двигателей и энергетических установок Казанского национального исследовательского технического университета им.~А.Н.~Туполева\,--\,КАИ}
\email{pvelikanov@mail.ru}
\orcid{0000-0003-0845-2880}
\address{Служебный адрес: 420008, г. Казань, ул. Кремлевская, 35. Домашний адрес: 
420126, г.~Казань, ул.~Адоратского, 9; тел. (843)5170899; моб. тел. 
89503106140.}

\authorrus[2]{Артюхин}{Юрий Павлович}
\authoreng[2]{Artyukhin}{Yuri P.}[Yu.\,P.]
\inforus{д-р физ.-мат. наук, профессор кафедры теоретической механики Казанского (Приволжского) федерального университета}
\orcid{0000-0002-6243-9145}
\email{ArtukhinYP@mail.ru}

\affilrus[1]{\orgname{Казанский национальный исследовательский технический университет им.~А.Н.~Туполева~--~КАИ}, ул.~К.~Маркса, 10, \city{Казань}, 420111, \country{Россия}}
\affileng[1]{\orgname{Kazan National Research Technical University named after A.N.~Tupolev~--~KAI}, 10, K.~Marx st., \city{Kazan}, 420111, \country{Russia}}
\affilrus[2]{\orgname{Казанский (Приволжский) федеральный университет}, ул.~Кремлевская, 18, \city{Казань}, 420008, \country{Россия}}
\affileng[2]{\orgname{Kazan (Volga Region) Federal University}, 18, Kremlevskaya st., \city{Kazan}, 
420008, \country{Russia}}

\reviewer{Гольцев}{Аркадий Сергеевич}
\inforev{д-р физ.-мат. наук, профессор Донецкого национального университета}
\affilrev{\orgname{Донецкий национальный университет}, \city{Донецк}, \country{Россия}}
\emailrev{gas_6464@mail.ru}

\review{Статья, бесспорно, находится в пределах тематики журнала. Подобные статьи до сих пор мне известны не были. Статья актуальна для развития инженерных методов расчета композитов в виде слоистых ортотропных (трансверсально-изотропных и др.) оболочек в условиях различного преобладания жесткости армирования волокон. Думаю, что название статьи не до конца отражает ее содержание; в статье рассматривается несколько композитов с различными свойствами, а также не только цилиндрические, но и сферические оболочки. В этой связи, как мне кажется, стоит уточнить название статьи, например: “Исследование композитов в виде слоистых цилиндрических и сферических ортотропных оболочек” (если предложенное название покажется слишком длинным, то без потери смысла и для подчеркивания универсальности методики "цилиндрических и сферических" можно убрать). Аннотация в статье достаточно ясная и адекватная. Введение содержательное, вполне ясное и адекватное. Используемые в статье методы и методики вполне соответствуют задачам исследования. Результаты исследований вполне ясные и четкие. Некоторые используемые в статье понятия в научной литературе имеют равнозначные синонимы в своих названиях. Например, модуль упругости Юнга и модуль упругости первого рода, модуль сдвига и модуль упругости второго рода и т.д. М.б. авторам следует сначала (там, где впервые используются) прописать все понятия (одно вне скобок, а остальные в скобках), а затем использовать в статье лишь одни внескобочные понятия? Также в статье написано: “Отказ от гипотезы Кирхгофа-Лява приводит к созданию модели оболочки типа Тимошенко, имеющей дополнительные степени свободы… Во-первых, гипотез Кирхгофа-Лява несколько, во-вторых, отказ от гипотезы Кирхгофа-Лява приводит к созданию модели оболочки не только типа Тимошенко, но и Амбарцумяна, Агаловяна и многих других. Заключение достаточно ясное и четкое. Объем статьи вполне достаточный для ее цельного восприятия. Стиль изложения научный, содержательный и достаточно хорошо изложенный. Две содержащихся в статье таблицы вполне адекватны тексту (рисунков статья не содержит). Аббревиатуры, формулы, единицы измерения вполне соответствуют принятым стандартам. Библиография статьи вполне соответствует ее содержанию. Общая оценка статьи - хорошая. Рекомендую принять статью к публикации}

\annotationrus{Для исследования композитов в виде слоистых ортотропных оболочек в условиях различного преобладания жесткости армирования волокон продемонстрированы возможности ряда упрощающих методик по переходу от системы дифференциальных уравнений в частных производных к одному дифференциальному уравнению в частных производных относительно функции прогиба.}

\annotationeng{To study composites in the form of layered orthotropic shells under conditions of different predominance of fiber reinforcement stiffness, the possibilities of a number of simplifying techniques for the~transition from a system of partial differential equations to a single partial differential equation with respect to the deflection function are demonstrated. Studies show that the phenomenon of cross-shear compliance strongly affects the operation of structures made of orthotropic fiber-reinforced materials. In~particular, this applies to the problems of the effect of local loads on orthotropic shells, because near the concentrated load, the shell experiences significant transverse deformations. A technique has been demonstrated that allowed a system of five differential equations with respect to displacement components and rotation angles of the section to lead to one resolving equation with respect to deflection. From the~obtained equation for layered orthotropic shells, as a special case, an equation for layered transversally isotropic shells can be obtained. The test tasks were solved under conditions of different predominance of~fiber reinforcement stiffness. Also, with the help of a complex representation of the equations of the~theory of shells, the problem of bending a~flat orthotropic shell by a transverse load was solved.}

\keywordsrus{композиты, слоистые ортотропные оболочки, модель оболочки типа Тимошенко, 
пологие ортотропные оболочки}

\keywordseng{composites, layered orthotropic shells, Timoshenko-type shell model, 
shallow orthotropic shells}

\contributionrus{Вклад каждого соавтора в процесс написания статьи на разных этапах ее 
создания: идея работы (Артюхин Ю.П.), вычисления и расчеты (Артюхин Ю.П., 
Великанов П.Г.), написание статьи, внесение правок и утверждение 
окончательного варианта (Великанов П.Г.).}

\contributioneng{The contribution of each co-author to the process of writing an article at 
different stages of its creation: the idea of the work (Y.P.~Artyukhin), 
calculations (Y.P.~Artyukhin, P.G.~Velikanov), writing the article, making 
edits and approving the final version (Y.P.~Artyukhin, P.G.~Velikanov).}

\date{2-02-2024}
\revised{15-04-2024}
\accepted{19-04-2024}

\maketitle

\section*{Введение}

Разработка и расчет конструкций, сочетающих в себе легкость и экономичность, 
с одной стороны, и высокую прочность, жесткость, устойчивость и надежность, 
с другой стороны, является актуальным для современных ответственных слоистых 
тонкостенных элементов, применяемых в авиа-, судостроении, в химическом 
машиностроении и др. Для успешного сочетания вышеперечисленных свойств 
вполне оправданным представляется использование в~конструкциях ортотропных 
пластин и оболочек, изготовленных из композиционных материалов~\cite{B01,B02,B03,B04,B05,B06,B07,B08,B09,B10}.

Сечения, нормальные к срединной поверхности, тонкостенной оболочки, 
изготовленной из композиционных материалов, до деформации, в процессе 
деформации отклоняются от нормали на некоторый угол поперечного сдвига. Это 
связано тем, что оболочка имеет слоистую структуру, причем каждый слой при 
изгибе сдвигается относительно соседнего слоя из-за малой податливости 
связующего (смолы, клея) в касательном направлении. Чем больше податливость, 
тем хуже выполняются гипотезы Кирхгофа--Лява и тем обоснованнее становится 
использование модели оболочки типа Тимошенко. 

Исследования показывают, что явление податливости поперечному сдвигу сильно 
влияет на работу конструкции из ортотропных материалов, армированных 
волокнами. Вышесказанное является особенно важным для задач в условиях 
действия локальных нагрузок на ортотропные оболочки, т.к. вблизи 
сосредоточенной нагрузки оболочка испытывает значительные поперечные 
деформации.

В статье продемонстрированы возможности ряда упрощающих методик по переходу 
от системы дифференциальных уравнений в частных производных к одному 
дифференциальному уравнению в частных производных относительно прогиба для 
трансверсально-изотропных и~ортотропных оболочек.

Проверка правильности предложенных методик была продемонстрирована на 
исследовании прогибов свободно опертой по торцам ортотропной оболочки в 
условиях различного преобладания жесткости армирования волокон при действии 
нагрузки по малой площадке. Полученные результаты мало отличаются от 
результатов, полученных с помощью комплексных уравнений, что подтверждает их 
пригодность для инженерных расчетов в силу их компактности. Также приводится 
методика решения задачи изгиба пологой ортотропной оболочки под действием 
поперечной нагрузки с помощью комплексного представления уравнения теории 
оболочек.


\section{Предварительные сведения }

В тонкостенных конструкциях, изготовленных из композиционных материалов, 
сечения, нормальные к срединной поверхности до деформации, в процессе 
деформации отклоняются от нормали на некоторый угол, называемый углом 
поперечного сдвига. Это объясняется тем, что оболочка имеет слоистую 
структуру (волокна в эпоксидной смоле), причем каждый слой при изгибе 
сдвигается относительно соседнего слоя по причине малой податливости смолы 
(клея) в касательном направлении. Такая податливость характеризуется отношением модуля упругости Юнга (модуля упругости первого рода) к модулю поперечного сдвига (модулю упругости второго рода) $E_{j}/\tilde{{G}}_{j3}$ ($j=1, 2$), 
которая для стеклопластиков может быть равной от 10 до 50, а для 
боропластиков может доходить и до 100. Причем, чем больше это отношение, тем 
хуже выполняются гипотезы Кирхгофа--Лява. Учет существенной анизотропии 
материала требует отказа от принятых схем решения для изотропных и слабо 
анизотропных оболочек.

Исследования показывают, что явление податливости поперечному сдвигу сильно 
влияет на работу конструкции из ортотропных материалов, армированных 
волокнами. В частности, вышесказанное относится к задачам о действии 
локальных нагрузок на ортотропные оболочки, т.к. вблизи сосредоточенной 
нагрузки оболочка испытывает значительные поперечные деформации.

\section{Постановка задачи }

Отказ от кинематической (геометрической) гипотезы Кирхгофа-Лява приводит к созданию различных моделей оболочки, наиболее распространенной среди которых считается модель оболочки типа Тимошенко, имеющей дополнительные степени свободы --- два независимых угла 
поворота сечения $\phi $ и $\psi $
\begin{equation}
\label{eq1}
\phi =\gamma_{13} -\frac{1}{A_{1} }\frac{\partial w}{\partial \alpha_{1} 
}+\frac{u}{R_{1} };
\quad
\psi =\gamma_{23} -\frac{1}{A_{2} }\frac{\partial w}{\partial \alpha_{2} 
}+\frac{v}{R_{2} },
\end{equation}
где $A_{j} $ --- параметры Ляме; $R_{j} $ --- главные радиусы кривизны 
поверхности; $\gamma_{j3} $ --- осредненные по толщине углы поперечного 
сдвига между нормалью к срединной поверхности оболочки и~направлением $\alpha_{j}$ $( {j=1, 2})$; $u$, $v$ и $w$ --- тангенциальные перемещения и прогиб соответственно. Углы $\phi $ и $\psi $ представляют собой полные углы поворота нормального сечения оболочки в~результате изгиба и поперечного сдвига. Если эпюру касательных напряжений $\tau_{j3}$ ($j=1, 2$) аппроксимировать параболой (аналог формулы Журавского Д.И. для балок), то перерезывающие (поперечные) силы по закону Гука можно записать в виде
\begin{equation}
\label{eq2}
Q_{1} =K_{1} \gamma_{13} ;
\quad
Q_{2} =K_{2} \gamma_{23} ,
\end{equation}
где 
\[
K_{j} =\frac{5}{6}\tilde{{G}}_{j3} h \quad \left( {j=1,2} \right).
\]

Компоненты изменения кривизн в ортотропной цилиндрической оболочке ($A_{1} =A_{2} =R$; $\alpha_{1} =\alpha$; $\alpha_{2} =\beta$) представимы в виде
\begin{equation}
\label{eq3}
\kappa_{1} =\frac{1}{R}\frac{\partial \phi }{\partial \alpha };
\quad
\kappa_{2} =\frac{1}{R}\frac{\partial \psi }{\partial \beta };
\quad
2\tau =\frac{1}{R}\left( {\frac{\partial \phi }{\partial \beta 
}+\frac{\partial \psi }{\partial \alpha }} \right).
\end{equation}

Соотношения упругости для ортотропных оболочек имеют вид~\cite{B11} 
\[
T_{1} =B_{1} \left( {\varepsilon_{1} +\nu_{2} \varepsilon_{2} } \right) \ (\change{1, 2}) \ S=\tilde{{G}}h\omega ;
\]
\begin{equation}
\label{eq4}
M_{1} =D_{1} \left( {\kappa_{1} +\nu_{2} \kappa_{2} } \right)\ (\change{1, 2})\ H=\left( {\tilde{{G}}h^{3}/6} \right)\tau ,
\end{equation}
где 
\[
B_{1} =\frac{E_{1} h}{1-\nu_{1} \nu_{2} },
\quad
D_{1} =\frac{E_{1} h^{3}}{12\left( {1-\nu_{1} \nu_{2} } \right)}.
\]
Здесь $B_{1}$ и $D_{1}$~--- жесткости ортотропной оболочки на растяжение и изгиб соответственно; символ ($\change{1, 2}$) означает, что последующее 
выражение получается из предыдущего путем перестановки индексов; 
$\tilde{{G}}$~--- модуль сдвига; $E_{j} $, $\nu_{j} $~--- модули упругости и 
коэффициенты Пуассона $j$-го направления; $T_{j} $, $S(M_{j}, H)$~--- 
тангенциальные усилия (моменты) $j$-го направления; $\varepsilon_{j} $, $\omega$ --- тангенциальные деформации $j$-го направления.

К вышеприведенным соотношениям упругости (\ref{eq4}) необходимо добавить также 
уравнения (\ref{eq2}).

Уравнения равновесия круговой цилиндрической оболочки в координатах $\alpha$, $\beta $ запишутся следующим образом ($q$ --- нагрузка):
\begin{equation}
\label{eq5}
\frac{\partial T_{1} }{\partial \alpha }+\frac{\partial S}{\partial \beta 
}=0;
\quad
\frac{\partial T_{2} }{\partial \beta }+\frac{\partial S}{\partial \alpha 
}-Q_{2} =0;
\end{equation}
\[
\frac{\partial Q_{1} }{\partial \alpha }+\frac{\partial Q_{2} }{\partial 
\beta }+T_{2} =Rq;
\quad
\frac{\partial M_{1} }{\partial \alpha }+\frac{\partial H}{\partial \beta 
}=RQ_{1} ;
\quad
\frac{\partial M_{2} }{\partial \beta }+\frac{\partial H}{\partial \alpha 
}=RQ_{2} .
\]

Тангенциальные деформации выражаются через касательные перемещения $u$, $v$ и 
прогиб $w$
\begin{equation}
\label{eq6}
\varepsilon_{1} =\frac{1}{R}\frac{\partial u}{\partial \alpha };
\quad
\varepsilon_{2} =\frac{1}{R}\left( {\frac{\partial v}{\partial \beta }+w} 
\right);
\quad
\omega =\frac{1}{R}\left( {\frac{\partial u}{\partial \beta }+\frac{\partial 
v}{\partial \alpha }} \right).
\end{equation}

Если решать задачу в перемещениях и углах поворота, то система (\ref{eq1}), (\ref{eq2}), 
(\ref{eq4}), (\ref{eq5}), (\ref{eq6}) сводится к пяти дифференциальным уравнениям десятого 
порядка относительно $u$, $v$, $w$, $\phi$, $\psi $. Приведем эту систему к одному 
разрешающему уравнению относительно прогиба $w$~\cite{B12,B13}. Сделаем 
предварительно ряд упрощений, которые вносят погрешность, не превышающую 
погрешности, присущей теории тонких оболочек.

Интегрируя второе уравнение (\ref{eq5}) по $\beta $, найдем
\begin{equation}
\label{eq7}
T_{2} =\int {\left( {Q_{2} -\frac{\partial S}{\partial \alpha }} \right)\d\beta } .
\end{equation}

В этом уравнении $Q_{2} $ является второстепенным по сравнению с остальными 
членами и~поэтому может быть опущен (он характеризует непологость оболочки). 
Его влияние посредством выражения (\ref{eq7}) будет отражено в последующем в 
третьем уравнении (\ref{eq5}). Систему (\ref{eq5}) заменим следующей:
\begin{equation}
\label{eq8}
\frac{\partial T_{1} }{\partial \alpha }+\frac{\partial S}{\partial \beta 
}=0;
\quad
\frac{\partial T_{2} }{\partial \beta }+\frac{\partial S}{\partial \alpha 
}=0;
\end{equation}
\[
\frac{\partial^{2}M_{1} }{\partial \alpha^{2}}+2\frac{\partial 
^{2}H}{\partial \alpha \partial \beta }+\frac{\partial^{2}M_{2} }{\partial 
\beta^{2}}-RT_{2} =-R^{2}q.
\]

Первым двум уравнениям (\ref{eq8}) удовлетворим с помощью функции усилий $F$, 
определяемой выражениями
\begin{equation}
\label{eq9}
T_{1} =\frac{1}{R^{2}}\frac{\partial^{2}F}{\partial \beta^{2}},
\quad
T_{2} =\frac{1}{R^{2}}\frac{\partial^{2}F}{\partial \alpha^{2}},
\quad
S=-\frac{1}{R^{2}}\frac{\partial^{2}F}{\partial \alpha \partial \beta }.
\end{equation}

Учитывая (\ref{eq7}), (\ref{eq9}) и допуская, что $M_{2} $ намного больше $H$, запишем 
третье уравнение (\ref{eq8}) в виде
\begin{equation}
\label{eq10}
\frac{\partial^{2}M_{1} }{\partial \alpha^{2}}+2\frac{\partial 
^{2}H}{\partial \alpha \partial \beta }+\frac{\partial^{2}M_{2} }{\partial 
\beta^{2}}+M_{2} -\frac{1}{R}\frac{\partial^{2}F}{\partial \alpha 
^{2}}=-R^{2}q.
\end{equation}
Здесь член уравнения $M_{2} $ учитывает непологость оболочки.

Преобразуем уравнения совместности деформаций
\[
\frac{\partial^{2}\varepsilon_{1} }{\partial \beta^{2}}-\frac{\partial 
^{2}\omega }{\partial \alpha \partial \beta }+\frac{\partial^{2}\varepsilon 
_{2} }{\partial \alpha^{2}}=\frac{1}{R}\frac{\partial^{2}w}{\partial 
\alpha^{2}},
\]
подставляя в него соотношения упругости
\[
\varepsilon_{1} =\frac{1}{E_{1} hR^{2}}\left( {\frac{\partial 
^{2}F}{\partial \beta^{2}}-\nu_{1} \frac{\partial^{2}F}{\partial \alpha 
^{2}}} \right);
\quad
\varepsilon_{2} =\frac{1}{E_{2} hR^{2}}\left( {\frac{\partial 
^{2}F}{\partial \alpha^{2}}-\nu_{2} \frac{\partial^{2}F}{\partial \beta 
^{2}}} \right);
\]
\[
\omega =-\frac{1}{\tilde{{G}}hR^{2}}\frac{\partial^{2}F}{\partial \alpha 
\partial \beta }.
\]
Тогда получим
\begin{equation}
\label{eq11}
\frac{\partial^{4}F}{\partial \alpha^{4}}+2\lambda_{1} \frac{\partial 
^{4}F}{\partial \alpha^{2}\partial \beta^{2}}+\theta \frac{\partial 
^{4}F}{\partial \beta^{4}}=RE_{2} h\frac{\partial^{2}w}{\partial \alpha 
^{2}},
\end{equation}
где 
\[
2\lambda_{1} =\frac{E_{2} }{\tilde{{G}}}-2\nu_{2}; \quad \theta =\delta 
=\frac{E_{2} }{E_{1} }.
\]

Введем новые функции $\xi $ и $\eta $, связанные с функциями углов поворота 
следующим образом:
\begin{equation}
\label{eq12}
\phi =\frac{\partial \xi }{\partial \alpha }+\frac{\partial \eta }{\partial 
\beta };
\quad
\psi =\frac{\partial \xi }{\partial \beta }-\frac{\partial \eta }{\partial 
\alpha },
\end{equation}
где $\xi $ --- потенциальная (деформационная) составляющая углов поворота, 
$\eta $ --- вихревая составляющая углов поворота (характеризует жесткий 
поворот).

Рассмотрим сначала случай, когда оболочка изготовлена из 
трансверсально"=изотропного материала. Тогда
\[
K_{1} =K_{2} ;
\quad
\lambda_{1} =\theta =1;
\quad
\nu_{1} =\nu_{2} =\nu ;
\quad
D_{1} =D_{2} =D.
\]
При этом уравнение (\ref{eq11}) перейдет в следующее:
\begin{equation}
\label{eq13}
\nabla^{4}F-REh\frac{\partial^{2}w}{\partial \alpha^{2}}=0.
\end{equation}

Используя соотношения упругости и (\ref{eq12}), из (\ref{eq10}) получим
\begin{equation}
\label{eq14}
\mu_{1} \nabla^{4}\xi =\nabla^{2}\xi +\left( {\nabla^{2}+1} 
\right)\frac{w}{R};
\quad
\mu_{1} =\frac{D_{1} }{K_{1} R^{2}}.
\end{equation}

Два других уравнения относительно $\xi ,\eta $ получим из уравнений моментов 
(\ref{eq5}), выразив их через $\phi ,\psi ,w$ с учетом (\ref{eq12}), используя во 
второстепенных членах уравнения приближенное равенство (условие 
нерастяжимости кольца $\varepsilon_{2} \approx 0)$
\[
v\approx -\int {w\d\beta } .
\]

Затем, дифференцируя по $\alpha $ и $\beta $ полученные выражения, складывая 
и вычитая одно из другого, найдем
\begin{equation}
\label{eq15}
\frac{1-\nu }{2}\mu_{1} \nabla^{4}\eta =\nabla^{2}\eta -\frac{1}{R}\int 
{\frac{\partial w}{\partial \alpha }\d\beta } ;
\end{equation}
\[
\nabla^{4}\xi +\frac{\partial^{2}\xi }{\partial \beta^{2}}+\nu 
\frac{\partial^{2}\xi }{\partial \alpha^{2}}-\frac{1}{D}\frac{\partial 
^{2}F}{\partial \alpha^{2}}=-\frac{R^{3}q}{D}+\left( {1-\nu } 
\right)\frac{\partial^{2}\eta }{\partial \alpha \partial \beta }.
\]

Сделаем оценку членов в уравнениях (\ref{eq14}), (\ref{eq15})
\[
\nabla^{4}\xi \sim \nabla^{2}w;
\quad
\frac{\partial^{2}\eta}{\partial \alpha \partial \beta } \sim \nabla^{-4}\frac{\partial^{2}w}{\partial \alpha^{2}}.
\]

Следовательно, влияние функции $\xi $ в уравнении (\ref{eq15}) по сравнению с 
функцией $\eta $ таково же, как соотношение операторов $\nabla^{6}$ и 
$\left( {\ldots} \right)_{\alpha \alpha } $. Ввиду того, что функция прогибов 
в~цилиндрической оболочке является существенно возрастающей при 
дифференцировании по окружной координате, влиянием функции $\eta $ в 
последнем уравнении (\ref{eq15}) можно пренебречь. Исключая затем из полученных 
уравнений функции $F$ и $\xi $, получим разрешающее уравнение относительно 
прогиба
\[
\nabla^{6}\left( {\nabla^{2}+1} \right)^{2}w- \dashuline{\left( {1-\nu } \right)\nabla^{4}\left( {\nabla^{2}+1} \right)\frac{\partial^{2}w}{\partial \alpha^{2}}}-4b^{4}\left( {\mu_{1} \nabla^{2}-1} \right)\frac{\partial^{4}}{\partial \alpha^{4}}\nabla^{2}w=-\frac{R^{4}}{D}\nabla^{6}\left( {\mu_{1} \nabla^{2}-1} \right)q.
\]

Как показано Морли~\cite{B14} подчеркнутым членом в этом уравнении на основании 
вышесказанного можно пренебречь.
Поэтому окончательно получаем
\begin{equation}
\label{eq16}
\nabla^{4}\left( {\nabla^{2}+1} \right)^{2}w-4b^{4}\left( {\mu_{1} \nabla 
^{2}-1} \right)\frac{\partial^{4}w}{\partial \alpha 
^{4}}=-\frac{R^{4}}{D}\nabla^{4}\left( {\mu_{1} \nabla^{2}-1} 
\right)q+f;
\end{equation}
\[
\nabla^{2}f=0;
\quad
4b^{4}=12\left( {1-\nu^{2}} \right)\;\gamma^{2}.
\]

Уравнение (\ref{eq16}) при $\mu_{1} =0$, $f=0$ переходит в уравнение Морли~\cite{B14}, при 
отбрасывании единицы в первом члене уравнения, которая существенна для 
длинных оболочек,~--- в уравнение Нагди~\cite{B15}.

По аналогии со случаем трансверсально-изотропного тела проведем такие же 
преобразования для ортотропной оболочки. Принимая во внимание 
вышеприведенные упрощения, получим
\begin{multline}\label{eq17}
 \nabla_{2}^{\ast } \left[ {\nabla_{1}^{\ast } +\theta \left( {\frac{\partial^{2}}{\partial \beta^{2}}+\nu_{1} \frac{\partial^{2}}{\partial \alpha^{2}}} \right)} \right]\left( {\nabla^{2}+1} \right)w-4b_{1}^{4} \left( {\mu_{1} \nabla_{K}^{4} -\nabla^{2}} \right)\frac{\partial^{4}w}{\partial \alpha^{4}}= \\
 = -\frac{R^{4}}{D_{1} }\nabla_{2}^{\ast } \left( {\mu_{1} \nabla_{K}^{4} -\nabla^{2}} \right)q,
\end{multline}
где 
\[
4b_{1}^{4} =12\left( {1-\nu_{1} \nu_{2} } \right)\theta \gamma^{2};
\quad 
\nabla_{K}^{4} =\frac{\partial^{4}}{\partial \alpha^{4}}+\left( {1+K_{0} 
} \right)\delta_{1} \frac{\partial^{4}}{\partial \alpha^{2}\partial \beta 
^{2}}+\theta K_{0} \frac{\partial^{4}}{\partial \beta^{4}};
\] 
\[
\nabla_{1}^{\ast } =\frac{\partial^{4}}{\partial \alpha^{4}}+2\delta_{1} 
\frac{\partial^{4}}{\partial \alpha^{2}\partial \beta^{2}}+\theta 
\frac{\partial^{4}}{\partial \beta^{4}};
\quad
\nabla_{2}^{\ast } =\frac{\partial^{4}}{\partial \alpha^{4}}+2\lambda 
_{1} \frac{\partial^{4}}{\partial \alpha^{2}\partial \beta^{2}}+\theta 
\frac{\partial^{4}}{\partial \beta^{4}};
\]
\[
\nabla_{1}^{4} =\nabla_{1}^{\ast } +\theta \frac{\partial^{2}}{\partial 
\beta^{2}};
\quad
\nabla_{2}^{4} =\nabla_{2}^{\ast } +\theta \frac{\partial^{2}}{\partial 
\beta^{2}};
\quad
K_{0} =\frac{K_{1} }{K_{2} };
\quad
\delta_{1} =\nu_{2} +\frac{2\tilde{{G}}\left( {1-\nu_{1} \nu_{2} } 
\right)}{E_{1} }.
\]

Сделаем следующие преобразования на основе соображений, описанных при выводе 
уравнения (\ref{eq16})
\[
\nabla_{2}^{\ast } \left[ {\nabla_{1}^{\ast } +\theta \left( 
{\frac{\partial^{2}}{\partial \beta^{2}}+\nu_{1} \frac{\partial 
^{2}}{\partial \alpha^{2}}} \right)} \right]\left( {\nabla^{2}+1} 
\right)\approx \nabla^{2}\nabla_{1}^{4} \nabla_{2}^{4} ;
\quad
\nabla_{K}^{4} \approx \nabla^{2}\left( {\frac{\partial^{2}}{\partial 
\alpha^{2}}+\theta K_{0} \frac{\partial^{2}}{\partial \beta^{2}}} 
\right).
\]
Тогда уравнение (\ref{eq17}) примет окончательный вид
\begin{multline}
\label{eq18}
\nabla_{1}^{4} \nabla_{2}^{4} w-4b_{1}^{4} \left[ {\mu_{1} \left( 
{\frac{\partial^{2}}{\partial \alpha^{2}}+\theta K_{0} \frac{\partial 
^{2}}{\partial \beta^{2}}} \right)-1} \right]\frac{\partial^{4}w}{\partial 
\alpha^{4}}=\\
=-\frac{R^{4}}{D_{1} }\nabla_{2}^{\ast } \left( {\mu_{1} 
\left( {\frac{\partial^{2}}{\partial \alpha^{2}}+\theta K_{0} 
\frac{\partial^{2}}{\partial \beta^{2}}} \right)-1} \right)q+f;
\end{multline}
\[
\nabla^{2}f=0.
\]

Для проверки уравнения (\ref{eq18}) были проведены вычисления прогибов свободно 
опертой по торцам ортотропной оболочки при действии нагрузки по малой 
площадке: 

при $\alpha =0$; $\alpha =\alpha_{1} =L/R$: $w=0$; $M_{1} =0$; 
$\psi =0$; $v=0$; $T_{1} =0$.

Прогибы под действием сосредоточенной силы имеют логарифмическую 
особенность, поэтому приходится распределять равномерно эту нагрузку $q$ по 
площади $2a\times 2\beta_{1} R$ с центром $\xi_{0} =\frac{x_{0} }{R}$.

Решение для прогиба в этом случае имеет вид
\begin{multline}\label{eq19}
 w^{\ast }\left( {\alpha ,\beta ,\xi_{0} } \right)=-\frac{12\left( {1-\nu 
_{1} \nu_{2} } \right)\gamma^{3}}{\pi \alpha_{1} }\left\{ 
{\sum\limits_{m=1}^\infty {\left[ {w_{m0} +\sum\limits_{n=1}^\infty {2w_{mn} 
\frac{\sin \left( {n\beta_{1} } \right)}{n\beta_{1} }\cos \left( {n\beta } 
\right)} } \right]} } \right.\times \\ 
 \times \left. {\sin \left( {\lambda_{m} \xi_{0} } \right)\frac{\sin 
\left( {\lambda_{m} a_{1} } \right)}{\lambda_{m} a_{1} }\sin \left( 
{\lambda_{m} \alpha } \right)} \right\} , 
\end{multline}
где 
\[
w^{\ast }=\frac{wE_{1} }{4qa\beta_{1} };
\quad
a_{1} =\frac{a}{R};
\quad 
\lambda_{m} =\frac{m\pi }{\alpha_{1} };
\]
\begin{multline*}
w_{mn} =\left( {\lambda_{m}^{4} +2\lambda_{1} \lambda_{m}^{2} 
n^{2}+\theta n^{4}} \right)\left[ {\mu_{1} \left( {\theta K_{0} 
n^{2}+\lambda_{m}^{2} } \right)+1} \right]\div \\ 
 \div \left\{ {\left[ {\lambda_{m}^{4} +2\delta_{1} \lambda_{m}^{2} 
n^{2}+\theta n^{2}\left( {n^{2}-1} \right)} \right]} \right.\left[ {\lambda 
_{m}^{4} +2\lambda_{1} \lambda_{m}^{2} n^{2}+\theta n^{2}\left( {n^{2}-1} 
\right)} \right]+ \\ 
 +\left. {4b_{1}^{4} \left[ {\mu_{1} \left( {\lambda_{m}^{2} +\theta K_{0} 
n^{2}} \right)+1} \right]\lambda_{m}^{4} } \right\} , 
\end{multline*}
для случая трансверсально-изотропной оболочки
\[
w_{mn} =\frac{\left( {\lambda_{m}^{2} +n^{2}} \right)^{2}\left[ {\mu_{1} 
\left( {\lambda_{m}^{2} +n^{2}} \right)+1} \right]}{\left[ {\left( {\lambda 
_{m}^{2} +n^{2}} \right)^{2}-n^{2}} \right]^{2}+4b_{1}^{4} \left[ {\mu_{1} 
\left( {\lambda_{m}^{2} +n^{2}} \right)+1} \right]\lambda_{m}^{4} }.
\]

При отсутствии поперечных сдвигов $\mu_{1} =0$ и действии сосредоточенной 
нагрузки $P$ из (\ref{eq19}) следует:
\[
\lim\limits_{\beta_{1} \to 0} \frac{\sin \left( {n\beta_{1} } 
\right)}{n\beta_{1} }=1;
\quad
\lim\limits_{a_{1} \to 0} \frac{\sin \left( {\lambda_{m} a_{1} } 
\right)}{\lambda_{m} a_{1} }=1;
\]
\[
f=0\text{ и }\xi_{0} =\frac{\alpha_{1} }{2}; \quad \alpha =\frac{\alpha_{1} }{2};
\quad \beta =0;
\]
\begin{equation}
\label{eq20}
w_{0M} =\frac{wE_{1} R}{P}=-\frac{12\left( {1-\nu_{1} \nu_{2} } 
\right)\gamma^{3}}{\pi \alpha_{1} }\sum\limits_{m=1, 3, 5,\ldots}^\infty 
{\sum\limits_{n=0}^\infty {\left( {2-\delta_{0}^{n} } \right)\bar{{w}}_{mn} 
} } ,
\end{equation}
где 
\[
\bar{{w}}_{mn} =\frac{\sin^{2}\left( {m\pi/2} \right)\left( 
{\lambda_{m}^{4} +2\lambda_{1} \lambda_{m}^{2} n^{2}+\theta n^{4}} 
\right)}{\left[ {\lambda_{m}^{4} +2\delta_{1} \lambda_{m}^{2} 
n^{2}+\theta n^{2}\left( {n^{2}-1} \right)} \right]\left[ {\lambda_{m}^{4} 
+2\lambda_{1} \lambda_{m}^{2} n^{2}+\theta n^{2}\left( {n^{2}-1} \right)} 
\right]+4b_{1}^{4} \lambda_{m}^{4} }.
\]

Проведем вычисления ряда (\ref{eq20}) для однонаправленного композита на основе 
табл.~\ref{tab1} при следующей геометрии $R/h=\gamma =100$, $\xi_{1} 
=L/R=1$, $m_{\max } =n_{\max } =191$, что дает при $\delta <1$ три 
верных знака, а при $\delta >1$ четыре верных знака (операции с успехом 
реализуемые, например, в пакете символьной математики Wolfram Mathematica~\cite{B16,B17}).

\begin{table}
\caption{Свойства однонаправленных композитов на основе эпоксидной смолы (волокна 
занимают порядка 60{\%} всего объема композита)~\cite{B18}}
\begin{tabularx}{\textwidth}{|P{30mm}|Y|Y|Y|Y|Y|Y|}
\hline
& 
$E_{1}$, ГПа& 
$E_{2}$, ГПа& 
$\nu_{1} $& 
$\nu_{2} $& 
$\tilde{{G}}$, ГПа& 
$N_\text{матер} $ \\
\hline
Углепластик (волокна AS)& 
140& 
8,96& 
0,3& 
0,0192& 
7,1& 
1 \\
\hline
Углепластик (волокна IM6)& 
200& 
11,1& 
0,32& 
0,01776& 
8,35& 
2 \\
\hline
Органопластик (волокна кевлар-49)& 
76& 
5,5& 
0,33& 
0,023882& 
2,33& 
3 \\
\hline
\end{tabularx}
\label{tab1}
\end{table}

\begin{table}
\caption{Сравнение прогибов для цилиндрической оболочки в условиях различного 
преобладания жесткости армирования волокон}
\begin{tabularx}{\textwidth}{|Y|P{19mm}|P{21mm}|P{20mm}|P{15mm}|P{18mm}|P{18mm}|}
\hline
$N_\text{матер} $& 
$1\left( {\delta =0,064} \right)$& 
$1\left( {\delta =15,625} \right)$& 
$2\left( {\delta =0,056} \right)$& 
$2\left( {\delta =18} \right)$& 
$3\left( {\delta =0,07} \right)$& 
$3\left( {\delta =13,8} \right)$ \\
\hline
$\left| {w_{0M} } \right|$& 
86095& 
2652,7& 
98013& 
2497,4& 
95346& 
3307 \\
\hline
$\left| {w_{0} } \right|$& 
86045& 
2651,8& 
97932& 
2496,1& 
95141& 
3303 \\
\hline
\end{tabularx}
\label{tab2}
\end{table}

Результаты вычислений материалов из таб.~\ref{tab1} представлены в табл.~\ref{tab2}. Эти 
результаты мало отличаются от результатов, полученных с помощью комплексных 
уравнений, что подтверждает их пригодность для инженерных расчетов в силу их 
компактности.

Уравнения (\ref{eq18}), (\ref{eq19}) представляют удобство для определения 
напряженно"=деформированного состояния оболочки, в которой можно не учитывать 
влияние торцов. Достаточно найти частное решение (\ref{eq19}), удовлетворяющее 
уравнению (\ref{eq18}) и условию затухания прогиба по длине оболочки, причем можно 
положить $f=0$. Тогда порядок уравнений будет равен восьми, как и в 
классической теории оболочек.

Форма уравнений (\ref{eq18}) позволяет делать вычисления для любых значений $\mu 
_{1} $, в том числе и для значения $\mu_{1} =0$. Исходная система уравнений 
(\ref{eq5}) не позволяла этого сделать в виду предельного перехода $K_{j} \to 
\infty $, $\gamma_{j3} \to 0$, $\left( {j=1, 2} \right)$.

Вычисления по формуле (\ref{eq19}) показывают, что влияние поперечных сдвигов на 
увеличение прогибов существенны для более толстых (средней толщины) оболочек 
$\gamma \leqslant 50$ и $\mu_{1} \geqslant 10$. Причем значения эти превосходят 
максимальные прогибы при $\mu_{1} =0$ в несколько раз. Например, для 
$\gamma =40$, $\mu_{1} =40$ это увеличение составляет 1,5, а при $\gamma 
=20$, $\mu_{1} =40$ уже 2,1, при $\gamma =20$, $\mu_{1} =60$ отношение 
равно 2,7, при $\gamma =10$, $\mu_{1} =40$ отношение равно 3,9.

В то же время, для тонких оболочек $\gamma \geqslant 100$ влияние поперечных 
сдвигов незначительно.

\section{Изгиб пологой ортотропной оболочки поперечной нагрузкой}

Пусть пологая ортотропная оболочка изгибается произвольной поперечной 
нагрузкой $q_{3} $. Решение задачи о жесткости такой оболочки сводится к 
отысканию функции $\tilde{{F}}$ из уравнения вида~\cite{B11,B19,B20}
\begin{equation}
\label{eq21}
\frac{\partial^{4}\tilde{{F}}}{\partial \xi^{4}}+2\lambda \;k_{\ast }^{2} 
\frac{\partial^{4}\tilde{{F}}}{\partial \xi^{2}\partial \eta^{2}}+\delta k_{\ast }^{4} \frac{\partial^{4}\tilde{{F}}}{\partial \eta^{4}}+\frac{ia^{2}}{c}\left( {k_{2} \frac{\partial^{2}\tilde{{F}}}{\partial 
\xi^{2}}+k_{1} k_{\ast }^{2} \frac{\partial^{2}\tilde{{F}}}{\partial \eta^{2}}} \right)+2\varepsilon k_{\ast }^{2} \frac{\partial 
^{4}\bar{{\tilde{{F}}}}}{\partial \xi^{2}\partial \eta 
^{2}}=-i\frac{a^{4}q_{3} }{c}.
\end{equation}
Здесь введены безразмерные координаты
\[
\xi =\frac{x}{a};
\quad
\eta =\frac{y}{b};
\quad
k_{\ast } =\frac{a}{b};
\]
$a$, $b$~--- размеры оболочки в плане.

Зная $\tilde{{F}}$, можно найти комплексные усилия
\[
\tilde{{T}}_{1} =-\frac{1}{b^{2}}\frac{\partial^{2}\tilde{{F}}}{\partial 
\eta^{2}};
\quad
\tilde{{T}}_{2} =-\frac{1}{a^{2}}\frac{\partial^{2}\tilde{{F}}}{\partial 
\xi^{2}};
\quad
\tilde{{S}}=\frac{1}{ab}\frac{\partial^{2}\tilde{{F}}}{\partial \xi 
\partial \eta },
\]
причем 
\[
w=-\frac{1}{\mu }\Im\tilde{{F}};
\quad
T_{1}=-\frac{1}{b^{2}}\Re\frac{\partial^{2}\tilde{{F}}}{\partial \eta^{2}}; 
\quad
T_{2} =-\frac{1}{a^{2}}\Re\frac{\partial^{2}\tilde{{F}}}{\partial \xi^{2}};
\quad
S=\frac{1}{ab}\Re\frac{\partial^{2}\tilde{{F}}}{\partial \xi \partial \eta }.
\]

Один из недостатков комплексного представления уравнения теории оболочек --- 
это трудность формулировки граничных условий в комплексном виде. Для 
некоторых видов граничных условий это удается сделать.

При расчете оболочек часто приходится иметь дело с задачей сопряжения двух 
оболочек. Если границей сопряжения является линия $\xi =\const$, то на этой 
линии должны выполняться условия непрерывности восьми величин: $u$, $v$, $w$, угла 
поворота $\partial w/\partial \xi$, усилий $S$, $T_{1} $, момента 
$M_{1} $ и обобщенной перерезывающей (поперечной) силы. Это требование может 
быть заменено четырьмя условиями для комплексных величин, требуя 
непрерывности $\tilde{{F}}$, $\partial \tilde{F}/\partial \xi$, $\partial^{2}\tilde{F}/\partial \xi^{2}$, $\partial^{3}\tilde{F}/\partial \xi^{3}$.

В случае свободного опирания точек контура оболочки достаточно выполнить на 
краю $\xi =\const$ условия
\begin{equation}
\label{eq22}
\tilde{{F}}=\frac{\partial^{2}\tilde{{F}}}{\partial \xi^{2}}=0,
\end{equation}
которые дают
\[
w=\frac{\partial^{2}w}{\partial \xi^{2}}=0;
\quad
T_{1} =T_{2} =0;
\quad
S\ne 0.
\]

Условия при $\xi =\const$
\begin{equation}
\label{eq23}
\tilde{{F}}=\frac{\partial \tilde{{F}}}{\partial \xi }=0
\end{equation}
позволяют удовлетворить скользящему защемлению
\[
w=\frac{\partial w}{\partial \xi }=T_{1} =S=0;
\quad
u\ne 0,
\quad
v\ne 0.
\]

Предполагая края оболочки свободно опертыми, построим решение уравнения 
(\ref{eq21}) в виде двойного тригонометрического ряда~\cite{B21}
\begin{equation}
\label{eq24}
\tilde{{F}}=\sum\limits_{m=1}^\infty {\sum\limits_{n=1}^\infty {F_{mn} } } 
\sin \left( {m\pi \xi } \right)\sin \left( {n\pi \eta } \right).
\end{equation}

Нагрузка представляется разложением
\[
q_{3} =\sum\limits_{m=1}^\infty {\sum\limits_{n=1}^\infty {q_{mn} } } \sin 
\left( {m\pi \xi } \right)\sin \left( {n\pi \eta } \right),
\]
где 
\[
q_{mn} =4\int\limits_0^1 {\int\limits_0^1 {q_{3} \sin \left( {m\pi \xi 
} \right)\sin \left( {n\pi \eta } \right) } } \d\xi \d\eta .
\]

Если на оболочку в точке $\left( {\xi_{0} ,\eta_{0} } \right)$ действует 
сила $P$, то 
\[
q_{mn} =\frac{4P}{ab}\sin \left( {m\pi \xi_{0} } \right)\sin \left( {n\pi 
\eta_{0} } \right).
\]

Для равномерной нагрузки эта величина равна
\[
q_{mn} =\frac{16q}{mn\pi^{2}}\left( {m, n=1, 3, 5, \ldots} \right).
\]

Коэффициенты ряда (\ref{eq24}) имеют вид
\begin{equation}
\label{eq25}
F_{mn} =\frac{c_{mn} }{\Delta_{mn} }\left[ {a_{mn}^{(2)} -i\left( 
{a_{mn}^{(1)} +b_{mn} } \right)} \right],
\end{equation}
где 
\[
a_{mn}^{(1)} =\pi^{4}\left( {m^{4}+2\lambda k_{\ast }^{2} m^{2}n^{2}+\delta k_{\ast }^{4} n^{4}} \right);
\quad
a_{mn}^{(2)} =\frac{\pi ^{2}a^{2}}{c}\left( {k_{2} m^{2}+k_{1} k_{\ast }^{2} n^{2}} \right);
\] 
\[
b_{mn} =2\varepsilon k_{\ast }^{2} m^{2}n^{2}\pi^{4};
\quad
\Delta_{mn} =\left( {a_{mn}^{(1)} } \right)^{2}+\left( {a_{mn}^{(2)} } 
\right)^{2}-b_{mn}^{2} ;
\quad
c_{mn} =\frac{a^{4}q_{mn} }{c}.
\]

Для гладкой нагрузки ряд (\ref{eq24}) сходится довольно быстро и для практических 
расчетов достаточно взять несколько членов ряда. Иначе ведет себя решение в 
случае сосредоточенной нагрузки: сходимость ряда оказывается настолько 
медленной, что $m$ и $n$ достигают сотен членов. При определении моментов 
ряды придется дважды дифференцировать, что еще больше ухудшает его 
сходимость. Причем в точке приложения нагрузки ряд для моментов будет 
расходящимся. Такое несоответствие с напряжениями в реальной конструкции, 
вытекающее из-за абстрагирования нагрузки в виде точечной силы, можно 
устранить распределением силы по некоторой малой площадке, что и будет 
соответствовать реальному нагружению.

Наличие в уравнении (\ref{eq21}) комплексно-сопряженной функции затрудняет решение 
задачи при краевых условиях, отличных от свободного опирания. Желательно 
выяснить значимость члена с $\varepsilon $. Решим предыдущую задачу другим 
методом. Применим метод последовательных приближений, считая в первом 
приближении $\varepsilon =0$, уточняя правую часть (\ref{eq21}) в последующих 
приближениях.

В первом приближении получаем
\[
F_{mn}^{(1)} =-\frac{ic_{mn} }{a_{mn}^{(1)} -ia_{mn}^{(2)} };
\]
во втором приближении:
\[
F_{mn}^{(2)} =-\frac{ic_{mn} \left( {1+g_{mn} } \right)}{a_{mn}^{(1)} 
-ia_{mn}^{(2)} }, \text{ где } g_{mn} =\frac{b_{mn} }{a_{mn}^{(1)} +ia_{mn}^{(2)} };
\]
в третьем приближении будем иметь
\[
F_{mn}^{(3)} =F_{mn}^{(1)} \left( {1+g_{mn} +g_{mn} \bar{{g}}_{mn} } 
\right).
\]

Увеличивая число приближений до бесконечности, можно составить следующий 
бесконечный ряд
\[
F_{mn} =F_{mn}^{(2)} \sum\limits_{s=0}^\infty {\left( {g_{mn} \bar{{g}}_{mn} 
} \right)^{s}} .
\]
Замечая, что
\[
g_{mn} \bar{{g}}_{mn} =\frac{4\varepsilon^{2}k_{\ast }^{4} m^{4}n^{4}\pi 
^{8}}{\left( {a_{mn}^{(1)} } \right)^{2}+\left( {a_{mn}^{(2)} } 
\right)^{2}}<\left( {\frac{\varepsilon }{\lambda }} \right)^{2}<1,
\]
ряд дает сумму
\[
F_{mn} =\frac{F_{mn}^{(2)} }{1-g_{mn} \bar{{g}}_{mn} }=-\frac{ic_{mn} 
}{\Delta_{mn} }\left( {a_{mn}^{(1)} +b_{mn} +ia_{mn}^{(2)} } \right),
\]
что не отличается от (\ref{eq25}).

Вычисления максимального прогиба под равномерной нагрузкой (операции с 
успехом реализуемые, например, в пакете символьной математики 
Wolfram Mathematica~\cite{B16,B17}), действующей на квадратную в плане пологую 
сферическую оболочку, изготовленную из углепластика с волокнами AS дают во 
втором приближении погрешность 2~{\%}, в третьем приближении всего 0,5~{\%}.

\section*{Заключение}

В статье было проведено исследование ряда упрощающих методик по переходу от 
системы дифференциальных уравнений в частных производных к одному 
дифференциальному уравнению в частных производных относительно прогиба для 
трансверсально-изотропных и~ортотропных оболочек.

Отработка предложенной методики была реализована при вычислении прогибов 
свободно опертой по торцам ортотропной оболочки в условиях различного 
преобладания жесткости армирования волокон при действии нагрузки по малой 
площадке. Полученные результаты мало отличаются от результатов, полученных с 
помощью комплексных уравнений, что подтверждает их пригодность для 
инженерных расчетов в силу их компактности. Также приводится методика 
решения задачи изгиба пологой ортотропной оболочки под действием поперечной 
нагрузки с помощью комплексного представления уравнения теории оболочек. 
Вычисления максимального прогиба под равномерной нагрузкой, действующей на 
квадратную в плане пологую ортотропную сферическую оболочку, изготовленную 
из углепластика с волокнами AS, уже в третьем приближении дают погрешность 
0,5~{\%}.

\begin{thebibliography}{21}
\bibitem{B01}
Артюхин, Ю.П., Грибов, А.П., \textit{Решение задач нелинейного деформирования пластин и пологих оболочек методом граничных элементов}. Казань, Фэн, 2002. \altbib{Artyukhin, Yu.P., Gribov, A.P., \textit{Reshenie zadach nelineynogo deformirovaniya plastin i pologikh obolochek metodom granichnykh elementov = Solving problems of nonlinear deformation of plates and flat shells by the boundary element method}. Kazan, Feng, 2002. (in Russian)}

\bibitem{B02}
Великанов, П.Г., Метод граничных интегральных уравнений для решения задач изгиба изотропной пластины, лежащей на сложном двухпараметрическом упругом основании. \textit{Известия Саратовского университета. Серия <<Математика. Механика. Информатика>>}, 2008, т.~8, вып.~1, с.~36--42. \altbib{Velikanov, P.G., The method of boundary integral equations for solving bending problems of an isotropic plate lying on a complex two-parameter elastic base. \textit{Izvestiya Saratovskogo universiteta. Seriya ``Matematika. Mekhanika. Informatika'' = Proc. of the Saratov University. The series ``Mathematics. Mechanics. Informatics''}, 2008, vol.~8, iss.~1, pp.~36--42. (in Russian)}

\bibitem{B03}
Великанов, П.Г., Артюхин, Ю.П., Куканов, Н.И., Решение задачи изгиба анизотропной пластины методом граничных элементов. В сб. \textit{Материалы Всероссийской научной конференции <<Актуальные проблемы механики сплошных сред -- 2020>>}, с.~105--111. \altbib{Velikanov, P.G., Artyukhin, Yu.P., Kukanov, N.I., Тhe solution of an anisotropic plate bending problem by the boundary element method. \textit{Materialy Vserossiyskoy nauchnoy konferentsii ``Aktual'nye problemy mekhaniki sploshnykh sred -- 2020'' = Proc. of the All-Russian Scientific Conference ``Actual Problems of Continuum Mechanics - 2020''}, pp.~105--111. (in Russian)}

\bibitem{B04}
Великанов, П.Г., Куканов, Н.И., Халитова, Д.М. Нелинейное деформирование цилиндрической панели ступенчато-переменной жесткости на упругом основании методом граничных элементов. В~сб. \textit{Материалы Всероссийской научной конференции <<Актуальные проблемы механики сплошных сред -- 2020>>}, с. 111--115. \altbib{Velikanov, P.G., Kukanov, N.I. , Khalitova, D.M., Nonlinear deformation of a cylindrical panel of step-variable stiffness on an elastic base by the method of boundary elements. \textit{Materialy Vserossiyskoy nauchnoy konferentsii ``Aktual'nye problemy mekhaniki sploshnykh sred -- 2020'' = Proc. of the All-Russian Scientific Conference ``Actual Problems of Continuum Mechanics - 2020''}, pp.~111--115. (in Russian)}

\bibitem{B05}
Великанов, П.Г., Куканов, Н.И., Халитова, Д.М., Использование непрямого метода граничных элементов для расчета изотропных пластин на упругом основании Винклера и Пастернака-Власова. \textit{Вестник Самарского университета. Естественнонаучная серия}, 2021, т.~27, №~2, с.~33--47. \altbib{Velikanov, P.G., Kukanov, N.I., Khalitova, D.M., Using the indirect boundary element method for calculating isotropic plates on an elastic base of Winkler and Pasternak-Vlasov. \textit{Vestnik Samarskogo universiteta. Estestvennonauchnaya seriya = Bull. of Samara University. Natural Science Series}, 2021, vol.~27, no.~2, pp.~33--47. (in Russian)}

\bibitem{B06}
Великанов, П.Г., Халитова, Д.М., Решение задач нелинейного деформирования анизотропных пластин и оболочек методом граничных элементов. \textit{Вестник Самарского университета. Естественнонаучная серия}, 2021, т.~27, №~2, с.~48--61. \altbib{Velikanov, P.G., Khalitova, D.M., Solving problems of nonlinear deformation of anisotropic plates and shells by the method of boundary elements. \textit{Vestnik Samarskogo universiteta. Estestvennonauchnaya seriya = Bull. of Samara University. Natural Science Series}, 2021, vol.~27, no.~2, pp.~48--61. (in Russian)}

\bibitem{B07}
Великанов, П.Г., Артюхин, Ю.П., Общая теория ортотропных оболочек. Часть I. \textit{Вестник Самарского университета. Естественнонаучная серия}, 2022, т.~28, №~1--2, с.~46--54. \altbib{Velikanov, P.G., Artyukhin, Yu.P., General theory of orthotropic shells. Part I. \textit{Vestnik Samarskogo universiteta. Estestvennonauchnaya seriya = Bull. of Samara University. Natural Science Series}, 2022, vol.~28, no.~1--2, pp.~46--54. (in Russian)}

\bibitem{B08}
Великанов, П.Г., Артюхин, Ю.П., Общая теория ортотропных оболочек. Часть II. \textit{Вестник Самарского университета. Естественнонаучная серия}, 2022, т.~28, №~3--4, с.~40--52. \altbib{Velikanov, P.G., Artyukhin, Yu.P., General theory of orthotropic shells. Part II. \textit{Vestnik Samarskogo universiteta. Estestvennonauchnaya seriya = Bull. of Samara University. Natural Science Series}, 2022, vol.~28, no.~3--4, pp.~40--52. (in Russian)}

\bibitem{B09}
Великанов, П.Г., Артюхин, Ю.П., Математические аналогии для решения задач прочности, устойчивости и колебаний ортотропных пластин и оболочек. \textit{Экологический вестник научных центров Черноморского экономического сотрудничества}, 2022, т.~19, №~3, с.~47--54. \altbib{Velikanov, P.G., Artyukhin, Yu.P., Mathematical analogies for solving problems of strength, stability and vibrations of orthotropic plates and shells. \textit{Ekologicheskiy vestnik nauchnykh tsentrov Chernomorskogo ekonomicheskogo sotrudnichestva = Ecological Bulletin of the scientific centers of the Black Sea Economic Cooperation}, 2022, vol.~19, no.~3, pp.~47--54. (in Russian)} \edn{JYGZJI} \doi{10.31429/vestnik-19-3-47-54}

\bibitem{B10}
Velikanov, P., Solution of contact problems of anisotropic plates bending on an elastic base using the compensating loads method. \textit{E3S Web of Conferences}, 2023, vol.~402 (International Scientific Siberian Transport Forum -- TransSiberia 2023), art.~11010. \doi{10.1051/e3sconf/202340211010}

\bibitem{B11}
Амбарцумян, С.А., \textit{Теория анизотропных оболочек}. Москва, Физматгиз, 1961. \altbib{Ambartsumyan, S.A., \textit{Teoriya anizotropnykh obolochek = Theory of anisotropic shells}. Moscow, Fizmatgiz, 1961. (in~Russian)}

\bibitem{B12}
Артюхин, Ю.П., Саченков, А.В., К расчету ортотропных пластин и оболочек. В сб. \textit{Исследования по теории пластин и оболочек}, вып. 5. Казань, Изд-во КГУ, 1967, с.~300--310. \altbib{Artyukhin, Yu.P., Savchenkov, A.V., To the calculation of orthotropic plates and shells. In: \textit{Issledovaniya po teorii plastin i obolochek = Research on the theory of plates and shells}, iss.~5, Kazan, Publ. of KSU, 1967, pp.~300--310. (in Russian)}

\bibitem{B13}
Саченков, А.В., О сведении расчета ортотропных пластин и оболочек. В сб. \textit{Исследования по теории пластин и оболочек}, вып. 11. Казань, Изд-во КГУ, 1975, с.~180--185. \altbib{Sachenkov, A.V., On the introduction of calculation of orthotropic plates and shells. \textit{Issledovaniya po teorii plastin i obolochek = Research on the theory of plates and shells}, iss.~11. Kazan, Publ. of KSU, 1975, pp.~180--185. (in Russian)}

\bibitem{B14}
Morley, L., An improvement on Donnell's approximation for thin-walled circular cylinders. \textit{Quart. J. Mech. Appl. Math.}, 1959, vol.~12, no.~1, p.~263. \doi{10.1093/qjmam/12.1.89}

\bibitem{B15}
Cooper, R.M., Cylindrical shell under line load. \textit{J. Appl. Mech.}, 1957, vol.~24, №~4, p.~553--558. \doi{10.1115/1.4011600}

\bibitem{B16}
Артюхин, Ю.П., Гурьянов, Н.Г., Котляр, Л.М., \textit{Система Математика 4.0 и ее приложения в механике}. Казань, Казанское математическое общество, Изд-во КамПИ, 2002. \altbib{Artyukhin, Yu.P., Guryanov, N.G., Kotlyar, L.M., \textit{Sistema Matematika 4.0 i ee prilozheniya v mekhanike = The~Mathematics 4.0 system and its applications in mechanics}. Kazan, Kazan Mathematical Society, Publ, of CamPI, 2002. (in Russian)}

\bibitem{B17}
Великанов, П.Г., \textit{Основы работы в системе Mathematiсa}. Казань, Издательство Казанского гос. техн. ун-та, 2010. \altbib{Velikanov, P.G., \textit{Osnovy raboty v sisteme Mathematisa = Fundamentals of work in the Mathematics system}. Kazan, Publ. of Kazan State Technical University, 2010. (in Russian)}

\bibitem{B18}
Мэттьюз, Ф., Роллингс, Р., \textit{Композитные материалы. Механика и технология}. Москва, Техносфера, 2004. \altbib{Matthews, F., Rawlings, R., \textit{Kompozitnye materialy. Mekhanika i tekhnologiya = Composite materials. Mechanics and technology}. Moscow, Technosphere, 2004. (in Russian)}

\bibitem{B19}
Артюхин, Ю.П., Расчет однослойных и многослойных ортотропных оболочек на локальные нагрузки. В сб. \textit{Исследования по теории пластин и оболочек}, вып. 4. Казань, Издательство КГУ, 1966, с.~91--110. \altbib{Artyukhin, Yu.P., Calculation of single-layer and multilayer orthotropic shells for local loads. In: \textit{Issledovaniya po teorii plastin i obolochek = Research on the theory of plates and shells}, vol.~4. Kazan, Publ. of KSU, 1966. pp.~91--110. (in Russian)}

\bibitem{B20}
Stanescu, K. Vissarion, V., A static-geometric analogy for thin elastic shells with orthotropy of the material and its application to the calculation of flat shells and cylindrical shells of circular cross-section. \textit{Revue de mécanique appliquée (RPR)}, 1958, vol.~3, no.~1.

\bibitem{B21}
Саченков, А.А., \textit{Цикл лекций по теории пластин и оболочек}. Казань, Издательство Казанского университета, 2014. \altbib{Sachenkov, A.A., \textit{Tsikl lektsiy po teorii plastin i obolochek = Lecture series on the theory of plates and shells}. Kazan, Kazan University Press, 2014. (in Russian)}
\end{thebibliography}

\end{document}
