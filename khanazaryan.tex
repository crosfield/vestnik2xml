% !TEX TS-program = pdflatexmk
\documentclass[press]{vestnik}

\draft{2}

\OJS{1062}
\EDN{JQKJSV}

\rubric{Механика}
%\usepackage{float}

\begin{document}

\udc{539.3} %% Индекс УДК

\titlerus[Гибридный полуаналитический метод моделирования плоских колебаний слоистых\ldots]{Гибридный полуаналитический метод моделирования плоских колебаний слоистых волноводов с~присоединенными элементами}

\addtotocrus{Гибридный полуаналитический метод моделирования плоских колебаний слоистых волноводов с~присоединенными элементами}

\titleeng[Hybrid semi-analytical approach to modeling in-plane motion of layered waveguides with bonded\ldots]{Hybrid semi-analytical approach to modeling in-plane motion of layered waveguides with bonded composite joints}

\addtotoceng{Hybrid Semi-Analytical Approach to Modeling In-Plane Motion of Layered Waveguides with Bonded Composite Joints}


\authorrus*{Ханазарян}{Артур Дереникович}
\authoreng*{Khanazaryan}{Artur D.}
\inforus{аспирант кафедры теории функций Кубанского государственного университета}
\infoeng{postgraduate student of the Department of Theory of Functions of the Kuban State University}
\orcid{0000-0003-4609-2900}
\email{artur97.10@mail.ru}
\spin{9109-0799}
\address{Рабочий: 350040, Краснодар, ул. Ставропольская, 149. 
   Домашний: 350915, ст-ца Елизаветинская, Новая, 2, тел. +7(952)9871013
   }

\affilrus{\orgname{Кубанский государственный университет}, ул. Ставропольская, 149, \city{Краснодар}, 350040, \country{Россия}}
\affileng{\orgname{Kuban State University}, Stavropolskaya str., 149, \city{Krasnodar}, 350040, \country{Russia}}

\reviewer{Кучумов}{Алексей Геннадьевич}
\inforev{д-р физ.-мат. наук, доцент Пермского национального исследовательского политехнического университета}
\emailrev{kychymov@inbox.ru}
\affilrev{\orgname{Пермский национальный исследовательский политехнический университет}, \city{Пермь}, \country{Россия}}

\review{Содержание статьи в пределах тематики журнала. Подобные статьи рецензенту не известны. Статья актуальна для приложений в области дефектоскопии и structure health monitoring. Название точно отражает содержание. Аннотация, введение и заключение ясные и адекватные. Использованные методы соответствуют задаче. Результаты исследований ясные и четкие. Объем статьи достаточный.  Стиль изложения хороший. Рисунки и таблицы адекватны тексту.  Аббревиатуры, формулы, единицы измерения соответствуют принятым стандартам. Библиография соответствует содержанию. Общая оценка статьи превосходная. Статья написана хорошим научным языком и может быть принята к печати без замечаний.}

\annotationrus{В настоящей работе предложен гибридный метод на основе метода спектральных элементов (МСЭ) и полуаналитического метода конечных элементов (ПАМКЭ) для изучения плоских колебаний составной структуры в частотной области. Такой подход дает возможность представить решение в протяженном волноводе в виде суперпозиции нормальных мод с помощью ПАМКЭ, а~смежные области дискретизировать с помощью МСЭ. На общей для двух областей границе задаются условия непрерывности перемещений и напряжений. Для сопряжения решений вводится вспомогательная функция перемещений, которая раскладывается по тем же базисным функциям, что применяются в~МСЭ и ПАМКЭ. Результаты моделирования сравниваются с расчетами в конечноэлементном пакете COMSOL Multyphysics, демонстрируется хорошее совпадение.}

\annotationeng{To study in-plane motion of a composite structure in the frequency domain, a hybrid scheme based on the spectral element method (SEM) and the semi-analytical finite element method (SAFEM) is proposed. The hybrid approach represents the solution in a waveguide via the SAFEM as a sum of normal modes and the adjacent regions are discretized using the SEM. On the common boundary for two domains, the continuity boundary conditions are specified for displacement and traction vectors. To couple solutions in the two domains, an unknown auxiliary displacement function is introduced, which is approximated using the same basis functions that are used in the SEM and the SAFEM. The simulation results are compared with the results obtained using the finite element commercial software COMSOL Multyphysics and a good agreement is demonstrated.}

\keywordsrus{гибридный метод, упругие волны, плоские колебания, составные структуры, метод спектральных элементов, полуаналитический метод}
\keywordseng{hybrid method, elastic waves, in-plane motion, joint, spectral element method, semi-analytical method}

\fundingrus{Исследование выполнено при финансовой поддержке Российского научного фонда (проект 22-11-00261).}
\fundingeng{The work was supported by the Russian Science Foundation (project 22-11-00261).}

\conflictofinterestrus{Автор заявляет об отсутствии конфликта интересов.}
\conflictofinteresteng{The author declare no competing interests.}


\date{3-06-2024}
\revised{24-06-2024}
\accepted{25-06-2024}

% НЕ УДАЛЯТЬ! Команда \maketitle генерирует заголовок статьи.
%
\maketitle

\section*{Введение}
Для разработки надежных методов обнаружения дефектов с помощью упругих волн требуются эффективные математические модели и алгоритмы. В случае сложных структур, содержащих области простой формы, можно оптимизировать вычислительные ресурсы, применяя комбинацию различных методов. Подобные гибридные подходы, объединяющие преимущества нескольких методов, часто применяются для решения задач математической физики \cite{Burago,Lisitsa}.

Наиболее распространенным прямым численным методом решения волновых задач является метод конечных элементов (МКЭ) \cite{Lu}. Ключевой недостаток как МКЭ, так и родственного ему метода спектральных элементов (МСЭ) \cite{Komatitsch,Ostachowicz}, предполагающий использование многочленов высоких порядков, — необходимость увеличения числа степеней свободы/элементов для достижения приемлемой точности решения. Это приводит к значительному увеличению вычислительных затрат при анализе протяжённых структур, таких как волноводы. Подобный недостаток, хотя в несколько меньшей степени, имеют метод граничных элементов (МГЭ) \cite{Igumnov} и метод масштабируемых граничных конечных элементов (ММГКЭ) \cite{Song}. В МГЭ численная дискретизация проводится на компакте размерностью на единицу меньшей, чем размерность пространства задачи, что при наличии аналитических представлений для матрицы Грина приводит к меньшим затратам машинного времени.

Разновидности метода граничных интегральных уравнений (МГИУ) \cite{Babeshko,Vatulyan,Manolis}, а также полуаналитический метод конечных элементов (ПАМКЭ) \cite{Bartoli} позволяют эффективно рассчитывать волновые поля в слоистых волноводах, но не могут напрямую описывать составные структуры сложной формы. Поэтому в тех случаях, когда в исследуемых структурах имеются протяженные области (волноводы), для изучения которых применимы полуаналитические подходы, прибегают к гибридным схемам. Так, в волновых задачах распространение получили гибридные методы, в рамках которых в протяженной области, в том числе неограниченной, решение строится с использованием полуаналитического метода, а смежные области дискретизируются одним из прямых численных методов, например, \cite{Vivar-Perez,Zou,Glushkov,Golub_2019,Malik,Novikov,Vareldzhan}.

Так как ПАМКЭ предполагает конечно-элементную дискретизацию только поперечного сечения волновода на основе МСЭ, то ПАМКЭ и МСЭ были взяты в качестве основы при разработке предлагаемого в данной работе гибридного метода для исследования динамического поведения структуры, состоящей из протяженного слоистого волновода конечной длины и~прилегающей к нему области прямоугольной формы. В случае областей произвольных форм достаточно стандартной модификации в МСЭ (см. \cite{Lisitsa,Bubenchikov}), поэтому для простоты изложения она здесь не приводится. Предлагаемый гибридный метод обращается к ПАМКЭ для описания распространения волн в упругом протяженном волноводе \cite{Bartoli} и к МСЭ для моделирования динамических колебаний в областях конечных размеров \cite{Shi,Bubenchikov,Golub_2015}. Настоящая работа является продолжением предыдущего исследования \cite{Khanazaryan}, где рассматривался скалярный случай (антиплоские колебания). Поэтому в работе рассматривается уже векторный случай (плоские колебания) и представляется общая схема построения решения путем «сшивки» решений для каждой из смыкающихся областей с помощью метода Галеркина. Приводятся численные результаты предлагаемого гибридного подхода и его сравнение с МКЭ. Демонстрируется хорошее совпадение.

\vspace{2mm}
\section{Постановка задачи}
%%%%%%%%%%%%%%%%%%%%%%%%%%%%%%%%%%%%%%%%%%%%%%%%%%%%%%%%%%%%%%%%%%%%%%%%%%%

Рассматриваются плоские установившиеся гармонические колебания составной области $\Omega = \Omega_1 \cup \Omega_2$. При этом область $\Omega_2 = \{0 \leqslant x_{1} \leqslant d_{2},  h_{3} \dm \leqslant x_{2}\leqslant h_{3}+h_{2}\} $ представляет собой протяженный волновод заданной толщины $h_{2}$, а $\Omega_{1}$ может быть произвольной формы и имеет общую границу $S_{c}$ с областью $\Omega_{2}$. Соответствующая задача теории упругости формулируется в декартовой системе координат $\bm{x} =\{x_1, x_2\}$. Уравнения движения для гармонических колебаний в плоской постановке с круговой частотой $\omega=2 \pi f$ в \textit{p}-й области $\Omega_{p}$ записываются в~терминах тензора напряжений $\sigma_{ij}$ и вектора перемещений с двумя ненулевыми компонентами $\bm{u}=\{u_{1}, u_{2}\}$ следующим образом:
\begin{equation} \label{motion_eq}
	\sum_{j=1}^{2} \frac{\partial \sigma_{ij}^{(p)}(\bm{x})}{\partial x_{j}} + \rho^{(p)} \omega^{2} u_{i}^{(p)}(\bm{x})=0, \quad p=1,2, \quad \bm{x} \in \Omega_{p}.
\end{equation}
Уравнения состояния 
\begin{equation} \label{state_eq}
	\sigma_{ij}^{(p)}(\bm{x})=C_{ijkl}^{(p)} \frac{\partial u_{k}^{(p)}(\bm{x})}{\partial x_{l}}, \quad p=1,2, \quad \bm{x} \in \Omega_{p}
\end{equation}
связывают компоненты тензора напряжений $\sigma_{ij}$ и вектора перемещений $u_{k}$. Здесь $C_{ijkl}^{(p)}$ и~$\rho^{(p)}$ --- компоненты тензора упругих постоянных и плотность \textit{p}-го материала соответственно. Подстановка уравнений~(\ref{state_eq}) в~(\ref{motion_eq}) позволяет переписать уравнения движения в терминах вектора перемещений для \textit{p}-й области $\Omega_{p}$
\begin{equation} \label{motion_eq_u}
	\sum_{j=1}^{2} \sum_{k=1}^{2} \sum_{l=1}^{2} C_{ijkl}^{(p)} \frac{\partial^{2} u_{k}^{(p)}(\bm{x})}{\partial x_{l} \partial x_{j}} + \rho^{(p)} \omega^{2} u_{i}^{(p)}(\bm{x})=0, \quad \bm{x} \in \Omega_{p}.
\end{equation}

\begin{figure}
	\begin{center}
		\includegraphics[width=.92\textwidth]{Fig_Statement02}
		\caption{Геометрия задачи в случае прямоугольной области $\Omega_{1}$} 
		\captionf{Geometry of the problem in the case of a rectangular region $\Omega_{1}$}\label{Fig_Statement02}
	\end{center}
\end{figure}

Внешняя граница области $\Omega$ разбивается на несколько частей $S \dm =  (\partial \Omega) = S_{0} \cup S_{u} \cup S_{\sigma}$ в~зависимости от типа граничных условий. На границе $S_{0}$ задается условие жесткого защемления
\begin{equation} \label{S_0}
	u_{1}({\bm{x}})=u_{2}({\bm{x}})=0, \quad {\bm{x}} \in S_{0},
\end{equation}
на границе $S_{u}$ перемещения приравниваются к константе $u_0 \neq 0$
\begin{equation} \label{S_u}
	u_{1}({\bm{x}})=u_{2}({\bm{x}}) = u_0, \quad {\bm{x}} \in S_{u},
\end{equation}
а граница $S_{\sigma}$ свободна от напряжений
\begin{equation} \label{S_sigm}
	\bm{\sigma}(\bm{x}) \bm{n} = 0, \quad {\bm{x}} \in S_{\sigma}.
\end{equation}
Здесь и далее $\bm{n}=\{n_{1}^{},n_{2}^{}\}$ --- вектор нормали. На внутренней границе $S_{c} = \Omega_1 \cap \Omega_2$ ставятся условия непрерывности перемещений и напряжений
\begin{equation} \label{Sc}
	\textbf{\textit{u}}^{(1)}(\bm{x}) = \textbf{\textit{u}}^{(2)}(\bm{x}), \quad  \bm{\sigma}^{(1)}(\bm{x}) \cdot \bm{n} = \bm{\sigma}^{(2)}(\bm{x}) \cdot \bm{n}, \quad {\bm{x}} \in S_{c}.
\end{equation}



Геометрия задачи, которая обсуждается в настоящей работе, приведена на рис. \ref{Fig_Statement02}. Здесь в~качестве области $\Omega_{1}$ без потери общности рассматривается область прямоугольной формы $\Omega_{1} = [-d_{1},0] \times [0,h_{1}+h_{2}+h_{3}]$.


%%%%%%%%%%%%%%%%%%%%%%%%%%%%%%%%%%%%%%%%%%%%%%%%%%%%%%%%%%%%%%%%%%%%%%%%%%%

\section{Общая схема решения}
%%%%%%%%%%%%%%%%%%%%%%%%%%%%%%%%%%%%%%%%%%%%%%%%%%%%%%%%%%%%%%%%%%%%%%%%%%%
Для построения решения с применением гибридной схемы необходимо сначала рассмотреть решение двух вспомогательных задач для областей $\Omega_{1}$ и $\Omega_{2}$, в которых неизвестная функция перемещений на общей границе $S_{c}$ предполагается известной, и решение строится соответственно с помощью МСЭ и ПАМКЭ. Таким образом, для «сшивки» двух решений на внутренней границе вводится вспомогательная неизвестная функция перемещений
\begin{equation} \label{Sq}
	u_{k}^{(1)}({\bm{x}}) = u_{k}^{(2)}({\bm{x}}) = q_{k}({\bm{x}}), \quad k=1, 2, \quad {\bm{x}} \in S_{c}.
\end{equation}
%%%%%%%%%%%%%%%%%%%%%%%%%%%%%%%%%%%%%%%%%%%%%%%%%%%%%%%%%%%%%%%%%%%%%%%%%%%

	\subsection{Метод спектральных элементов (МСЭ)}
%%%%%%%%%%%%%%%%%%%%%%%%%%%%%%%%%%%%%%%%%%%%%%%%%%%%%%%%%%%%%%%%%%%%%%%%%%%
Для области $\Omega_{1}$ рассматривается вариационная формулировка уравнения~(\ref{motion_eq})
\begin{equation} \label{var_1}
	\sum_{i=1}^{2} \sum_{j=1}^{2} \int\limits_{\Omega_1} 
	\frac{\partial \sigma_{ij}^{(1)}(\bm{x}) }{\partial x_{j} } v_{k^{'}i}^{(1)}(\bm{x}) \d V + \rho^{(1)} \omega^{2} \sum_{i=1}^{2} \int\limits_{\Omega_1} u_{i}^{(1)}(\bm{x}) v_{k^{'}i}^{(1)}(\bm{x}) \d V = 0.
\end{equation}
вместе с граничными условиями (\ref{S_0})–(\ref{S_sigm}) на тех границах, которые относятся к $\Omega_{1}$, и условием (\ref{Sq}), где функция перемещений предполагается известной. Применяя первую формулу Грина, уравнение (\ref{var_1}) можно переписать в слабой постановке
\begin{multline} \label{weak_form1}
	\sum_{i=1}^{2} \sum_{j=1}^{2} \,\oint\limits_{\partial \Omega_1} \sigma_{ij}^{(1)}(\bm{x}) v_{k^{'}i}^{(1)}(\bm{x}) n_{j} \d l - \sum_{i=1}^{2} \sum_{j=1}^{2} \,\int\limits_{\Omega_1} \sigma_{ij}^{(1)}(\bm{x}) \frac{\partial v_{k^{'}i}^{(1)}(\bm{x}) }{\partial x_{j} } \d V + 
	\\
	+\rho^{(1)} \omega^{2} \sum_{i=1}^{2} \,\int\limits_{\Omega_1} u_{i}^{(1)}(\bm{x}) v_{k^{'}i}^{(1)}(\bm{x}) \d V = 0.
\end{multline}
В МСЭ в качестве базисных функций для аппроксимации $u_{i}^{(1)}({\bm{x}})$ используются интерполяционные полиномы $C^{i_{k}} (\xi_{k}^{l_{k}})$  на узлах Гаусса--Лежандра--Лобатто \cite{Golub_2015}. Для прямоугольной области $\Omega_{1}$, производится разбиение на прямоугольные элементы (на $M_{1}$ по оси $x_{1}$ и на $M_{2}=M_{2}^{1}+M_{2}^{2}+M_{2}^{3}$ по оси $x_{2}$)
$$
\Omega_{1}=\bigcup_{l_{1}=1}^{M_{1}} \left[x_{1}^{l_{1}}, x_{1}^{l_{1}+1}\right] \times \bigcup_{l_{2}=1}^{M_{2}} \left[x_{2}^{l_{2}}, x_{2}^{l_{2}+1}\right].
$$
При этом глобальная система координат $x_{k}$ связана с локальной системой координат $\xi_{k}^{l_{k}}$ на каждом элементе $l_{k}$ соотношениями вида
\begin{equation*}
	\begin{split}
		\xi_{k}^{l_k} = \frac{2 x_i - x_{k}^{l_k + 1} -x_{k}^{l_k}}{x_{k}^{l_k + 1} -x_{k}^{l_k}},\quad x_i &= \frac{x_{k}^{l_k + 1} -x_{k}^{l_k}}{2} \xi_{k}^{l_k} + \frac{x_{k}^{l_k + 1} + x_{k}^{l_k}}{2},	\\
		\frac{\d }{\d  x_k} = S^{l_k} \frac{\d }{\d  \xi_{k}^{l_k}},\quad S^{l_k} &= \frac{2}{x_{k}^{l_k + 1} -x_{k}^{l_k}},\quad k=1, 2.
	\end{split}
\end{equation*}
Тогда решение уравнения (\ref{motion_eq}) можно представить как
\begin{equation} \label{u_1k}
	u_{k}^{(1)}(\bm{x})  =  \sum_{l_{1}=1}^{M_{1}} \sum_{l_{2}=1}^{M_{2}} \sum_{i_{1}=1}^{N+1} \sum_{i_{2}=1}^{N+1} y_{k}^{l_{1}l_{2}i_{1}i_{2}} C^{i_{1}}(\xi_{1}^{l_{1}}) C^{i_{2}}(\xi_{2}^{l_{2}})  = \sum_{I_{1}=1}^{2G_{1}} y^{I_{1}} C^{i_{1}}(\xi_{1}^{l_{1}}) C^{i_{2}}(\xi_{2}^{l_{2}}), \quad k=1, 2,
\end{equation}
где функция $I_1(k, l_1, l_2, i_1, i_2) = (k-1)G_{1}+(i_1-1)(M_{2}N+1) \dm + (M_{2} N +1)N(l_1-1)+N(l_2-1)+i_2$ используется для индексации узлов, а величина $G_1 =(M_{1}N+1)(M_{2}N+1)$ –- это общее количество узлов. В узлах Гаусса--Лежандра--Лобатто $\chi_{j}$ значения интерполяционных полиномов совпадают с дельта-символом Кронекера
$$
C^{i}(\chi_{j})=\delta_{ij},
$$
а значения производной $\d C^{i}(\chi_{j}) / \d \xi$ вычисляются аналитически \cite{Golub_2015} и обозначаются как $D_{j}^{i}$. 

Для дискретизации уравнения (\ref{weak_form1}) применяется метод Бубнова--Галеркина, согласно которому тестовые функции $v_{k^{'}i}^{(1)}(\bm{x})$ берутся совпадающими с базисными функциями
\begin{equation} \label{v_1}
	v_{k^{'}i}^{(1)}(x_1,x_2) = \delta_{k^{'}i} C^{i_{1}^{'}}(\xi_{1}^{l_{1}^{'}}) C^{i_{2}^{'}}(\xi_{2}^{l_{2}^{'}}), 
\end{equation}
которые в соответствии с условием (\ref{S_0}) на границе $S_{0}$ равны нулю
\begin{equation} \label{v_1_0}
	v_{k^{'}i}^{(1)}({\bm{x}}) = 0,\quad \bm{x} \in S_{0}.
\end{equation}
В дальнейшем, для краткости, тестовые функции нумеруются индексом $I_{1}^{'}(k^{'},l_{1}^{'},l_{2}^{'},i_{1}^{'},i_{2}^{'})$ по аналогии с индексом $I_1(k,l_1,l_2,i_1,i_2)$.

Интегралы в (\ref{weak_form1}) при использовании (\ref{u_1k})--(\ref{v_1_0}) могут быть вычислены на узлах Гаусса--Лежандра--Лобатто $\chi_{k}$ с помощью квадратуры Лобатто \cite{Ostachowicz}
$$
\int\limits_{-1}^{1} f(x) \d x \approx \sum_{k=1}^{N+1} w_k f(\chi_{k}).
$$
После подстановки (\ref{u_1k}) и (\ref{v_1}) в (\ref{weak_form1}) и учета условия (\ref{v_1_0}) формируется система линейных алгебраических уравнений (СЛАУ)
$$
\sum_{I_1 = 1}^{2G_{1}} A_{I_{1}^{'} I_{1}^{}}^{1} y^{I_1} = \eta_{}^{{I_{1}^{'}}},
$$
где
\begin{multline}
	A_{I_{1}^{'} I_{1}}^{1} = \delta_{l_{1}^{} l_{1}^{'}} \delta_{l_{2}^{} l_{2}^{'}} 
	\Bigg( 
	-C_{k^{'}1k^{}2}^{(1)} w_{i_{1}}^{} D_{i_{1}}^{i_{1}^{'}} w_{i_{2}^{'}} D_{i_{2}^{'}}^{i_{2}}
	-C_{k^{'}2k^{}1}^{(1)} w_{i_{1}^{'}} D_{i_{1}^{'}}^{i_{1}} w_{i_{2}}^{} D_{i_{2}}^{i_{2}^{'}}-
	\\
	-C_{k^{'}1k^{}1}^{(1)} \frac{S^{l_{1}}}{S^{l_{2}}} w_{i_{2}}^{} \delta_{i_{2}^{} i_{2}^{'}} \sum_{k_{1}=1}^{N+1} w_{k_{1}}^{} D_{k_{1}}^{i_{1}^{}} D_{k_{1}}^{i_{1}^{'}}
	-C_{k^{'}2k^{}2}^{(1)} \frac{S^{l_2}}{S^{l_1}} w_{i_1}^{} \delta_{i_{1}^{} i_{1}^{'}} \sum_{k_{2}=1}^{N+1} w_{k_{2}}^{} D_{k_{2}}^{i_{2}^{}} D_{k_{2}}^{i_{2}^{'}}+
	\\
	+\rho^{(1)} \omega^{2} \frac{1}{S^{l_{1}}S^{l_{2}}} w_{i_{1}}^{} w_{i_{2}}^{} \delta_{k^{} k^{'}} \delta_{i_{1}^{} i_{1}^{'}} \delta_{i_{2}^{} i_{2}^{'}}
	\Bigg)
\end{multline}
и $\eta_{}^{{I_{1}^{'}}}$ --- компоненты вектора столбца, которые равны нулю, за исключением тех, что соответствуют узлам на границе $S_{c}$
$$
\eta^{{I_{1}^{'}}}= 
\left\{ 
\begin{aligned}
	&q_{k}(x_{2}), \ k=1, 2, \; l_{1}^{'}=M_{1}, \; i_{1}^{'}=N+1, \; l_{2}^{'} \in [M_{2}^{1}+1, M_{2}^{1}+M_{2}^{2}],& \\[2mm]
	&0, \text{ иначе}.&
\end{aligned} 
\right.
$$
При этом условие (\ref{S_sigm}) дает нулевой вклад контурного интеграла в (\ref{weak_form1}), а для выполнения условия (\ref{S_0}) необходимо изменить соответствующие строки матрицы:
$$
A_{I_{1}^{'}(k^{'},1,l_{2}^{'},1,i_{2}^{'}) I_{1}^{}}^{1} = 0, \quad A_{I_{1}^{'}(k^{'},1,l_{2}^{'},1,i_{2}^{'}) I_{1}^{'}(k^{'},1,l_{2}^{'},1,i_{2}^{'})}^{1} = 1.
$$
%%%%%%%%%%%%%%%%%%%%%%%%%%%%%%%%%%%%%%%%%%%%%%%%%%%%%%%%%%%%%%%%%%%%%%%%%%%

	\subsection{Полуаналитический метод конечных элементов (ПАМКЭ) для протяженных слоистых волноводов} \label{SAFE}
%%%%%%%%%%%%%%%%%%%%%%%%%%%%%%%%%%%%%%%%%%%%%%%%%%%%%%%%%%%%%%%%%%%%%%%%%%%
Для области $\Omega_{2}$, по аналогии с $\Omega_{1}$, используется вариационная формулировка уравнения (\ref{motion_eq_u}) с граничными условиями (\ref{S_0})–(\ref{S_sigm}) на тех границах, которые относятся к $\Omega_{2}$, и условием (\ref{Sq}). При этом к (\ref{motion_eq_u}), согласно ПАМКЭ, предварительно применяется прямое преобразование Фурье по $x_{1}$ с параметром преобразования $\alpha$, что приводит к уравнению
\begin{multline}\label{var_2}
	\sum_{k=1}^{2} \sum_{i=1}^{2} 
	\Bigg(
	C_{i2k2}^{(2)} \int\limits_{h_{3}}^{h_{2}+h_{3}} \frac{\partial^{2} U_{k}(\alpha,x_{2}) }{\partial x_{2}^{2} } v_{k^{'}i}^{(2)}(x_{2}) \d x_{2} -
	 {\rm{i}} \alpha \Big( C_{i2k1}^{(2)} +C_{i1k2}^{(2)} \Big) \int\limits_{h_{3}}^{h_{2}+h_{3}} \frac{\partial U_{k}(\alpha,x_{2}) }{\partial x_{2} } v_{k^{'}i}^{(2)}(x_{2}) \d x_{2} -
	\\
	-\alpha^{2} C_{i1k1}^{(2)} \int\limits_{h_{3}}^{h_{2}+h_{3}} U_{k}(\alpha,x_{2}) v_{k^{'}i}^{(2)}(x_{2}) \d x_{2}
	\Bigg)
	+ \int\limits_{h_{3}}^{h_{2}+h_{3}} \rho^{(2)} \omega^{2} U_{i}(\alpha,x_{2}) v_{k^{'}i}^{(2)}(x_{2}) \d x_{2} = 0
\end{multline}
относительно преобразования Фурье для перемещений
$$
U_{k}(\alpha,x_2) = \mathcal{F}_{x_1} \left[  u_{k}^{(2)}(x_1,x_2) \right] = \int_{-\infty}^{+\infty} u_{k}^{(2)}(x_1,x_2) e^{{\rm{i}} \alpha x_1}\d x_1, \quad k=1, 2.
$$
С учетом~(\ref{state_eq}) можно переписать граничное условие~(\ref{S_sigm}) относительно Фурье-символа $U_{k}(\alpha,x_{2})$
\begin{multline} \label{sigma_U}
	\sigma_{i2}^{(2)}(\bm{x})=\sum_{k=1}^{2}    
	\Bigg(
	C_{i2k1}^{(2)} \frac{\partial u_{k}^{(2)}(\bm{x}) }{\partial x_{1} } + C_{i2k2}^{(2)} \frac{\partial u_{k}^{(2)}(\bm{x}) }{\partial x_{2} } 
	\Bigg) = 
	\\
	=\sum_{k=1}^{2}    
	\Bigg(
	- {\rm{i}} \alpha C_{i2k1}^{(2)} U_{k}(\alpha,x_{2}) + C_{i2k2}^{(2)} \frac{\partial U_{k}(\alpha,x_{2}) }{\partial x_{2} } 
	\Bigg) = 0, \quad {\bm{x}} \in S_{\sigma}.
\end{multline}
Интегрируя по частям и подставляя условие (\ref{sigma_U}), уравнение (\ref{var_2}) приводится к слабой постановке
\begin{multline} \label{weak_form_U_k}
		\sum_{k=1}^{2} \sum_{i=1}^{2} 
		\Bigg(
		-C_{i2k2}^{(2)} \int\limits_{h_{3}}^{h_{2}+h_{3}} \frac{\partial U_{k}(\alpha,x_{2}) }{\partial x_{2} } \frac{\partial v_{k^{'}i}^{(2)}(x_{2}) }{\partial x_{2} } d x_{2} +
		\\
		+{\rm{i}} \alpha C_{i2k1}^{(2)} 
		\Big(
		U_{k}(\alpha,h_{2}+h_{3})v_{k^{'}i}^{(2)}(h_{2}+h_{3}) - U_{k}(\alpha,h_{3}) v_{k^{'}i}^{(2)}(h_{3}) 
		\Big) -
		\\
		- {\rm{i}} \alpha \Big( C_{i2k1}^{(2)} +C_{i1k2}^{(2)} \Big) \int\limits_{h_{3}}^{h_{2}+h_{3}} \frac{\partial U_{k}(\alpha,x_{2}) }{\partial x_{2} } v_{k^{'}i}^{(2)}(x_{2}) \d x_{2}
		- \alpha^{2} C_{i1k1}^{(2)} \int\limits_{h_{3}}^{h_{2}+h_{3}} U_{k}(\alpha,x_{2}) v_{k^{'}i}^{(2)}(x_{2}) \d x_{2}
		\Bigg) +
		\\
		+\rho^{(2)} \omega^{2} \sum_{i=1}^{2} \int\limits_{h_{3}}^{h_{2}+h_{3}} U_{i}(\alpha,x_{2}) v_{k^{'}i}^{(2)}(x_{2}) \d x_{2} = 0.
\end{multline}

Поскольку предполагается осуществить «сшивку» решений двух подзадач с помощью гибридной схемы, Фурье-символ $U_{k}(\alpha,x_{2})$ и вспомогательную функцию $q_{k}(x_{2})$ целесообразно разложить по тем же самым базисным функциям
\begin{equation} \label{U_k}
	U_{k}(\alpha,x_2)  =  \sum_{l=1}^{M_{2}^{2}} \sum_{s=1}^{N+1} \beta_{k}^{ls}(\alpha) C^{s}(\xi^l)
	= \sum_{I_2=1}^{2G_{2}} \beta^{I_2}(\alpha) C^{s}(\xi^l), \quad k=1, 2,
\end{equation}
\begin{equation} \label{q_k}
	  q_{k}(x_2) = \sum_{r=1}^{M_{2}^{2}} \sum_{j=1}^{N+1} \gamma_{k}^{rj} C^{j}(\xi^r) = \sum_{I_2=1}^{2G_{2}} \gamma^{I_2} C^{j}(\xi^r), \quad k=1, 2,
\end{equation}
где $I_2(k,l,s) = (k-1)G_{2}+N(l-1)+s$ используется для индексации узлов в поперечном сечении волновода, а величина $G_{2}=M_{2}^{2}N+1$ -- это общее количество узлов.

Для дискретизации уравнения (\ref{weak_form_U_k}) снова применяется метод Бубнова–Галеркина. Согласно ему тестовые функции $v_{k^{'}i}^{(2)}({x_2})$ берутся совпадающими с базисными функциями
\begin{equation} \label{v_2k}
	v_{k^{'}i}^{(2)}(x_2) = \delta_{k^{'}i} C^{s^{'}}(\xi^{l^{'}}).
\end{equation}
В дальнейшем индекс $I_{2}^{'}(k^{'},l^{'},s^{'})$ используется для нумерации тестовых функций таким же образом, как и индекс $I_{2}(k,l,s)$. 
С учетом подстановки представлений (\ref{U_k}) и (\ref{v_2k}) в (\ref{weak_form_U_k}), а также условия (\ref{S_sigm}) формируется СЛАУ относительно вектора неизвестных $\bm{\beta}$
\begin{equation} \label{Slau_2}
\sum_{I_2=1}^{2G_2} \Big( -K_{I_{2}^{'} I_{2}^{}}^{0} - {\rm{i}} \alpha K_{I_{2}^{'} I_{2}^{}}^{1} - \alpha^{2} K_{I_{2}^{'} I_{2}^{}}^{2} + \omega^{2} M_{I_{2}^{'} I_{2}^{}} \Big) \beta^{I_2} = 0,
\end{equation}
где
$$
K_{I_{2}^{'} I_{2}^{}}^{0} = C_{k^{'}2k2}^{(2)} S_{l} \delta_{l^{} l^{'}} \sum_{n=1}^{N+1} w_{n} D_{n}^{s} D_{n}^{s^{'}},
$$
\begin{equation*}
		K_{I_{2}^{'} I_{2}^{}}^{1} = \Big( C_{k^{'}2k1}^{(2)} + C_{k^{'}1k2}^{(2)} \Big) \delta_{l^{} l^{'}} w_{s^{'}} D_{s^{'}}^{s} - 
		C_{k^{'}2k1}^{(2)} \Big( \delta_{s N+1} \delta_{l M_{2}^{2}} \delta_{s^{'} N+1} \delta_{l^{'} M_{2}^{2}} - \delta_{s 1} \delta_{l 1} \delta_{s^{'} 1} \delta_{l^{'} 1} \Big),
\end{equation*}
$$
K_{I_{2}^{'} I_{2}^{}}^{2} = C_{k^{'}1k1}^{(2)} \frac{1}{S_{l}}  \delta_{l^{} l^{'}} w_{s} \delta_{s s{'}},
$$
$$
M_{I_{2}^{'} I_{2}^{}} = \rho^{2} \frac{1}{S_{l}}  \delta_{l^{} l^{'}} w_{s} \delta_{s s{'}} \delta_{k k{'}}.
$$

Уравнение (\ref{Slau_2}) не сводится напрямую к задаче на собственные значения относительно вектора неизвестных $\bm{\beta}$, поэтому рассматривается эквивалентная система уже относительно вектора неизвестных $\bm{b} = \{ \bm{\beta},\lambda \bm{\beta} \}^{T}$
$$
\big( \bm{B} - \lambda \bm{E}_{4G_{2} \times 4G_{2}}^{} \big) \bm{b} = 0,
$$
$$
{\bm{B}}=
\begin{Vmatrix}
	0 								& \bm{E}_{2G_{2} \times 2G_{2}}^{}  \\[2mm]
	(\bm{K}^{2})^{-1} (\bm{K}^{0} - \omega^{2} \bm{M}) & (\bm{K}^{2})^{-1} \bm{K}^{1}
\end{Vmatrix},
$$
где $\lambda = {\rm{i}} \alpha$, а $\bm{E}$ --- единичная матрица соответствующей размерности. Тогда взаимосвязь между волновым числом $\alpha=- \rm{i} \lambda$ и частотой $\omega$, необходимая для построения различных дисперсионных характеристик волновода, может быть получена путем решения характеристического (дисперсионного) уравнения относительно частоты $\omega$:
\begin{equation} \label{disp}
\text{det} \big( \bm{B}(\omega) - \lambda \bm{E} \big) = 0.
\end{equation}
Полученные при решении задачи (\ref{disp}) собственные числа $\alpha$ и собственные вектора $\bm{\beta}$ используются для построения решения в протяженном волноводе $\Omega_{2}$
\begin{multline} \label{u_2k}
		u_{k}^{(2)}(x_1,x_2) = \sum_{m=1}^{2}\sum_{n=1}^{2G_2} c_{n}^{m} U_{k}(\alpha_{n}^{m},x_2,\beta_{k}^{nm}) \mathrm{e}^{ {\rm{i}} \alpha_{n}^{m} (x_{1}^{m}-x_{1}^{}) } = \\
		= \sum_{\tilde{I}_{2} = 1}^{4G_{2}} c^{\tilde{I}_{2}} U_{k}(\alpha^{\tilde{I}_{2}},x_2,\beta_{k}^{\tilde{I}_{2}}) \mathrm{e}^{ {\rm{i}} \alpha_{}^{\tilde{I}_{2}} (x_{1}^{m}-x_{1}^{})},\quad k=1, 2, \quad x_{1}^{1} = 0,\quad x_{1}^{2} = d_{2}^{},
\end{multline}
где $\tilde{I}_{2}(m,n)=2G_{2}(m-1)+n$. Предварительно необходимо отсортировать $\alpha$ и соответствующие им $\bm{\beta}$ так, чтобы $\Im (\alpha) \leqslant 0$ при $m=1$ и  $\Im (\alpha) \geqslant 0$ при $m=2$. Следует также отметить, что для реализации гибридной схемы, по сути, уже достаточно представления (\ref{u_2k}), так как неизвестные значения коэффициентов разложения могут быть найдены подстановкой в~граничные условия на торцах волновода с использованием проекционных методов.

\begin{table}
	\caption{Значения волновых чисел для упругого слоя из алюминия толщиной 1 мм при различных значениях спектральных элементов на частоте $f=10$ МГц} \label{Wave_number_compare}
	\captiont{Wave number values for an elastic aluminum layer 1 mm thick for different values of spectral elements at a frequency of $f=10$ MHz}
	\begin{center}
		\begin{tabularx}{\textwidth}{|Y|Y|Y|Y|Y|Y|Y|}
			%			\hline
			%			& \multicolumn{6}{c|}{Конфигурация спектрального элемента}
			%			\\
			\hline
			Спект- & $N=3,$   &  $N=5,$   &  $N=13,$  &  $N=5,$  &  $N=4,$ &  $N=6,$ 
			\\
			ральные   & $M_{2}^{2}=1,$   & $M_{2}^{2}=1,$ & $M_{2}^{2}=1,$ & $M_{2}^{2}=2,$ & $M_{2}^{2}=5,$ & $M_{2}^{2}=6,$
			\\
			элементы& $G_{2}=4$   & $G_{2}=6$ & $G_{2}=14$  & $G_{2}=11$ & $G_{2}=21$ & $G_{2}=37$ 
			\\
			\hline
			\multicolumn{7}{|P{.9\textwidth}|}{Волновое число $\alpha_{1}^{1}$, мм$^{-1}$ моды A0}
			\\
			\hline
			МГИУ & \multicolumn{6}{Y|}{21,6522}
			\\
			\hline
			ПАМКЭ & 20,4597 & 21,1951  & 21,6522 & 21,5717 & 21,6427 & 21,6522
			\\
			\hline
			$\varepsilon$, \% & 5,83 & 2,16  & 1,18 $\cdot 10^{-5}$ & 0,37 & 0,04  & 7,62 $\cdot 10^{-6}$ 
			\\
			\hline
			\multicolumn{7}{|P{.9\textwidth}|}{Волновое число $\alpha_{2}^{1}$, мм$^{-1}$ моды S0}
			\\
			\hline
			МГИУ & \multicolumn{6}{Y|}{21,6429}
			\\
			\hline
			ПАМКЭ & 20,405 & 21,0058  & 21,6429 & 21,5614 & 21,6333 & 21,6429
			\\
			\hline
			$\varepsilon$, \% & 6,07 & 3,03 & 3,25 $\cdot  10^{-6}$ & 0,38 & 0,04 &  7,67 $\cdot 10^{-6}$
			\\
			\hline
		\end{tabularx}
	\end{center}
\end{table} 

Расчитанные с помощью ПАМКЭ волновые числа $\alpha$ и медленности $s=\alpha / \omega$ для упругого слоя из алюминия толщиной $h_{2} = 1$ мм на частоте $f=10$ МГц сравниваются с аналогичными расчетами МГИУ в табл.~\ref{Wave_number_compare} и на рис.~\ref{slowness}. Относительная погрешность $\varepsilon$ в табл.~\ref{Wave_number_compare} рассчитывалась по формуле
$$
\varepsilon = \frac{\left| \alpha_\text{МГИУ} -\alpha_\text{ПАМКЭ}  \right|}{\left| \alpha_\text{ПАМКЭ} \right|} \times 100.
$$
Из табл.~\ref{Wave_number_compare} видно хорошее совпадение волновых чисел моды A0 и S0 на частоте $f=10$ МГц, расчитанных с помощью МГИУ и ПАМКЭ при достаточно высоком порядке интерполяционных полиномов ($N=13$). Однако можно добиться близкой точности при меньшем порядке интерполяционных полиномов, например, при $N=7$ и $M_{2}^{2}=2$ (рис.~\ref{slowness}). Также из рис.~\ref{slowness} видно, что для частот до 1 МГц достаточно минимального количества узлов ($N=3$, $M_{2}^{2}=1$) для получения близких с МГИУ результатов.

\begin{figure}
	\begin{center}
		\includegraphics[width=1 \textwidth]{slowness_SAFEM_and_BIEM_v2}
		\caption{Медленности, расчитанные для упругого слоя из алюминия толщиной 1 мм при различных значениях спектральных элементов}
		\captionf{Slownesses calculated for an elastic layer of aluminum 1 mm thick for different values of spectral elements} \label{slowness}
	\end{center}
\end{figure}

	\subsection{Гибридная схема}
%%%%%%%%%%%%%%%%%%%%%%%%%%%%%%%%%%%%%%%%%%%%%%%%%%%%%%%%%%%%%%%%%%%%%%%%%%%
Для построения решения рассматриваемой задачи (\ref{motion_eq_u})--(\ref{S_sigm}), согласно гибридной схеме, вводится неизвестная функция перемещений на общей границе $S_{c}$. При этом необходимо обеспечить непрерывность перемещений и напряжений (\ref{Sc}) на интерфейсе. В случае МСЭ и~области $\Omega_{1}$ это приводит к изменению строк матрицы $\bm{A}^{1}$, соответствующих узлам на границе $S_{c}$:
$$
A_{I_{1}^{'}(k^{'},M_{1},l_{2}^{'},N+1,i_{2}^{'}) I_{1}^{}}^{1} = 0, \quad A_{I_{1}^{'}(k^{'},M_{1},l_{2}^{'},N+1,i_{2}^{'}) I_{1}^{'}(k^{'},M_{1},l_{2}^{'},N+1,i_{2}^{'})}^{1} = 1,
$$
и к формированию матрицы  $\bm{A}^{2}$ размера $2G_{1} \times 2G_{2}$, элементы которой равны нулю, за исключением тех, что также соответствуют узлам на границе $S_{c}$:
$$
A_{I_{1}^{'}I_{2}}^{2}= 
\left\{ 
\begin{array}{cl}
	1, &\ k=1, 2, \; l_{1}^{'}=M_{1}, \; i_{1}^{'}=N+1, \; l_{2}^{'} \in [M_{2}^{1}+1, M_{2}^{1}+M_{2}^{2}], 
	\\
	0, &\text{ иначе}.
\end{array} 
\right.
$$
Тогда можно выписать следующее уравнение:
$$
\bm{A}^{1} \bm{y} = \bm{A}^{2} \bm{\gamma},
$$
и в области $\Omega_{2}$ будут выполняться граничные условия (\ref{S_u}), (\ref{Sc}) и (\ref{Sq}), если здесь к нахождению решения применить метод Галеркина.

Тестовые функции в виде (\ref{v_2k}), а также подстановка разложений (\ref{q_k}) и (\ref{u_2k}) в условия (\ref{S_u}) и (\ref{Sq}) позволяют получить следующие соотношения с помощью метода Галеркина
\begin{equation*}
\left\{
	\begin{aligned}
			&\sum_{m=1}^{2} \sum_{n=1}^{2G_{2}} c_{n}^{m} \mathrm{e}^{ {\rm{i}} {\alpha_{n}^{m}} (x_{1}^{m}-x_{1}^{1})}  \sum_{l=1}^{M_{2}^{2}} \sum_{s=1}^{N+1} \beta_{k^{'}}^{ls(n)}(\alpha) \frac{1}{S^{l}} w_{s^{'}} \delta_{l_{}^{} l_{}^{'}} \delta_{s_{}^{} s_{}^{'}}  -
			\sum_{r=1}^{M_{2}^{2}} \sum_{j=1}^{N+1} \gamma_{k^{'}}^{rj} \frac{1}{S^{l}} w_{s^{'}}^{} \delta_{r_{}^{} l_{}^{'}} \delta_{j_{}^{} s_{}^{'}} = 0, 
			\\
			&\sum_{m=1}^{2} \sum_{n=1}^{2G_{2}} c_{n}^{m} \mathrm{e}^{ {\rm{i}} {\alpha_{n}^{m}} (x_{1}^{m}-x_{1}^{2})}  \sum_{l=1}^{M_{2}^{2}} \sum_{s=1}^{N+1} \beta_{k^{'}}^{ls(n)}(\alpha) \frac{1}{S^{l}} w_{s^{'}} \delta_{l_{}^{} l_{}^{'}} \delta_{s_{}^{} s_{}^{'}} = u_{0}^{} \frac{1}{S^{l}} w_{s^{'}}^{},
		\end{aligned}
\right.
\end{equation*}
\[
		B_{I_{2}^{'}\tilde{I}_{2}}^{1} = \mathrm{e}^{ {\rm{i}} {\alpha_{n}^{m}} (x_{1}^{m}-x_{1}^{1})}  \sum_{l=1}^{M_{2}^{2}} \sum_{s=1}^{N+1} \beta_{k^{'}}^{ls(n)}(\alpha) \frac{1}{S^{l}} w_{s^{'}} \delta_{l_{}^{} l_{}^{'}} \delta_{s_{}^{} s_{}^{'}}, 
\]
\[
		B_{I_{2}^{'}\tilde{I}_{2}}^{2} = \mathrm{e}^{ {\rm{i}} {\alpha_{n}^{m}} (x_{1}^{m}-x_{1}^{2})} \sum_{l=1}^{M_{2}^{2}} \sum_{s=1}^{N+1} \beta_{k^{'}}^{ls(n)}(\alpha) \frac{1}{S^{l}} w_{s^{'}} \delta_{l_{}^{} l_{}^{'}} \delta_{s_{}^{} s_{}^{'}},
\]
\[
		B_{I_{2}^{'}I_{2}^{}}^{3} = \frac{1}{S^{l}} w_{s^{'}} \delta_{r_{}^{} l_{}^{'}} \delta_{j_{}^{} s_{}^{'}},
\]
\[
		g_{}^{I_{2}^{'}} = u_{0}^{} \frac{1}{S^{l}} w_{s^{'}}^{}.
\]
Аналогично с учетом тестовых функций (\ref{v_2k}) условие непрерывности напряжений на интерфейсе (\ref{Sc}) представляется в виде
\begin{multline*}
		\sum_{l_{1}=1}^{M_{1}} \sum_{l_{2}=1}^{M_{2}} \sum_{i_{1}=1}^{N+1} \sum_{i_{2}=1}^{N+1} y_{k}^{l_{1}l_{2}i_{1}i_{2}}  \sum_{k=1}^{2} \Bigg( C_{i1k1}^{(1)} D_{N+1}^{i_{1}^{}} \delta_{l_{1}^{} M_{1}^{}} \frac{S^{l_{1}}}{S^{l_{2}}} w_{s^{'}}^{} \delta_{i_{2}^{} s_{}^{'}} \delta_{l_{2}^{} l_{}^{'}} + C_{i1k2}^{(1)} \delta_{i_{1}^{} N+1} \delta_{l_{1}^{} M_{1}^{}} w_{s^{'}}^{} \delta_{l_{2}^{} l_{}^{'}} D_{s^{'}}^{i_{2}^{}}	
		\Bigg) - 
		\\
		- \sum_{m=1}^{2} \sum_{n=1}^{2G_{2}} c_{n}^{m} \sum_{k=1}^{2}
		\Bigg( -C_{i1k1}^{(2)} {\rm{i}} {\alpha_{n}^{m}} \mathrm{e}^{ {\rm{i}} {\alpha_{n}^{m}} (x_{1}^{m}-x_{1}^{1})} \sum_{l=1}^{M_{2}^{2}} \sum_{s=1}^{N+1} 		\beta_{k^{}}^{ls(n)}(\alpha) \frac{1}{S^{l}} w_{s^{'}}^{} \delta_{s_{}^{} s_{}^{'}} \delta_{l_{}^{} l_{}^{'}} +
		\\
		+ C_{i1k2}^{(2)} \mathrm{e}^{ {\rm{i}} {\alpha_{n}^{m}} (x_{1}^{m}-x_{1}^{1})}  \sum_{l=1}^{M_{2}^{2}} \sum_{s=1}^{N+1} \beta_{k^{}}^{ls(n)}(\alpha) w_{s^{'}}^{} \delta_{l_{}^{} l_{}^{'}} D_{s_{}^{'}}^{s} 
		\Bigg) = 0
\end{multline*}	
и
\[
		Q_{I_{2}^{'}I_{1}^{}}^{i1} = \delta_{l_{1}^{} M_{1}^{}} \delta_{l_{2}^{} l_{}^{'}} \sum_{k=1}^{2} \Bigg(  C_{i1k1}^{(1)} D_{N+1}^{i_{1}^{}}  \frac{S^{l_{1}}}{S^{l_{2}}} w_{s^{'}}^{} \delta_{i_{2}^{} s_{}^{'}}  +  C_{i1k2}^{(1)} \delta_{i_{1}^{} N+1} w_{s^{'}}^{} D_{s^{'}}^{i_{2}^{}} \Bigg),
\]
\begin{multline*}
		Q_{I_{2}^{'}\tilde{I}_{2}}^{i2} = \sum_{k=1}^{2}
		\Bigg( -C_{i1k1}^{(2)} {\rm{i}} {\alpha_{n}^{m}} \mathrm{e}^{ {\rm{i}} {\alpha_{n}^{m}} (x_{1}^{m}-x_{1}^{1})} \sum_{l=1}^{M_{2}^{2}} \sum_{s=1}^{N+1} 		\beta_{k^{}}^{ls(n)}(\alpha) \frac{1}{S^{l}} w_{s^{'}}^{} \delta_{s_{}^{} s_{}^{'}} \delta_{l_{}^{} l_{}^{'}}  +
		\\
		+ C_{i1k2}^{(2)} \mathrm{e}^{ {\rm{i}} {\alpha_{n}^{m}} (x_{1}^{m}-x_{1}^{1})} \sum_{l=1}^{M_{2}^{2}} \sum_{s=1}^{N+1} \beta_{k^{}}^{ls(n)}(\alpha) w_{s^{'}}^{} \delta_{l_{}^{} l_{}^{'}} D_{s_{}^{'}}^{s} 
		\Bigg).
\end{multline*}
%%%%%%%%%%%%%%%%%%%%%%%%%%%%%%%%%%%%%%%%%%%%%%%%%%%%%%%%%%%%%%%%%%%%%%%%%%%

В результате  проделанных выше преобразований для составной области $\Omega$ в случае гибридного подхода к решению формируется СЛАУ относительно коэффициентов разложений  $u_{k}^{(1)}$, $u_{k}^{(2)}$ и $q_{k}$  соответственно в виде (\ref{u_1k}), (\ref{u_2k}) и (\ref{q_k}), или в краткой записи
\begin{equation*}
	\begin{gathered}
		\bm{A}^{1} \bm{y} = \bm{A}^{2} \bm{\gamma}, \quad
		\bm{B}^{1} \bm{c} = \bm{B}^{3} \bm{\gamma}, \quad
		\bm{B}^{2} \bm{c} = \bm{g}, \quad
		\bm{Q}^{i1} \bm{y} = \bm{Q}^{i2} \bm{c}.
	\end{gathered}
\end{equation*}
Вид уравнений позволяет «свернуть» и упростить эту систему уравнений для численных расчетов путем исключения коэффициентов разложения для искусственно введенной неизвестной функции перемещений $q_{k}$. В результате система 
\begin{equation*}
	\begin{gathered}
		\bm{A}^{1} \bm{u} = \tilde{\bm{A}}^{2} \bm{\gamma}, \quad
		\bm{B}^{2} \bm{c} = \bm{g}, \quad
		\bm{Q}^{i1} \bm{u} = \bm{Q}^{i2} \bm{c}.
	\end{gathered}
\end{equation*}
содержит только коэффициенты разложений для $u_{k}^{(1)}$, $u_{k}^{(2)}$, где $\tilde{\bm{A}}^{2} \dm = \bm{A}^{2} (\bm{B}^{3})^{-1} \bm{B}^{1}$.
%%%%%%%%%%%%%%%%%%%%%%%%%%%%%%%%%%%%%%%%%%%%%%%%%%%%%%%%%%%%%%%%%%%%%%%%%%%

\section{Численный анализ}
%	\subsection{Сравнение с другими методами}
%%%%%%%%%%%%%%%%%%%%%%%%%%%%%%%%%%%%%%%%%%%%%%%%%%%%%%%%%%%%%%%%%%%%%%%%%%%
Компьютерная реализация полуаналитического гибридного метода выполнена в пакете прикладных программ MATLAB. Основные характеристики исследуемой составной структуры приведены в табл.~\ref{mat}. Задаваемые на границе $S_{u}$ перемещения равнялись: $u_{0}=1$ мм. Результаты, полученные с помощью реализованного гибридного метода, сравниваются с МКЭ (COMSOL Multiphysics). В конечноэлементном пакете COMSOL использовались полиномы пятой степени (Quintic Lagrange discretization).

На рис. \ref{u1_x}--\ref{u2_x} изображены перемещения $u_{1}$, $u_{2}$, а на рис.~\ref{s11}--\ref{s22} напряжения $\sigma_{11}$, $\sigma_{12}$, $\sigma_{22}$  во всей составной области $\Omega$, вычисленные с помощью гибридной схемы при $N=7$, $M_{1}=M_{2}^{2}=4$ (рис.~\ref{u1_xb}--\ref{u2_xb} и \ref{s11b}--\ref{s22b}) и с помощью МКЭ (рис.~\ref{u1_xa}--\ref{u2_xa} и \ref{s11a}--\ref{s22a}) для большой частоты $f= 5$~МГц. Можно видеть хорошее совпадение с расчетом МКЭ.

\begin{table}[b!] 
	\caption{Свойства материалов и геометрические параметры волноводной структуры} \label{mat}
	\captiont{Material properties and geometric parameters of the waveguide structure}
	\begin{center}
		\begin{tabularx}{\textwidth}{|Y|P{20mm}|P{20mm}|Y|Y|Y|Y|Y|}
			\hline
			Область  & Материал &  Параметры  &$\rho_{i}$,& $d_{i}$,& $h_{1}$,& $h_{2}$,& $h_{3}$,
			\\
			& &    Ламе, ГПа     &кг/м$^{3}$ &мм & мм & мм & мм
			\\
			\hline
			$\Omega_{1}$ & Сталь & $\lambda=104,4$; $\mu=80$ & 7850 & 1 & 1 & 1 & 1 
			\\
			\hline
			$\Omega_{2}$ & Алюминий & $\lambda=55,5$; $\mu=26,1$ & 2700 & 10 & -- & 1  & -- 
			\\
			\hline
		\end{tabularx}
	\end{center}
\end{table}

\begin{figure}
	\centering
	\begin{subfigure}{.95\textwidth}
		\includegraphics[width=\textwidth]{hammer_ux_g_5MHz_a}
		\caption{}\label{u1_xa}
	\end{subfigure}
		\begin{subfigure}{.95\textwidth}
		\includegraphics[width=\textwidth]{hammer_ux_g_5MHz_b}
		\caption{}\label{u1_xb}
	\end{subfigure}
		\caption{Перемещения $u_{1}(x_{1},x_{2})$, рассчитанные с помощью МКЭ (а) и гибридной схемы (б) при $f=5$~МГц и $h_{3}=1$ мм}
		\captionf{Displacements $u_{1}(x_{1},x_{2})$ calculated using FEM (а) and hybrid scheme (б) at $f=5$ MHz and $h_{3}=1$ mm} \label{u1_x}
\end{figure}

\begin{figure}
	\centering
	\begin{subfigure}{.95\textwidth}
		\includegraphics[width=\textwidth]{hammer_uz_g_5MHz_a}
		\caption{}\label{u2_xa}
	\end{subfigure}
		\begin{subfigure}{.95\textwidth}
		\includegraphics[width=\textwidth]{hammer_uz_g_5MHz_b}
		\caption{}\label{u2_xb}
	\end{subfigure}
		\caption{Перемещения $u_{2}(x_{1},x_{2})$, рассчитанные с помощью МКЭ (а) и гибридной схемы (б) при $f=5$~МГц и $h_{3}=1$ мм}
		\captionf{Displacements $u_{2}(x_{1},x_{2})$ calculated using FEM (а) and hybrid scheme (б) at $f=5$~MHz and $h_{3}=1$ mm} \label{u2_x}

\end{figure}

\begin{figure}
	\centering
	\begin{subfigure}{.95\textwidth}
		\includegraphics[width=\textwidth]{hammer_s11_g_5MHz_a}
		\caption{}\label{s11a}
	\end{subfigure}
		\begin{subfigure}{.95\textwidth}
		\includegraphics[width=\textwidth]{hammer_s11_g_5MHz_b}
		\caption{}\label{s11b}
	\end{subfigure}
		\caption{Напряжения $\sigma_{11}(x_{1},x_{2})$, рассчитанные с помощью МКЭ (а) и гибридной схемы (б) при $f= 5$~МГц и $h_{3}=1$ мм}
		\captionf{Stresses $\sigma_{11}(x_{1},x_{2})$ calculated using FEM (а) and hybrid circuit (б) at $f= 5$~MHz and $h_{3}=1$~mm} \label{s11}
\end{figure}

\begin{figure}
	\centering
	\begin{subfigure}{.95\textwidth}
		\includegraphics[width=\textwidth]{hammer_s12_g_5MHz_a}
		\caption{}\label{s12a}
	\end{subfigure}
		\begin{subfigure}{.95\textwidth}
		\includegraphics[width=\textwidth]{hammer_s12_g_5MHz_b}
		\caption{}\label{s12b}
	\end{subfigure}
		\caption{Напряжения $\sigma_{12}(x_{1},x_{2})$, рассчитанные с помощью МКЭ (а) и гибридной схемы (б) при $f=5$~МГц и $h_{3}=1$ мм}
		\captionf{Stresses $\sigma_{12}(x_{1},x_{2})$ calculated using FEM (а) and hybrid circuit (б) at $f=5$~MHz and $h_{3}=1$ mm} \label{s12}
\end{figure}

\begin{figure}
	\centering
	\begin{subfigure}{.95\textwidth}
		\includegraphics[width=\textwidth]{hammer_s22_g_5MHz_a}
		\caption{}\label{s22a}
	\end{subfigure}
		\begin{subfigure}{.95\textwidth}
		\includegraphics[width=\textwidth]{hammer_s22_g_5MHz_b}
		\caption{}\label{s22b}
	\end{subfigure}
		\caption{Напряжения $\sigma_{22}(x_{1},x_{2})$, рассчитанные с помощью МКЭ (а) и гибридной схемы (б) при $f= 5$~МГц и $h_{3}=1$ мм}
		\captionf{Stresses $\sigma_{22}(x_{1},x_{2})$ calculated using FEM (а) and hybrid circuit (б) at $f=5$~MHz and $h_{3}=1$ mm} \label{s22}
\end{figure}

Для исследования сходимости гибридного метода вводятся и вычисляются относительные погрешности модулей векторов перемещений $|{\bm{u_{}^{}}}|$ и $|{\bm{u}}_{comsol}^{}|$ в норме пространства $L_{1}$, полученных, соответственно, с использованием гибридной схемы и МКЭ на «мелкой» сетке для каждой области $\Omega_{p}$
$$
\varepsilon_{p} = \frac{\parallel |{\bm{u_{}^{}}}|-|{\bm{u}}_{comsol}^{}| \parallel_{L_{1}^{}(\Omega_{p})}^{}}{\parallel |{\bm{u}}_{comsol}^{}| \parallel_{L_{1}^{}(\Omega_{p})}^{}}.
$$
На рис. \ref{eps_p} видно уменьшение относительной погрешности расчета на основе гибридной схемы с увеличением количества спектральных элементов при $N=4$ и при $N=7$.
\begin{figure}
	\centering
	\includegraphics[width=.25\textwidth]{Epsilon_0}\\
	\begin{subfigure}{.49\textwidth}
		\includegraphics[width=\textwidth]{Epsilon_a}
		\caption{}\label{eps_pa}
	\end{subfigure}
	\begin{subfigure}{.49\textwidth}
		\includegraphics[width=\textwidth]{Epsilon_b}
		\caption{}\label{eps_pb}
	\end{subfigure}		 
		\caption{Относительные погрешности гибридной схемы $\varepsilon_{p}$, рассчитанные в области $\Omega_{1}$ (а) и~области~$\Omega_{2}$~(б) при $f= 5$~МГц и $h_{3}=1$ мм} 
		\captionf{Relative errors of the hybrid $\varepsilon_{p}$ circuit calculated in the $\Omega_{1}$ region (а) and the~$\Omega_{2}$~region~(б) at $f= 5$~MHz and $h_{3}=1$ mm}\label{eps_p}
\end{figure}

%%%%%%%%%%%%%%%%%%%%%%%%%%%%%%%%%%%%%%%%%%%%%%%%%%%%%%%%%%%%%%%%%%%%%%%%%%%

\section*{Заключение}
%%%%%%%%%%%%%%%%%%%%%%%%%%%%%%%%%%%%%%%%%%%%%%%%%%%%%%%%%%%%%%%%%%%%%%%%%%%
В работе представлен гибридный полуаналитический метод решения динамических задач теории упругости для составной упругой структуры, составленной из слоистого протяженного волновода и области конечных размеров. Для описания плоских колебаний в частотной области используются метод спектральных элементов и полуаналитический метод конечных элементов (ПАМКЭ).

Продемонстрирована сходимость гибридного метода и показано хорошее совпадение с~результатами, полученными методом конечных элементов в пакете COMSOL Multiphysics\textsuperscript\textregistered. Следует отметить, что для протяженных волноводов точность построения решения с помощью ПАМКЭ контролировалась путем сравнения с методом граничных интегральных уравнений (МГИУ). Можно заключить, что ПАМКЭ обеспечивает близкую к МГИУ точность, если на данной частоте количество узлов превосходит количество распространяющихся мод.

Предложенный гибридный метод может быть полезен для изучения волновых процессов в~акустических метаматериалах (например, для метаматериалов в~виде пластины с периодическим массивом тонких полосовых полостей) \cite{Golub_2023} и проведения параметрического анализа с~целью определения наиболее оптимальной конфигурации.

%%%%%%%%%%%%%%%%%%%%%%%%%%%%%%%%%%%%%%%%%%%%%%%%%%%%%%%%%%%%%%%%%%%%%%%%%%%


%%%%%%%%%%%%%%%%%%%%%%%%%%%%%%%%%%%%%%%%%%%%%%%%%%%%%%%%%%%%%%%%%%%%%%%%%%%%%%%
% СЕКЦИЯ ОПИСАНИЯ ИСПОЛЬЗОВАННОЙ ЛИТЕРАТУРЫ
%
%
\begin{thebibliography}{23}

%\bibitem{B01}
%Фамилия,~И.\,О., \emph{Название книги}. Москва, Издательство, 1999. [Familiya, I.\,O., \emph{Nazvanie knigi = Book Name}, Moscow, Izdatelstvo, 1999. (in Russian)]
%
%\bibitem{B02}
%Фамилия,~И.\,О., Название статьи. \emph{Название журнала}, 1999, \No~1, с.~00--00. [Familiya, I.\,O., Article name. \emph{Nazvanie zhurnala = Journal Name}, 1999, no.~1, pp.~00--00. (in Russian)]

\bibitem{Burago}
Бураго,~Н.\,Г., Никитин,~И.\,С., Якушев,~В.\,Л., Гибридный численный метод решения нестационарных задач механики сплошной среды с применением адаптивных наложенных сеток. \emph{Журнал вычислительной математики и математической физики}, 2016, т.~56, \No~6, с.~1082--1092. \altbib{Burago,~N.\,G., Nikitin,~I.\,S., Yakushev,~V.\,L., Hybrid numerical method for solving nonstationary problems of continuum mechanics using adaptive superimposed grids. \emph{Zhurnal vychislitel’noy matematiki i matematicheskoy fiziki = Journal of Computational Mathematics and Mathematical Physics}, 2016, vol.~56, no.~6, pp.~1082--1092. (in Russian)} \doi{10.7868/s0044466916060107}

\bibitem{Lisitsa}
Lisitsa,~V., Tcheverda,~V., Botter,~C., Combination of the discontinuous Galerkin method with finite differences for simulation of seismic wave propagation. \emph{Journal of Computational Physics}, 2016, vol.~311, pp.~142--157. \doi{10.1016/j.jcp.2016.02.005}

\bibitem{Lu}
Lu,~J.\,-F., Liu,~Y., Feng,~Q.\,-S., Wavenumber domain finite element model for the dynamic analysis of the layered soil with embedded structures. \emph{European Journal of Mechanics - A/Solids}, 2022, vol.~96, pp.~104696. \doi{10.1016/j.euromechsol.2022.104696}

\bibitem{Komatitsch}
Komatitsch,~D., Vilotte,~J.\,-P., Vai,~R., Castillo-Covarrubias,~J.\,M., Sánchez-Sesma,~F.\,J., The~spectral element method for elastic wave equations – application to 2-D and 3-D seismic problems. \emph{International Journal for Numerical Methods in Engineering}, 1999, vol.~45, no.~9, pp.~1139--1164. \doi{10.1002/(sici)1097-0207(19990730)45:9<1139::aid-nme617>3.0.co;2-t}

\bibitem{Ostachowicz}
Ostachowicz,~W., Kudela,~P., Krawzuk,~M., Zak,~A.\,J., \emph{Guided Waves in Strutures for SHM The Time-Domain Spetral Element Method}. John Wiley \& Sons, 2012. \doi{10.1002/9781119965855}

\bibitem{Igumnov}
Баженов,~В.\,Г., Игумнов,~Л.\,А., \emph{Методы граничных интегральных уравнений и граничных элементов в решении задач трехмерной динамической теории упругости с сопряженными полями}. Москва, Физматлит, 2008. \altbib{Bazhenov,~V.\,G., Igumnov,~L.\,A., \emph{Metody granichnykh integral’nykh uravneniy i granichnykh elementov v reshenii zadach trekhmernoy dinamicheskoy teorii uprugosti s sopryazhennymi polyami = The boundary integral equation method and the boundary element method for three-dimensional elastodynamic problems with conjugate fields}. Moscow, Fizmatlit, 2008. (in Russian)}

\bibitem{Song}
Song,~С., Wolf,~J.\,P., The scaled boundary finite-element method–alias consistent infinitesimal finite-element cell method–for elastodynamics. \emph{Computer Methods in Applied Mechanics and Engineering}, 1997, vol.~147, no.~3-4, pp.~329--355. \doi{10.1016/s0045-7825(97)00021-2}

\bibitem{Babeshko}
Бабешко,~В.\,А., Глушков,~Е.\,В., Зинченко,~Ж.\,Ф., \emph{Динамика неоднородных линейно–упругих сред}. Москва, Наука, 1989. \altbib{Babeshko,~V.\,A., Glushkov,~E.\,V., Zinchenko,~Zh.\,F., \emph{Dinamika neodnorodnykh lineyno–uprugikh sred = Dynamics of inhomogeneous linearly elastic media}. Moscow, Nauka, 1989. (in Russian)}

\bibitem{Vatulyan}
Ватульян,~А.\,О., Шамшин,~В.\,М., Новый вариант граничных интегральных уравнений и их применение к динамическим пространственным задачам теории упругости. \emph{Прикладная математика и~механика}, 1998, т.~62, \No~3, с.~462--469. \altbib{Vatulyan,~A.\,O., Shamshin,~V.\,M., A new version of boundary integral equations and their application to dynamic spatial problems of elasticity theory. \emph{Prikladnaja matematika i mehanika = Applied Mathematics and Mechanics}, 1998, vol.~62, no.~3, pp.~462--469. (in Russian)}

\bibitem{Manolis}
Manolis,~G.\,D., Dineva,~P.\,S., Rangelov,~T.\,V., Wuttke,~F., State-of-the-Art for the BIEM. \emph{Solid Mechanics and Its Applications}, 2017, vol.~240, pp.~9--52. \doi{10.1007/978-3-319-45206-7_2}

\bibitem{Bartoli}
Bartoli,~I., Marzani,~A., Lanza di Scalea,~F., Viola,~E., Modeling wave propagation in damped waveguides of arbitrary cross-section. \emph{Journal of Sound and Vibration}, 2006, vol.~295, no.~3-4, pp.~685--707. \doi{10.1016/j.jsv.2006.01.021}

\bibitem{Vivar-Perez}
Vivar-Perez,~J.\,M., Duczek,~S., Gabbert,~U., Analytical and higher order finite element hybrid approach for an efficient simulation of ultrasonic guided waves I: 2D-analysis. \emph{Smart Structures and Systems}, 2014, vol.~13, no.~4, pp.~587--614. \doi{10.12989/sss.2014.13.4.587}

\bibitem{Zou}
Zou,~F., Aliabadi,~M.\,H., A boundary element method for detection of damages and self-diagnosis of transducers using electro-mechanical impedance. \emph{Smart Materials and Structures}, 2015, vol.~24, no.~9, pp.~095015. \doi{10.1088/0964-1726/24/9/095015}

\bibitem{Glushkov}
Глушков,~Е.\,В., Глушкова,~Н.\,В., Евдокимов,~А.\,А., Гибридная численно-аналитическая схема для расчета дифракции упругих волн в локально неоднородных волноводах. \emph{Акустический журнал}, 2018, \No~1, с.~3--12. \altbib{Glushkov,~E.\,V., Glushkova,~N.\,V., Evdokimov,~A.\,A., Hybrid numerical-analyticalscheme for calculating elastic wave diffraction in locally inhomogeneous waveguides. \emph{Akusticheskiy zhurnal = Acoustical Physics}, 2018, no.~1, pp.~33--12. (in Russian)} \doi{10.7868/S0320791918010082}

\bibitem{Golub_2019}
Golub,~M.\,V., Shpak,~A.\,N., Semi-analytical hybrid approach for the simulation of layered waveguide with a partially debonded piezoelectric structure. \emph{Applied Mathematical Modelling}, 2019, vol.~65, pp.~234--255. \doi{10.1016/j.apm.2018.08.019}

\bibitem{Malik}
Malik,~M.\,K., Chronopoulos,~D., Tanner,~G., Transient ultrasonic guided wave simulation in layered composite structures using a hybrid wave and finite element scheme. \emph{Composite Structures}, 2020, vol.~246, pp.~112376. \doi{10.1016/j.compstruct.2020.112376}

\bibitem{Novikov}
Новиков,~О.\,И., Евдокимов,~А.\,А., Реализация гибридного численно-аналитического подхода для решения задач дифракции SH-волн на препятствиях произвольной формы. \emph{Экологический вестник научных центров Черноморского экономического сотрудничества}, 2020, т.~17, \No~2, с.~49--56. \altbib{Novikov,~O.\,I., Evdokimov,~A.\,A., Implementation of a hybrid numerical-analytical approach for solving the problems of SH-wave diffraction by arbitrary-shaped obstacles. \emph{Ekologicheskiy vestnik nauchnykh tsentrov Chernomorskogo ekonomicheskogo sotrudnichestva = Ecological Bulletin of Research Centers of the Black Sea Economic Cooperation}, 2020, vol.~17 no.~2, pp.~49--56. (in Russian)} \doi{10.31429/vestnik-17-2-49-56}

\bibitem{Vareldzhan}
Варелджан,~М.\,В., Двухэтапная вычислительная схема для моделирования возбуждения упругих колебаний в изотропном слое поверхностным пьезопреобразователем. \emph{Экологический вестник научных центров Черноморского экономического сотрудничества}, 2024, т.~21, \No~1, с.~57--69. \altbib{Vareldzhan,~M.\,V., Two-step computational scheme for modeling the excitation of elastic waves by an ultrasonic piezoelectric transducer. \emph{Jekologicheskij vestnik nauchnyh centrov Chernomorskogo jekonomicheskogo sotrudnichestva = Ecological Bulletin of Research Centers of the Black Sea Economic Cooperation}, 2024, vol.~21, no.~1, pp.~57--69. (in Russian)} \doi{10.31429/vestnik-21-1-57-69}

\bibitem{Shi}
Shi,~L., Zhou,~Y., Wang,~J\,-M., Zhuang,~M., Liu,~N., Liu,~Q.\,H., Spectral element method for elastic and acoustic waves in frequency domain. \emph{Journal of Computational Physics}, 2016, vol.~327, pp.~19--38. \doi{10.1016/j.jcp.2016.09.036}

\bibitem{Bubenchikov}
Бубенчиков,~А.\,М., Попонин,~В.\,С., Мельникова,~В.\,Н., Математическая постановка и решение пространственных краевых задач методом спектральных элементов. \emph{Вест. Том. гос. ун-та. Математика и механика}, 2008, \No~3, с.~70--76. \altbib{Bubenchikov,~A.\,M., Poponin,~V.\,S., Mel’nikova,~V.\,N., The mathematical statement and solution of spatial boundary value problems by means of spectral element method. \emph{Matematika i mekhanika = Mathematics and Mechanics}, 2008, no.~3, pp.~70--76. (in Russian)}

\bibitem{Golub_2015}
Голуб,~М.\,В., Шпак,~А.\,Н., Бюте,~И., Фритцен,~К.\,-П., Моделирование гармонических колебаний и~определение резонансных частот полосового пьезоэлектрического актуатора методом конечных элементов высокого порядка точности. \emph{Вычислительная механика сплошных сред}, 2015, т.~8, \No~4, с.~397--407. \altbib{Golub,~M.\,V., Shpak,~A.\,N., Buethe,~I., Fritzen,~C.\,-P., Harmonic motion simulation and resonance frequencies determination of a piezoelectric strip-like actuator using high precision finite element method. \emph{Vychislitel'naya mekhanika sploshnykh sred = Computational continuum mechanics}, 2015, vol.~8, no.~4, pp.~397--407. (in Russian)} \doi{10.7242/1999-6691/2015.8.4.34}

\bibitem{Khanazaryan}
Ханазарян,~А.\,Д., Голуб,~М.\,В., Гибридный метод для моделирования антиплоских колебаний слоистых волноводов с присоединенными элементами. \emph{Вычислительная механика сплошных сред}, 2023, т.~16, \No~1, с.~101--114. \altbib{Khanazaryan,~A.\,D., Golub,~M.\,V., Hybrid method for modelling anti-plane vibrations of layered waveguides with bonded composite joints. \emph{Vychislitel'naya mekhanika sploshnykh sred = Computational continuum mechanics}, 2023, vol.~16, no.~1, pp.~101--114. (in Russian)} \doi{10.7242/1999-6691/2023.16.1.8}

\bibitem{Golub_2023}
Golub,~M.\,V., Moroz,~I.\,A., Wang,~Y., Khanazaryan,~A.\,D., Kanishchev,~K.\,K., Okoneshnikova,~E.\,A., Shpak,~A.\,N., Mareev,~S.\,A., Zhang,~C., Design and Manufacturing of the Multi-Layered Metamaterial Plate with Interfacial Crack-like Voids and Experimental-Theoretical Study of the Guided Wave Propagation. \emph{Acoustics}, 2023, vol.~5, pp.~122--135. \doi{10.3390/acoustics5010008}

\end{thebibliography}

\end{document}
