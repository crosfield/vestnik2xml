% !TEX TS-program = pdflatexmk
\documentclass[press]{vestnik}

\begin{document}

\draft{1}
\OJS{1059}

\rubric{Механика}

\udc{539.3}

\titlerus{Исследование задачи термоупругости для трансверсально-изотропного тела вращения}

\addtotocrus{Исследование задачи термоупругости для трансверсально-изотропного тела вращения}

\titleeng{Study of the thermoelasticity problem for~a~transversally isotropic body of rotation}

\addtotoceng{Study of the Thermoelasticity Problem for~a~Transversally Isotropic Body of Rotation}

\authorrus*{Иванычев}{Дмитрий Алексеевич}
\authoreng*{Ivanychev}{Dmitry A.}
\inforus{канд. физ.-мат. наук, доцент кафедры общей механики Липецкого государственного технического университета}
\email{Lsivdmal@mail.ru}
\orcid{0000-0002-7736-9311}

\affilrus{\orgname{Липецкий государственный технический университет}, ул.~Интернациональная, 5, \city{Липецк}, 398600, \country{Россия}}
\affileng{\orgname{Lipetsk State Technical University}, Iternationalnaya~st., 5, \city{Lipetsk}, 398600, \country{Russia}}

\reviewer{Левина}{Екатерина Юрьевна}
\inforev{к.т.н, доцент, кафедра физики, Московский государствен-ный технический университет им. Н.Э. Баумана}
\emailrev{hensi-l@yandex.ru}
\affilrev{\orgname{Московский государственный технический университет им. Н.Э. Баумана}, \city{Москва}, \country{Россия}}

\review{}

\annotationrus{В работе представлена математическая модель построения термоупругих полей 
для анизотропного цилиндрического тела конечной длины. Цилиндр находится в 
равновесии под действием установившегося поля температур, при отсутствии 
дополнительных источников тепла. Материал цилиндра обладает прямолинейной 
трансверсальной изотропией. Модель строится на основе энергетического метода 
граничных состояний. Базис пространства внутренних состояний в составе 
метода граничных состояний формируется согласно общему представлению, 
выражающее пространственное напряженно-деформированное состояние через 
совокупность плоских вспомогательных состояний. В~качестве таких состояний 
выступают плоские решения термоупругой задачи. После формирования базиса 
внутренних состояний проводится его ортогонализация, искомые 
характеристики термоупругого поля раскладываются в ряд Фурье по элементам 
ортонормированного базиса, где в качестве коэффициентов выступают 
квадратуры.

Приведено решение термоупругой задачи для кругового в плане цилиндра из 
гипотетического трансверсально-изотропного материала. Представлены явные и 
косвенные признаки сходимости решения задачи и результат представлен в 
графическом виде.}

\keywordsrus{термоупругость, метод граничных состояний, трансверсально-изотропные материалы, пространственная задача}

\annotationeng{The paper presents a mathematical model for constructing thermoelastic 
fields for an anisotropic cylindrical body of finite length. The cylinder is 
in equilibrium under the influence of a steady temperature field, in the 
absence of additional heat sources. The material of the cylinder has 
rectilinear transversal isotropy. The model is built on the basis of the 
energy method of boundary states. The basis of the space of internal states 
as part of the boundary states method is formed according to the general 
representation, expressing the spatial stress-strain state through a set of 
plane auxiliary states. Such states are plane solutions of the thermoelastic 
problem. After forming the basis of internal states, it is orthogonalized, 
and the desired characteristics of the thermoelastic field are expanded into 
a Fourier series according to the elements of the orthonormal basis, where 
quadratures act as coefficients.

A solution to the thermoelastic problem for a circular cylinder made of a 
hypothetical transversally isotropic material is presented. Explicit and 
indirect signs of convergence of the solution to the problem are presented 
and the result is presented in graphical form.}

\keywordseng{thermoelasticity, boundary state method, transversally isotropic 
materials, spatial problem}

\date{13-05-2024}
\revised{30-05-2024}
\accepted{5-06-2024}

\maketitle

Современное машиностроение с целью оптимизации формы деталей, возможности 
использования современных материалов, таких как композиты, полимерные 
материалы и др., предъявляет высокое требование к прочностным расчетам тел, 
изготовленных из этих материалов. Зачастую дополнительно эти тела пребывают 
в экстремальных температурных условиях, что вкупе со сложностью структуры 
(анизотропия упругих свойств) материала усложняет расчет их 
напряженно-деформированного состояния. Температурные поля в объектах 
машиностроения определяют характер распределения температурных напряжений и 
деформаций, которые при известных условиях могут вывести деталь из строя. 
Важность решения задачи о~величине и характере распределения температурных 
напряжений в телах из усложненных по свойствам материалов очевидна.

Решению задач термоупругости для анизотропных тел, с той или иной симметрией 
упругих свойств, посвящено множество работ. Например, в работе~\cite{B01} 
исследуется задачи термомеханики для неоднородных анизотропных сред. 
Предлагаемый подход решения задачи предполагает разбиения на подзадачи 
структурного и параметрического синтеза, каждая из которых решается разными 
методами. В работе~\cite{B02} рассмотрена плоская стационарная задача Дирихле, 
когда на границе тела заданы перемещения и температура. Задача сводится к 
системе интегральных уравнений, и, если граница принадлежит классу Ляпунова, 
то система разрешима по Фредгольму. В работе~\cite{B03} рассмотрена осесимметричная 
задача по определению напряжений в~полом цилиндрическом пуансоне при горячем 
деформировании заготовок. Построению матриц Грина трехмерной теории 
термоупругости посвящена работа~\cite{B04}. Представлены интегральный и 
полиномиальный подходы формирования матриц Грина как часть 
гранично-элементного моделирования, а также подход на основе двойных рядов 
Фурье. Ряд работ посвящен исследованию задач термоупругости для слоистых 
анизотропных материалов, например, в~работе~\cite{B05} представлена неклассическая 
модель связанной задачи термоупругого деформирования слоистых анизотропных 
оболочек и пластин. В пространстве изображений по Лапласу строится 
пространственный функционал, который, с учетом допущений, удалось свести к 
двумерному и вывести из него корректные дифференциальные уравнения и 
связанные краевые условия. В работе~\cite{B06} представлен метод отсчетных 
поверхностей для анализа стационарных задач термоупругости для слоистых 
анизотропных пластин, подвергающихся термической нагрузке. В работе~\cite{B07} 
методом граничных элементов рассматривается трехмерная линейная 
математическая теория термоупругости. Выписаны граничные интегральные 
уравнения и гранично-элементная схема их решения. В работе~\cite{B08} исследуется 
влияние скачкообразного изменения температуры на напряженное состояние 
многосвязной анизотропной бесконечной пластинки. Приближенные решения 
получены численными методами. В работе~\cite{B09} рассмотрен термоупругий 
осесимметричный изгиб кольцевых пластин, изготовленных из 
разносопротивляющихся ортотропных материалов, обладающих к тому же 
цилиндрической анизотропией. Приведена система сложных неоднородных 
уравнений и результаты их исследования с помощью метода конечных разностей.

В работе~\cite{B10} представлено решение задачи по определению осесимметричного 
термоупругого поля для трансверсально-изотропного тела вращения средствами 
метода граничных состояний. Решение строилось на основе плоской задачи 
теории термоупругости. Приведено решение для ступенчатого кругового 
цилиндра, находящегося в установившемся поле температур, осесимметричном 
относительно оси вращения. А в работе~\cite{B11} осесимметричная задача 
термоупругости решена в совокупности с задачей эластостатики и с задачей по 
определению упругого поля от действия массовых сил. Общее решение 
представляет собой суммы решений частных задач.

Целью настоящей работы является развитие аналитического метода определения 
напряженно-деформированного состояния, предложенного в работе~\cite{B10}, на класс 
неосесимметричных термоупругих задач для анизотропных тел вращения.

\section{Постановка задачи}

Рассматривается термоупругое равновесие трансверсально-изотропного тела, 
ограниченного одной или несколькими коаксиальными поверхностями вращения 
(рис. \ref{fig1}) и находящийся под действием неосесимметричного поля установившихся 
температур $T$, заданных по циклическому закону. Ось анизотропии 
трансверсально-изотропного тела совпадает с геометрической осью вращения 
$z$. Принята цилиндрическая система координат. Массовые силы отсутствуют. 
Поверхность тела свободна от усилий и ограничений кинематического характера. 
Область тела не имеет вырезов и пустот. Материал однороден.

\begin{figure}[h]
\centering
\includegraphics[width=.22\textwidth]{ivanychev1}
\caption{Трансверсально-изотропное тело вращения}
\captionf{Transversally isotropic body of rotation}
\label{fig1}
\end{figure}

Задача состоит в определении напряженно-деформированного состояния, 
возникающего в~теле, под действием заданного поля температур.

\section{Дифференциальные и интегральные соотношения для анизотропной среды}

Дифференциальные уравнение равновесия при отсутствии объемных сил~\cite{B12}
\[
\frac{\partial \tau_{zr} }{\partial z}+\frac{\partial \sigma_{r} 
}{\partial r}+\frac{1}{r}\frac{\partial \tau_{r\theta } }{\partial \theta 
}+\frac{\sigma_{r} -\sigma_{\theta } }{r}=0;
\]
\begin{equation}
\label{eq1}
\frac{\partial \tau_{z\theta } }{\partial z}+\frac{\partial \tau_{r\theta 
} }{\partial r}+\frac{1}{r}\frac{\partial \sigma_{\theta } }{\partial 
\theta }+2\frac{\tau_{r\theta } }{r}=0;
\end{equation}
\[
\frac{\partial \sigma_{z} }{\partial z}+\frac{\partial \tau_{zr} 
}{\partial r}+\frac{1}{r}\frac{\partial \tau_{z\theta } }{\partial \theta 
}+\frac{\tau_{zr} }{r}=0.
\]
Соотношения Коши~\cite{B12}
\begin{equation}
\label{eq2}
\varepsilon_{r} =\frac{\partial u}{\partial r};
\quad
\varepsilon_{z} =\frac{\partial w}{\partial z};
\quad
\varepsilon_{\theta } =\frac{1}{r}\frac{\partial v}{\partial \theta 
}+\frac{u}{r};
\end{equation}
\[
\gamma_{zr} =\frac{\partial w}{\partial r}+\frac{\partial u}{\partial z};
\quad
\gamma_{r\theta } =\frac{1}{r}\frac{\partial u}{\partial \theta 
}+\frac{\partial v}{\partial r}-\frac{v}{r};
\quad
\gamma_{z\theta } =\frac{\partial w}{\partial z}+\frac{1}{r}\frac{\partial 
w}{\partial \theta }.
\]
Уравнения совместности деформаций~\cite{B13}
\[
\frac{1}{r^{2}}\frac{\partial }{\partial r}\left(r^{2}\frac{\partial \varepsilon 
_{\theta } }{\partial r}\right)+\frac{1}{r^{2}}\frac{\partial^{2}\varepsilon 
_{rr} }{\partial \theta^{2}}-\frac{1}{r}\frac{\partial \varepsilon_{r} 
}{\partial r}-\frac{2}{r^{2}}\frac{\partial }{\partial r}\left(r\frac{\partial 
\varepsilon_{r\theta } }{\partial \theta }\right)=0;
\]
\[
\frac{1}{r^{2}}\frac{\partial }{\partial r}\left(r\frac{\partial \varepsilon 
_{z\theta } }{\partial \theta }\right)+\frac{1}{r}\frac{\partial^{2}\varepsilon 
_{r\theta } }{\partial \theta \partial z}-\frac{1}{r^{2}}\frac{\partial 
^{2}\varepsilon_{zr} }{\partial \theta^{2}}-\frac{1}{r}\frac{\partial 
}{\partial r}\left(r\frac{\partial \varepsilon_{\theta \theta } }{\partial 
z}\right)+\frac{1}{r}\frac{\partial \varepsilon_{rr} }{\partial z}=0;
\]
\begin{equation}
\label{eq3}
-\frac{\partial }{\partial r}\left(\frac{1}{r}\frac{\partial (r\varepsilon 
_{z\theta } )}{\partial r}\right)+\frac{1}{r^{2}}\frac{\partial }{\partial 
r}(r^{2}\frac{\partial \varepsilon_{r\theta } }{\partial z})+\frac{\partial 
}{\partial r}\left(\frac{1}{r}\frac{\partial \varepsilon_{zr} }{\partial \theta 
}\right)-\frac{1}{r}\frac{\partial^{2}\varepsilon_{rr} }{\partial \theta \partial z}=0;
\end{equation}
\[
\frac{1}{r^{2}}\frac{\partial^{2}\varepsilon_{zz} }{\partial \theta 
^{2}}+\frac{1}{r}\frac{\partial \varepsilon_{zz} }{\partial 
r}+\frac{\partial^{2}\varepsilon_{\theta \theta } }{\partial 
z^{2}}-\frac{2}{r}\frac{\partial^{2}\varepsilon_{z\theta } }{\partial 
\theta \,\partial z}-\frac{2}{r}\frac{\partial \varepsilon_{zr} }{\partial 
z}=0;
\]
\[
\frac{1}{r}\frac{\partial^{2}\varepsilon_{zr} }{\partial \theta \partial 
z}+r\frac{\partial }{\partial r}\left(\frac{1}{r}\frac{\partial \varepsilon 
_{z\theta } }{\partial z}\right)-\frac{\partial^{2}\varepsilon_{r\theta } 
}{\partial z^{2}}-\frac{\partial }{\partial r}\left(\frac{1}{r}\frac{\partial 
\varepsilon_{zz} }{\partial \theta }\right)=0;
\]
\[
\frac{\partial^{2}\varepsilon_{rr} }{\partial z^{2}}+\frac{\partial 
^{2}\varepsilon_{zz} }{\partial r^{2}}-2\frac{\partial^{2}\varepsilon 
_{zr} }{\partial r\,\partial z}=0.
\]
Обобщенный закон Гука~\cite{B12}
\[
\varepsilon_{r} =\frac{1}{E_{r} }(\sigma_{r} -\nu_{r} \sigma_{\theta } 
)-\frac{\nu_{z} }{E_{z} }\sigma_{z} +\alpha_{r} T;
\]
\begin{equation}
\label{eq4}
\varepsilon_{z} =\frac{1}{E_{z} }\left[ {\sigma_{z} -\nu_{z} (\sigma_{r} 
+\sigma_{\theta } )} \right]+\alpha_{z} T;
\end{equation}
\[
\varepsilon_{\theta } =\frac{1}{E_{r} }(\sigma_{\theta } -\nu_{r} \sigma 
_{r} )-\frac{\nu_{z} }{E_{z} }\sigma_{z} +\alpha_{r} T;
\]
\[
\gamma_{zr} =\frac{1}{G_{z} }\tau_{zr} ;
\quad
\gamma_{z\theta } =\frac{1}{G_{z} }\tau_{z\theta } ;
\quad
\gamma_{r\theta } =\frac{1}{G_{r} }\tau_{r\theta } =\frac{2(1+\nu_{r} 
)}{E_{r} }\tau_{r\theta } .
\]
Уравнение теплопроводности~\cite{B12}
\begin{equation}
\label{eq5}
\frac{k_{r} }{k_{z} }\frac{\partial^{2}T}{\partial z^{2}}+\frac{\partial 
^{2}T}{\partial r^{2}}+\frac{1}{r}\frac{\partial T}{\partial r}=0.
\end{equation}
Здесь: $u$, $v$, $w$ --- компоненты вектора перемещения вдоль осей $r,\theta ,z$ 
соответственно; $\varepsilon_{r} $, $\varepsilon_{\theta } $, $\varepsilon 
_{z} $, $\gamma_{r\theta } $, $\gamma_{zr} $, $\gamma_{z\theta } $~--- 
компоненты тензора деформаций; $\sigma_{r} $, $\sigma_{\theta } $, $\sigma 
_{z} $, $\tau_{r\theta } $, $\tau_{zr} $, $\tau_{z\theta } $~--- 
компоненты тензора напряжений; $E_{z} $ и $E_{r} $~--- модули упругости 
соответственно в направлении оси $z$ и~в~плоскости изотропии; $\nu_{z} $~--- 
коэффициент Пуассона, характеризующий сжатие вдоль оси $r$ при растяжении вдоль 
оси $z$; $\nu_{r} $~--- коэффициент Пуассона, характеризующий поперечное сжатие 
в плоскости изотропии при растяжении в этой же плоскости; $G_{r} $ и $G_{z} 
$~--- модули сдвига в плоскости изотропии и перпендикулярной к ней; $k_{z} $ 
и $k_{r} $~--- коэффициенты теплопроводности в направлении оси симметрии и 
перпендикулярно к ней; $\alpha_{z} $ и $\alpha_{r} $~--- коэффициентов 
температурного расширения в направлении осей $z$ и $r$ соответственно.

\section{Метод решения}

В работе~\cite{B12} методом интегральных наложений установлена зависимость между 
пространственным напряженно-деформированным состоянием упругого 
трансверсально-изотропного тела вращения и некоторыми вспомогательными 
двумерными состояниями, компоненты которого зависят от двух координат $z$ и $y$ 
(переменных). Ось $\eta $ перпендикулярна плоскости $zy$. В качестве плоских 
вспомогательных состояний используется плоская деформация ${\rm {\bf 
u}}^{pl}=\{u_{y}^{pl} ,u_{\eta }^{pl} ,u_{z}^{pl} \}$, возникающая в 
бесконечных цилиндрах, имеющих в каждой точке плоскость упругой симметрии, 
параллельную плоскости $zy$ (направление $\eta$).

Переход к пространственному состоянию в цилиндрических координатах 
осуществляется по зависимостям
\[
u_{n} =\frac{1}{2\pi }\left(\int\limits_0^\pi {\left(u_{y}^{pl} +u_{\eta }^{pl} \right)\cos 
[(n-1)\beta ]\d\beta } +\int\limits_0^\pi {\left(u_{y}^{pl} -u_{\eta }^{pl} \right)\cos 
[(n+1)\beta ]\d\beta } \right);
\]
\begin{equation}
\label{eq6}
v_{n} =\frac{1}{2\pi }\left(\int\limits_0^\pi {\left(u_{y}^{pl} +u_{\eta }^{pl} \right)\cos 
[(n-1)\beta ]\d\beta } -\int\limits_0^\pi {\left(u_{y}^{pl} -u_{\eta }^{pl} \right)\cos 
[(n+1)\beta ]\d\beta } \right);
\end{equation}
\[
w_{n} =\frac{1}{\pi }\int\limits_0^\pi {u_{z}^{pl} \cos (n\beta )\d\beta } ;
\quad
y=r\cos (\beta );
\]
\[
u=\sum\limits_{n=a}^b {[u_{n} \cos (n\theta )+u_{n} \sin (n\theta )]} ;
\]
\begin{equation}
\label{eq7}
v=\sum\limits_{n=a}^b {[-v_{n} \sin (n\theta )+v_{n} \cos (n\theta )]} ;
\end{equation}
\[
w=\sum\limits_{n=а}^b {[w_{n} \cos (n\theta )+w_{n} \sin (n\theta )]} .
\]
Деформации вычисляются через соотношения Коши (\ref{eq2}), а напряжения через закон 
Гука (\ref{eq4}).

Установившееся температурное поле $T^{pl}(z, y)$ плоского вспомогательного 
состояния с отсутствующими внутри источниками тепла удовлетворяет уравнению 
теплопроводности~\cite{B12}
\[
\left( {k_{z} \frac{\partial^{2}}{\partial z^{2}}+k_{r} \frac{\partial 
^{2}}{\partial y^{2}}} \right)T^{pl}(z, y)=0,
\]
где:
\[
T^{pl} =\frac{g_{0} }{E_{z} }\Re[\phi_{0}^{'} \left( {\varsigma_{0} } 
\right)],
\quad
\varsigma_{0} =z/\gamma_{0} +iy,
\quad
\gamma_{0} =\sqrt {k_{z} /z_{r} } ,
\]
\[
g_{0} =\frac{(E_{z} -\nu_{z}^{2} E_{r} )(\gamma_{1}^{2} -\gamma_{0}^{2} 
)(\gamma_{2}^{2} -\gamma_{0}^{2} )}{\gamma_{0}^{2} (\alpha_{z} E_{z} 
+\alpha_{r} \nu_{z} E_{r} )-\alpha_{r} E_{z} (1+\nu_{r} )}.
\]
Перемещения и напряжения, соответствующие температурному полю~\cite{B12}
\[
u_{z}^{pl} =\Re[p_{0} \phi_{0} \left( {\varsigma_{0} } \right)];
\quad
u_{y}^{pl} =\Re[iq_{0} \phi_{0} \left( {\varsigma_{0} } \right)];
\quad
u_{\eta }^{pl} =0;
\]
\begin{equation}
\label{eq8}
\sigma_{z}^{pl} =-\Re[\gamma_{0}^{2} \phi_{0}^{'} \left( {\varsigma_{0} } 
\right)];
\quad
\sigma_{y}^{pl} =\Re[\phi_{0}^{'} \left( {\varsigma_{0} } \right)];
\end{equation}
\[
\sigma_{zy}^{pl} =-\Re[\gamma_{0} \phi_{0}^{'} \left( {\varsigma_{0} } 
\right)];
\quad
\sigma_{\eta }^{pl} =\Re[\left( {1-\varepsilon_{0} } \right)\phi_{0}^{'} 
\left( {\varsigma_{0} } \right)],
\]
где $p_{0} $, $q_{0} $, $\varepsilon_{0} $~--- постоянные, зависящие от 
упругих констант и коэффициентов температурного расширения $\alpha_{z} $ и 
$\alpha_{r} $, $\phi_{0} \left( {\varsigma_{0} } \right)$~--- некоторая 
аналитическая функция переменной $\varsigma_{0} $.

Переход от плоских температурных состояний к пространственным производится 
по зависимостям (\ref{eq6}) и (\ref{eq7}).

Определение упругого состояния анизотропного тела осуществляется средствами, 
схожими со средствами метода граничных состояний (МГС)~\cite{B14}. В качестве 
базиса в пространстве внутренних состояний $\Xi $ принимаются наборы
\[
\Xi =\left\{ {\xi_{1} , \xi_{2} , \xi_{3} ,\ldots,\xi_{k} ,\ldots} 
\right\};
\quad
\xi_{k} =\{u_{i}^{(k)} ,\varepsilon_{ij}^{(k)} ,\sigma_{ij}^{(k)} 
,T_{i}^{(k)} \}.
\]
Базис пространства $\Xi $ можно сконструировать, придавая функции $\phi_{0} 
$ в (\ref{eq8}) следующие значения:
\[
\phi_{0} =\varsigma_{0}^{n} ,
\quad
n=1, 2, 3,\ldots.
\]
Ортонормирование базиса пространства $\Xi $ осуществляется по разработанному 
рекурсивно-матричному алгоритму ортогонализации~\cite{B15}, где в качестве 
перекрестных скалярных произведений принимаются (например, для 1-го и 2-го 
состояния)
\[
(\xi_{1} ,\xi_{2} )=\int\limits_V {T_{1} } T_{2} \d V,
\]
После восстановления поля температур, удовлетворяющего уравнению 
теплопроводности (\ref{eq5}), определяются вектор перемещения, тензоры деформаций и 
напряжений, соответствующих температурному состоянию. Каждому базисному 
вектору $T_{k} $ соответствуют вектор перемещения и тензоры деформаций и 
напряжений, в совокупности образующие внутреннее состояние от действия поля 
температур $\xi_{k} $.

Решение представляет собой ряд Фурье
\begin{equation}
\label{eq9}
\xi_{0} =\sum\limits_{k=1}^\infty {c_{k} \xi_{k} } ;
\quad
c_{k} =\int\limits_V {T_{k} } T\d V,
\end{equation}
где $T_{k} $~--- температура в базисном элементе $\xi_{k} $, $T$~--- 
заданное поле температур.

В развернутом виде:
\begin{equation}
\label{eq10}
u_{i} =\sum\limits_{k=1}^\infty {c_{k} u_{i}^{(k)} } ;
\quad
\varepsilon_{ij} =\sum\limits_{k=1}^\infty {c_{k} \varepsilon_{ij}^{(k)}} ;
\quad
\sigma_{ij} =\sum\limits_{k=1}^\infty {c_{k} \sigma_{ij}^{(k)} } ;
\quad
Т=\sum\limits_{k=1}^\infty {c_{k} Т^{(k)} } .
\end{equation}

\section{Решение задачи}

Рассмотрим круговой в плане цилиндр из гипотетического 
трансверсально"=изотропного материала, по свойствам схожего с алевролитом~\cite{B16}. Задача решается в обезразмеренном виде; упругие и термомеханические 
характеристики материала, область тела и заданная функция температуры 
подлежат процедуре обезразмеривания~\cite{B17}. После процедуры обезразмеривания 
параметров задачи, упругие характеристики материала: $E_{z} =6,21$; $E_{r} 
=5,68$; $G_{r} =2,29$; $G_{z} =2,55$; $\nu_{z} =0,22$; $\nu_{r} =0,24$; 
область цилиндра $V=\{ \left. (r,\theta ,z) \right| 0\leqslant r\leqslant 1,\ 0\leqslant 
\theta \leqslant 2\pi ,\ -2\leqslant z\leqslant 2\}$; коэффициенты теплопроводности по осям 
координат~\cite{B18}: $k_{z} =1,6$, $k_{r} =6,5$; коэффициенты температурного 
расширения~\cite{B19}: $\alpha_{z} =6,7$, $\alpha_{r} =8,64$. Заданное поле 
температур описывается функцией $T=r z^{4}\cos \theta $.

При построении базиса внутренних состояний рекомендации, описанные в работах~\cite{B20} и~\cite{B21}, пригодны и для построения такового в задаче термоупругости. Так 
как заданная температура зависит только от косинуса, то базис пространства 
внутренних состояний будем формировать из левых частей выражений (\ref{eq7})
\begin{equation}
\label{eq11}
u=\sum\limits_{n=a}^b {[u_{n} \cos (n\theta )]} ;
\quad
v=\sum\limits_{n=a}^b {[-v_{n} \sin (n\theta )]} ;
\quad
w=\sum\limits_{n=а}^b {[w_{n} \cos (n\theta )]} ,
\end{equation}
Причем, так как в заданном поле температур при угловой координате $n=1$, то в 
выражениях (\ref{eq11}) $a=b=1$.

После построения базиса внутренних состояний по соотношениям (\ref{eq6}) и (\ref{eq7}), 
проводится ортонормирование его элементов, которое включает в себя 
исключение линейно зависимых элементов, а также элементов для которых $T=0$. 
Функции температуры в ортонормированных базисных элементах представлены в 
табл.~\ref{tab1} (показано 4 элемента).

\begin{table}
\caption{Функции температуры в ортонормированных базисных элементах}
\captiont{Temperature functions in orthonormal basis elements}
\begin{tabularx}{\textwidth}{|P{28pt}|Y|}\hline
& 
$T$ \\
\hline
$\xi_{1} $& 
$0,564 r\cos \theta $ \\
\hline
$\xi_{2} $& 
$0,488 rz\cos \theta $ \\
\hline
$\xi_{3} $& 
$(-0,611-0,029 r^{3}+0,473 rz^{2})\cos \theta $ \\
\hline
$\xi_{4} $& 
$(-1,061 rz -0,086 r^{3}z+0,473 rz^{3})\cos \theta $ \\
\hline
\end{tabularx}
\label{tab1}
\end{table}

Для решения задачи потребовался базис внутренних состояний из 50 элементов. 
Коэффициенты Фурье (\ref{eq9}) $c_{k} =\{ 5,6718; 0; 7,2467; 0; 2,1447; 0, -0,0778; 0; -0,11246; 0; -0,0871\ldots \}$. Решение формируется рядами (\ref{eq10}).

Хотя и косвенно, но исследовать полученные ряды на сходимость возможно, 
используя <<насыщение>> (рис.~\ref{fig2}) суммы Бесселя (левая часть неравенства 
Бесселя).

\begin{figure}[t]
\centering
\includegraphics[width=.42\textwidth]{ivanychev2}
\caption{Сумма Бесселя}
\captionf{Bessel sum}
\label{fig2}
\end{figure}

\begin{figure}[t]
\centering
\begin{subfigure}{.37\textwidth}
\includegraphics[width=\textwidth]{ivanychev3a}
\caption{}
\label{fig3a}
\end{subfigure}
\begin{subfigure}{.37\textwidth}
\includegraphics[width=\textwidth]{ivanychev3b}
\caption{}
\label{fig3b}
\end{subfigure}
\begin{subfigure}{.37\textwidth}
\includegraphics[width=\textwidth]{ivanychev3c}
\caption{}
\label{fig3c}
\end{subfigure}
\caption{Верификация решения}
\captionf{Verification of solution}
\label{fig3}
\end{figure}

Также оценка точности осуществляется верификацией заданной функции 
температуры с~полученной в результате решения. На рис.~\ref{fig3} приведено 
сравнение функций температур на границе тела. Заданная функция --- штриховая 
линия; восстановленная --- сплошная линия.

Анализ полученного решения показал, что максимальная относительная 
погрешность составила 3,6~{\%} (средний график, координаты точки: 
$r=0,5$, $\theta =0$, $z=2$). Точность решения задачи повышается при увеличении 
числа используемых элементов базиса внутренних состояний.

Компоненты термоупругого поля показаны на рис.~\ref{fig4} в виде изолиний. В силу симметрии компонент напряженно-деформированного состояния (НДС) относительно плоскости $z = 0$, показано меридианное сечение с~$\theta =0$ и $0\leqslant z\leqslant 2$. Значения на графике указаны в масштабе с~масштабным коэффициентом $\kappa $, т.е истинное значение показанной характеристики НДС равно значению на графике, умноженному на $\kappa $. Все характеристики показаны 
для меридианного сечения с угловой координатой $\theta =0$.

\begin{figure}[t!]
\centering
\begin{subfigure}{.39\textwidth}
\includegraphics[width=\textwidth]{ivanychev4a}
\caption{}
\label{fig4a}
\end{subfigure}
\qquad
\begin{subfigure}{.39\textwidth}
\includegraphics[width=\textwidth]{ivanychev4b}
\caption{}
\label{fig4b}
\end{subfigure}
\begin{subfigure}{.39\textwidth}
\includegraphics[width=\textwidth]{ivanychev4c}
\caption{}
\label{fig4c}
\end{subfigure}
\qquad
\begin{subfigure}{.39\textwidth}
\includegraphics[width=\textwidth]{ivanychev4d}
\caption{}
\label{fig4d}
\end{subfigure}
\caption{Изолинии: а) перемещение $u$, $\kappa =1$, б)~перемещение $w$, $\kappa 
=1$, в)~напряжение $\sigma_{rr}$, $\kappa =10^{2}$, г)~напряжение $\sigma 
_{zz}$, $\kappa =10^{2}$}
\captionf{Isolines: а) displacement $u$, $\kappa =1$, б)~displacement $w$, $\kappa =1$, в)~stress $\sigma_{rr}$, $\kappa =10^{ 2}$, г)~stress $\sigma _{zz}$, $\kappa =10^{2}$}
\label{fig4}
\end{figure}

\addtocounter{figure}{-1}
\begin{figure}
\centering
\begin{subfigure}{.39\textwidth}
\addtocounter{subfigure}{4}
\includegraphics[width=\textwidth]{ivanychev4e}
\caption{}
\label{fig4e}
\end{subfigure}
\qquad
\begin{subfigure}{.39\textwidth}
\includegraphics[width=\textwidth]{ivanychev4f}
\caption{}
\label{fig4f}
\end{subfigure}
\caption{Изолинии: д)~напряжение $\tau_{rz}$, $\kappa =10^{2}$, е)~температура $T$, $\kappa =1$}
\captionf{Isolines: д)~stress $\tau_{rz}$, $\kappa =10^{2}$, е)~temperature $T$, $\kappa =1$}
\label{fig4}
\end{figure}

\begin{figure}
\centering
\begin{subfigure}{.39\textwidth}
\includegraphics[width=\textwidth]{ivanychev5a}
\caption{}
\label{fig5a}
\end{subfigure}
\qquad
\begin{subfigure}{.39\textwidth}
\includegraphics[width=\textwidth]{ivanychev5b}
\caption{}
\label{fig5b}
\end{subfigure}
\begin{subfigure}{.39\textwidth}
\includegraphics[width=\textwidth]{ivanychev5c}
\caption{}
\label{fig5c}
\end{subfigure}
\qquad
\begin{subfigure}{.39\textwidth}\centering
\includegraphics[width=.5\textwidth]{ivanychev5d}
\caption{}
\label{fig5d}
\end{subfigure}
\caption{Изолинии: а)~перемещение $v$, $\kappa =1$, б)~напряжение $\sigma 
_{r\theta }$, $\kappa =1$, в)~напряжение $\sigma_{z\theta }$, $\kappa 
=10^{2}$, г)~контур деформированного состояния тела}
\captionf{Isolines: а)~displacement $v$, $\kappa =1$, б)~stress $\sigma_{r\theta }$, $\kappa =1$, в)~stress $\sigma_{z\theta } $, $\kappa =10^{2}$, г)~contour of the deformed state of the body}
\label{fig5}
\end{figure}

Компоненты упругого состояния $v$, $\tau_{r\theta } $, $\tau_{z\theta } $ 
зависят от $\sin \theta $, поэтому их изолинии представим в сечении с 
$\theta = \pi/2$ (рис.~\ref{fig5}). На рис.~\ref{fig5d} представлен контур 
деформированного состояния тела.



Полученные компоненты термоупругого поля строго удовлетворяют соотношениям 
(\ref{eq1})--(\ref{eq5}).

\section*{Выводы}

Таким образом, в работе сформулирован подход решения задачи по определению 
напряженно"=деформированного состояния анизотропного тела, вызванного 
действием поля температур, заданных по циклическому закону. Компоненты 
упругого поля зависят от всех трех координат и носят неосесимметричный 
характер.

Однако предложенная методика не является общей для любого класса 
рассматриваемых областей (односвязных и многосвязных) и вида заданной 
функции, описывающей температурное поле внутри тела; если эту 
функцию возможно разложить на том или ином промежутке в~тригонометрический 
ряд по косинусам или синусам, то, не исключая принцип независимости действия 
сил, данная методика вполне применима.

Полученное решение имеет аналитический вид, что позволяет легко проводить 
анализ полученных характеристик напряженно"=деформированного состояния 
деталей, применяемых в машиностроении.

\begin{thebibliography}{21}
\bibitem{B01}
Ломазов, В.А., Ломазова, В.И., Построение математической модели при решении 
задач термомеханики. \emph{Вестник Нижегородского университета им.~Н.И.~Лобачевского}, 2011, №~4(4), с.~1582--1584. \altbib{Lomazov, V.A., Lomazova V.I., Construction of a mathematical model for solving problems of thermomechanics. \emph{Vestnik Nizhegorodskogo universiteta im.~N.I.~Lobachevskogo = Vestnik of Lobachevsky University of Nizhni Novgorod}, 2011, no.~4(4), pp.~1582--1584. (in Russian)}

\bibitem{B02}
Богдан, Ю.А., Задача Дирихле в двумерной стационарной анизотропной теормоупругости. \emph{Вестник самарского государственного технического университета. Серия Физико-математические науки}, 2010, № 5(21), с.~64--71. \altbib{Bogdan, Yu.A., Dirichlet problem in two-dimensional stationary anisotropic theormoelasticity. \emph{Vestnik samarskogo gosudarstvennogo tekhnicheskogo universiteta. Seriya Fiziko-matematicheskiye nauki = Journal of Samara State Technical University, Ser. Physical and Mathematical Sciences)}, 2010, no.~5(21), pp.~64--71. (in Russian)}

\bibitem{B03}
Фатеев, В.И., Термоупругие напряжения в полом осесимметричном водоохлаждаемом пуансоне горячего деформирования. \emph{Известия Тульского государственного университета. Технические науки}, 2009, №~1-1, с.~98--104. \altbib{Fateyev, V.I., Thermoelastic stresses in a hollow axisymmetric water-cooled punch of hot deformation]. \emph{Izvestiya Tul'skogo gosudarstvennogo universiteta. Tekhnicheskiye nauki = Izvestiya Tula State University}, 2009, no. 1-1, pp. 98--104. (in Russian)}

\bibitem{B04}
Пазин, В.П., Сравнительный анализ подходов к построению матрицы Грина трехмерной теории термоупругости. \emph{Вестник Нижегородского университета им.~Н.И.~Лобачевского}, 2014, №~4(1), с.~250--253. \altbib{Pazin, V.P., Comparative analysis of approaches to constructing the Green's matrix of the three-dimensional theory of thermoelasticity. \emph{Vestnik Nizhegorodskogo universiteta im.~N.I.~Lobachevskogo = Vestnik of Lobachevsky University of Nizhni Novgorod}, 2014, no. 4(1), pp.~250--253. (in Russian)}

\bibitem{B05}
Андреев, А.Н., Математическая модель термоупругого деформирования слоистых композитных оболочек и пластин. \emph{Известия Алтайского государственного университета}, 2014, вып.~1, №~1(81), с.~19--21. \altbib{Andreyev, A.N., Mathematical model of thermoelastic deformation of layered composite shells and plates. \emph{Izvestiya Altayskogo gosudarstvennogo universiteta = Izvestiya of Altai State University}, 2014, iss. 1, no.~1(81), pp. 19--21. (in Russian)}

\bibitem{B06}
Kulikov, G.M., Mamontov, A.A., Three-dimensional thermoelastic analysis of laminated anisotropic plates. \emph{Вестник ТГТУ}, 2013, т.~19, №~4, с.~853--863. \altbib{Kulikov, G.M., Mamontov, A.A., Three-dimensional thermoelastic analysis of laminated anisotropic plates. \textit{Vestnik TGTU = Transactions of the TSTU}, 2013, vol.~19, no.~4, pp.~853--863. (in Russian)}

\bibitem{B07}
Ратаушко, Я.Ю., Анализ термоупругой динамики трехмерных тел методом граничных элементов. \emph{Вестник Нижегородского университета им.~Н.И.~Лобачевского}, 2011, №~4(4), с.~1736--1737. \altbib{Rataushko, Ya.Yu., Analysis of thermoelastic dynamics of three-dimensional bodies using the boundary element method. \emph{Vestnik Nizhegorodskogo universiteta im.~N.I.~Lobachevskogo = Vestnik of Lobachevsky University of Nizhni Novgorod}, 2011, no.~4(4), pp.~1736--1737. (in Russian)}

\bibitem{B08}
Глушанков, Е.С., Приближенное решение задачи термоупругости для многосвязной анизотропной пластинки при скачках температуры на контурах. \emph{Журнал теоретической и прикладной механики}, 2022, №~3(80), с.~3--13. \altbib{Glushankov, Ye.S., Approximate solution of the thermoelasticity problem for a multiply connected anisotropic plate with temperature jumps on the contours. \textit{Zhurnal teoreticheskoy i prikladnoy mekhaniki = Journal of Theoretical and Applied Mechanics}, 2022, no.~3(80), pp.~3--13. (in Russian)} \doi{10.24412/0136-4545-2022-3-5-13}

\bibitem{B09}
Самсоненко, Г.И., Трещёв, А.А., Термоупругий изгиб кольцевых пластин средней толщины из ортотропных разносопротивляющихся материалов. \emph{Известия ТулГУ. Технические науки}, 2012, вып.~1, с.~238--244. \altbib{Samsonenko, G.I., Treshchov A.A., Thermoelastic bending of medium-thick annular plates made of orthotropic materials of different resistance. \emph{Izvestiya Tula State University}, 2012, iss. 1, pp. 238--244. (in Russian)}

\bibitem{B10}
Иванычев, Д.А., Решение задач термоупругости для анизотропных тел вращения. \emph{Труды МАИ}, 2019, №~106, с.~1--19. \altbib{Ivanychev, D.A., Solution of thermoelasticity problems for anisotropic bodies of rotation. \emph{Trudy MAI}, 2019, no.~106, pp.~1--19. (in Russian)}

\bibitem{B11}
Ivanychev, D.A., Levina, E.Yu., Solution of thermoelasticity problems for solids of revolution with transversal isotropic feature and a body force. \emph{Journal of Physics: Conference Series}, 2019, vol.~1348, art.~012058. \doi{10.1088/1742-6596/1348/1/012058}

\bibitem{B12}
Александров, А.Я., Соловьев, Ю.И., \emph{Пространственные задачи теории упругости (применение методов теории функций комплексного переменного)}. Москва, Наука, 1978. \altbib{Aleksandrov, A.Ya., Solov'ev, Yu.I., \emph{Prostranstvenny`e zadachi teorii uprugosti (primenenie metodov teorii funkcij kompleksnogo peremennogo) = Spatial problems of the theory of elasticity (application of methods of the theory of functions of a complex variable)}. Moscow, Nauka, 1978. (in Russian)}

\bibitem{B13}
Лурье, А.И., \emph{Пространственные задачи теории упругости}. Москва, Госиздат 
технико-теоретической литературы, 1955. \altbib{Lur'ye, A.I., \emph{Prostranstvennyye zadachi teorii uprugosti = Spatial problems of the theory of elasticity}. Moscow, Gosizdat tekhniko-teoreticheskoy literatury, 1955. (in Russian)}

\bibitem{B14}
Пеньков, В.Б., Пеньков, В.В., Метод граничных состояний для решения задач линейной механики. \emph{Дальневосточный математический журнал}, 2001, т.~2, №~2, с.~115--137. \altbib{Penkov, V.B., Penkov V.V., The boundary state method for solving linear mechanics problems. \emph{Dalnevostochnyj matematicheskij zhurnal = Far Eastern Mathematical Journal}, 2001, vol.~2, no.~2, pp.~115--137. (in Russian)}

\bibitem{B15}
Саталкина, Л.В., Наращивание базиса пространства состояний при жестких ограничениях к энергоемкости вычислений. В \emph{Сб. тезисов докладов научной конференции студентов и аспирантов Липецкого государственного технического университета}. Липецк, ЛГТУ, 2007, с.~130--131. \altbib{Satalkina, L.V., Building up the basis of the state space with hard constraints on the energy intensity of computations. In: \emph{Sbornik tezisov dokladov nauchnoj konferencii studentov i aspirantov Lipeczkogo gosudarstvennogo texnicheskogo universiteta = Collection of theses of reports of the scientific conference of students and graduate students of Lipetsk State Technical University}, Lipeczk, LGTU, 2007, pp.~130--131. (in Russian)}

\bibitem{B16}
Лехницкий, С.Г., \emph{Теория упругости анизотропного тела}. Москва, Наука, 1977. \altbib{Lekhnitskiy, S.G., \emph{Teoriya uprugosti anizotropnogo tela = Theory of elasticity of anisotropic body}. Moscow, Nauka, 1977. (in Russian)}

\bibitem{B17}
Левина, Л.В., Новикова, О.С., Пеньков, В.Б., Полнопараметрическое решение задачи теории упругости односвязного ограниченного тела. \emph{Вестник ЛГТУ}, 2016, №~2(28), с.~16--24. \altbib{Levina, L.V., Novikova, O.S., Penkov, V.B., Full-parameter solution of the problem of the theory of elasticity of a simply connected bounded body. \emph{Vestnik LGTU = Bulletin of Lipetsk State Technical University}, 2016, no.~2(28), pp.~16--24. (in Russian)}

\bibitem{B18}
Юдин, В.А., Королёв, А.В., Афанаскин, И.В., Вольпин, С.Г., \emph{Теплоёмкость и теплопроводность пород и флюидов баженовской свиты -- исходные данные для численного моделирования тепловых способов разработки}. Москва, ФГУ ФНЦ НИИСИ РАН, 2015. \altbib{Yudin, V.A., Korolyov, A.V., Afanaskin, I.V., Vol`pin, S.G., \emph{Teployomkost` i teploprovodnost` porod i flyuidov bazhenovskoj svity` -- isxodny`e danny`e dlya chislennogo modelirovaniya teplovy`x sposobov razrabotki = Heat capacity and thermal conductivity of rocks and fluids of the Bazhenov formation are the initial data for the numerical simulation of thermal methods of development}. Moscow, FGU FNCz NIISI RAN, 2015. (in Russian)}

\bibitem{B19}
Дортман, Н.Б. (под ред.), \emph{Физические свойства горных пород и полезных ископаемых (петрофизика). Справочник геофизика}. Москва, Недра, 1984. \altbib{Dortman, N.B. (ed.), \emph{Fizicheskiye svoystva gornykh porod i poleznykh iskopayemykh (petrofizika). Spravochnik geofizika = Physical properties of rocks and minerals (petrophysics). Geophysicist's Handbook}. Moscow, Nedra, 1984. (in Russian)}

\bibitem{B20}
Иванычев, Д.А., Решение неосесимметричной задачи эластостатики для 
трансверсально-изотропного тела вращения. \emph{Вестник МГТУ им.~Н.Э.~Баумана. 
Сер. Естественные науки}, 2022, №~2(101), с.~4--21. \altbib{Ivanychev, D.A., Solution of a non-axisymmetric elastostatics problem for a transversely isotropic body of rotation. \emph{Vestnik MGTU im.~N.E.~Baumana. Ser. Yestestvennyye nauki = Herald of the Bauman Moscow State Technical University. Series Natural Sciences}, 2022, no.~2(101), pp.~4--21. (in Russian)} \doi{10.18698/1812-3368-2022-2-4-21}

\bibitem{B21}
Иванычев, Д.А., Левина, Л.В., Определение неосесимметричных упругих полей в анизотропных телах вращения, вызванных действием объемных сил. \emph{Вестник МГТУ им.~Н.Э.~Баумана. Сер. Естественные науки}, 2022, №~4(103), с.~22--38. \altbib{Ivanychev, D.A., Levina, L.V. Determination of non-axisymmetric elastic fields in anisotropic bodies of rotation caused by the action of body forces. \emph{Vestnik MGTU im.~N.E.~Baumana. Ser. Yestestvennyye nauki = Herald of the Bauman Moscow State Technical University. Series Natural Sciences}, 2022, no.~3(103), pp.~22--38. (in Russian)} \doi{10.18698/1812-3368-2022-4-22-38}
\end{thebibliography}
\end{document}