% !TEX TS-program = pdflatexmk
\documentclass[press]{vestnik}

\begin{document}

\draft{1}
\OJS{1063}
\EDN{BNQFFZ}

\rubric{Физика}

\udc{538.958:535.37}

\titlerus[Оптические свойства многокомпонентных боратных стекол\ldots]{Оптические свойства многокомпонентных боратных стекол, легированных трехвалентным ионами тербия}

\addtotocrus{Оптические свойства многокомпонентных боратных стекол, легированных трехвалентным ионами тербия}

\titleeng{Optical properties of multicomponent borate glasses doped with trivalent terbium ions}

\addtotoceng{Optical Properties of Multicomponent Borate Glasses Doped with Trivalent Terbium Ions}

\authorrus*{Хаммуд}{Алаа}
\authoreng*{Hammoud}{Alaa}
\inforus{канд. физ.-мат. наук, доцент кафедры радиофизики и нанотехнологий}
\email{allahammsss@gmail.com}
\orcid{0000-0001-9076-4618}
\spin{4054-3239}
\address{Хаммуд Алаа, $+$7 (918) 6254837, г. Краснодар, 350058, ул. Ставропольская, 203, кв. 59}

\authorrus{Исаев}{Владислав Андреевич}
\authoreng*{Isaev}{Vladislav A.}
\inforus{д-р физ.-мат. наук, профессор, профессор кафедры теоретической физики и компьютерных технологий}
\email{vlisaev.v@yandex.ru}
\orcid{0000-0003-1955-8904}
\spin{2697-3560}

\affilrus{\orgname{Кубанский государственный университет}, ул. Ставропольская, 149, \city{Краснодар}, 350040, \country{Россия}}
\affileng{\orgname{Kuban State University}, 149 Stavropolskaya Str., \city{Krasnodar}, 350040, \country{Russia}}

\reviewer{Вербенко}{Илья Александрович}
\inforev{д-р физ.-мат. наук, директор НИИ физики Южного федерального университета}
\emailrev{iaverbenko@sfedu.ru}
\affilrev{\orgname{Южный федеральный университет}, \city{Ростов-на-Дону}, \country{Россия}}

\review{Представленная статья посвящена разработке новых материалов для эффективного преобразования высокоэнергетического излучения в видимый свет, что перспективно для выполнения задач по конструированию дозиметрических датчиков, в частности, используемых при радиологическом контроле промышленных объектов. Сама по себе тематика разработки подобных материалов не нова, но сохраняет свою актуальность. Объекты исследования, многокомпонентные боратные стекла, уникальных составов, полученные авторами, исследованы в качестве преобразователей энергии излучений впервые, что, безусловно, обеспечивает новизну работы. Статья актуальна для создания новых материалов для дозиметрического контроля. Название работы лаконично и точно отражает содержание статьи. Аннотация, введение и заключение ясные и адекватные. Использованные методы соответствуют задаче. Результаты исследований ясные и четкие.  Объем статьи достаточный. Стиль изложения хороший. Рисунки и таблицы адекватны тексту. Аббревиатуры, формулы, единицы измерения соответствуют принятым стандартам. Библиография соответствует содержанию. Общая оценка статьи превосходная. Статья представляет собой качественную обработку актуального экспериментального материала, в перспективе она могла бы быть дополнена и стать основой публикаций более общего характера. Статья рекомендуется к публикации.}

\annotationrus{На основе боратных стекол, легированных ионами Tb$^{3+}$, получены люминофоры в~зеленом спектральном диапазоне. Установлены спектральные и люминесцентные свойства многокомпонентных боратных стекол с различной концентрацией Tb$^{3+}$. Наиболее интенсивная люминесценция соответствует переходу $^{5}$D$_{4}\to {}^{7}$F$_{5}$ в ионе Tb$^{3+}$, длина волны которого составляет 542 нм.}

\keywordsrus{боратные стекла, запрещенная зона, люминесценция, время жизни}

\annotationeng{The paper presents the results of studies of spectral and luminescent properties of new series glass multi compositions based on borate and bismuth glass heavily doped with terbium, and investigate the regularity of the characteristics of the obtained samples at different concentrations of bismuth and activator and the mutual effect on the borate glass matrix. It is shown that the intensity of spontaneous radiation increases with an increase in the concentration of terbium. Three main absorption bands were found due to fluctuations of complexes in borate glass. Characteristic absorption peaks of terbium ions at~367, 377 and 485 nm are observed, corresponding to the transitions $^{7}$F$_{6} \to {}^{5}$D$_{4}$, $^{7}$F$_{6} \to ^{5}$G$_{6}$, $^{7}$F$_{6} \to {}^{5}$L$_{10}$. The physical parameters of a multicomponent glass doped with Tb$^{3+}$ ions are calculated. The resulting phosphors have high photoluminescent properties and can be used in various optical applications.}

\keywordseng{borate glasses, forbidden zone, luminescence, lifetime}

\date{5-06-2024}
\revised{20-06-2024}
\accepted{21-06-2024}

\maketitle[1.03]

\section*{Введение}

Бинарные и сложные оксиды металлов в кристаллическом и аморфном состоянии, 
легированные трехвалентными редкоземельными ионами (RE$^{3+})$, в настоящее 
время находят широкое применение во многих областях науки и техники. 
Аморфные среды, легированные RE$^{3+}$ используются как эффективные 
спектральные трансформаторы вакуумного ультрафиолетового излучения (ВУФ) в 
видимое свечение с квантовым выходом больше 1, а также в~индивидуальной 
дозиметрии рентгеновской и гамма радиации, кроме того, активированное 
RE$^{3+}$ стекло является перспективной активной средой для твердотельных 
лазеров, а также для эффективных квантово-оптических устройств~\cite{B01,B02,B03,B04,B05}. В 
частности, при возбуждении на длине волны 369 нм наблюдается эффективная 
люминесценция на переходе $^{5}$D$_{4}\to^{7}$F$_{5}$ в~ионе Tb$^{3+}$ 
в~зеленой области спектра ($\lambda = 545$~нм). Поиск новых составов 
стекол и стеклокерамик, легированных Tb$^{3+}$, с его различными физическими 
и фотолюминесцентными свойствами, является актуальной задачей для 
исследования, для соответствия требованиям быстро расширяющегося 
технологического рынка.

В настоящей работе представлены результаты исследований спектральных и 
люминесцентных свойств новой серии образцов на основе боратного и 
висмутового стекла, легированного ионами тербия.

\section{Синтез образцов и интенсивность спонтанного излучения}

Для изготовления образцов использовалось следующее сырье: H$_{3}$BO$_{3}$; 
NaCO$_{3}$; PbO; Bi$_{2}$O$_{3}$; Tb$_{4}$O$_{7}$ чистотой около 99,99~{\%}. 
Многокомпонентные боратные стекла молярного состава 70B$_{2}$O$_{3}$~-- 10Na$_{2}$O~--~15PbO~--~(5-x)Bi$_{2}$O$_{3}$~-- xTb$_{2}$O$_{3}$, где $x=0$; 1; 2; 3; 5~мол.~{\%} были получены с~использованием метода закалки расплава и обозначены следующим образом: Tb0--Bi5; Tb1--Bi4; Tb2--Bi3; Tb3--Bi2; Tb5--Bi0 соответственно, в зависимости от значения активатора $x$.

\begin{figure}[b!]
\centering
\begin{subfigure}{.67\textwidth}
\includegraphics[width=\textwidth]{hammud1_a}
\caption{}\label{fig1a}
\end{subfigure}
\begin{subfigure}{.67\textwidth}
\includegraphics[width=\textwidth]{hammud1_0}\\
\includegraphics[width=\textwidth]{hammud1_b}
\caption{}\label{fig1b}
\end{subfigure}
\caption{Образцы многокомпонентных боратных стекол: а) образцы с различной 
концентрацией Tb$^{3+}$; б) свечение образцов под воздействием 
ультрафиолетового излучения (365 нм)}
\captionf{Samples of multicomponent borate glasses: (а)~Sample series of Tb$^{3+}$ -- doped bismuth borate glass with different concentrations; (б)~Glasses under 365 nm excitation}
\label{fig1}
\end{figure}

Этапы приготовления стекла по этой методике аналогичны тем, которые 
приведены в~работах~\cite{B06,B07}. Следует отметить, что однородность расплава 
достигается в температурном диапазоне от 700 до 950~\textdegree C в 
зависимости от концентрации тербия. При разливке расплава в платиновый 
тигель, его предварительно нагревали до температуры 350~\textdegree C (ниже 
температуры стеклования боратного висмутового стекла (T$_{g})$~\cite{B06,B08}), чтобы 
предотвратить любой возможный процесс кристаллизации и уменьшения 
вероятности возникновения деформаций из-за термомеханических напряжений, 
образующихся внутри матрицы полученных образцов. Процесс охлаждения стекол 
от 350~\textdegree C до комнатной температуры происходит со скоростью 1~градус в минуту. На~рис.~\ref{fig1a} показана серия изготовленных образцов после 
полировки и~шлифования, толщина которых составляла приблизительно 2,8 мм 
\textpm 0,1~мм, за исключением BBiT--5, который не обрабатывался.

На рис.~\ref{fig1b} показано, что синтезированные стекла под воздействием 
ультрафиолетового излучения (365 нм) светятся ярко-зеленым цветом, за 
исключением образца BBiT--0, который не был легирован. Визуально видно, что 
интенсивность спонтанного излучения возрастает с~увеличением концентрации 
тербия.

\begin{figure}[b!]
\centering
\parbox{.48\textwidth}{\centering
\includegraphics[width=.48\textwidth]{hammud2}
\caption{Инфракрасные Фурье-спектры боратного стекла}
\captionf{Infrared Fourier spectra of borate glass\\[10mm]}
\label{fig2}}
\quad
\parbox{.48\textwidth}{\centering
\includegraphics[width=.48\textwidth]{hammud3}
\caption{Спектры пропускания многокомпонентного стекла бората висмута с~различной концентрацией Tb$_{2}$O$_{3}$}
\captionf{Transmission spectra of multicomponent bismuth borate glass with different concentrations of Tb$_{2}$O$_{3}$}
\label{fig3}}
\end{figure}

\section{Спектральный анализ. Ширина запрещенной зоны}

Спектры поглощения для полученных образцов измерялись на Фурье-спектрометре \linebreak 
VERTEX~70 (рис.~\ref{fig2}).~В~результате обнаружены три основные полосы поглощения, 
обусловленные колебаниями комплексов в боратном стекле: BO$_{4}$ 
тетраэдрические комплексы на 800~cm$^{-1}$, BO$_{3}$ тригональные на 1300~cm$^{-1}$ и небольшой пик поглощения при 680~cm$^{-1}$, обусловленный 
деформационным колебаниями различных боратных кластеров (пентаборатных 
групп, состоящих из комплексов BO$_{4}$ и BO$_{3})$~\cite{B09,B10,B11,B12,B13}. Эти группы 
содержат большое количество немостиковых атомов кислорода (NBO) и 
предполагают взаимодействие тетраэдра BO$_{4}$ с~немостиковыми 
кислородсодержащими комплексами BO$_{3}$~\cite{B09}.

Спектры пропускания регистрировались на спектрофотометре Hitachi u-3000 
(рис.~\ref{fig3}). Стекло, не легированное Tb$_{2}$O$_{3}$ Tb0--Bi5, обладает высокой 
прозрачностью около 80~{\%} в видимой области без учета отражения на 
поверхности образца. В образцах с $x= 1$; 2; 3 наблюдаются характерные пики 
поглощения ионов тербия на 367, 377 и 485~нм, соответствующие переходам 
$^{7}$F$_{6} \to {}^{5}$D$_{4}$, $^{7}$F$_{6} \to {}^{5}$G$_{6}$, 
$^{7}$F$_{6} \to {}^{5}$L$_{10}$.

Различные в спектрах пропускания образцов с увеличением концентрации тербия 
обусловлено:

-- во-первых, снижение прозрачности образца Tb3--Bi2 в области от 400 до 1100 нм 
непосредственно связано с образованием кристаллитов различных размеров и 
типов внутри стекла, что вместе с сопутствующими дефектами приводит к 
рассеянию света;

-- во-вторых, незначительное увеличение прозрачности образцов в 
ультрафиолетовом диапазоне обусловлено уменьшением концентрация висмута с 
увеличением содержания тербия в матрице. Висмут играет существенную роль в 
увеличении поглощения стекла в ближнем ультрафиолетовом диапазоне~\cite{B07,B14}. 
Предел пропускания смещается с 340 нм (для чистого стекла) до 328 нм (для 
стекла с содержанием тербия 3 мол.~{\%}).

В качестве доказательства первого утверждения для образца Tb3--Bi2 на 
сканирующем электронном микроскопе модели JSM-7500F были сняты СЭМ 
изображения (рис.~\ref{fig4}). Данные СЭМ показывают, что образец содержит кристаллиты 
размерами до 30~мкм.

\begin{figure}
\centerline{\includegraphics[width=.55\textwidth]{hammud4}}
\caption{СЭМ изображение образца Tb3--Bi2 с включением кристаллитов}
\captionf{SEM image of Tb3--Bi2 sample with inclusion of crystallites}
\label{fig4}
\end{figure}

Ширина запрещенной зоны была рассчитана по спектру поглощения в соответствии 
с~уравнением~\cite{B15,B16}
\[
\left( \alpha h\nu \right)^{p}=A\left( h\nu -E_{g} \right),
\]
где $p=2$ и $0,5$ для разрешенных прямых и непрямых переходов, 
соответственно; $E_{g}$ --- ширина запрещенной зоны; A --- константа 
пропорциональности, для стекла равная единице; $\nu $~--- частота; $\alpha $~--- 
коэффициент поглощения; $h$ --- постоянная Планка.

\begin{figure}
\centering
\begin{subfigure}{.44\textwidth}
\includegraphics[width=\textwidth]{hammud5_a}
\caption{}\label{fig5a}
\end{subfigure}
\qquad
\begin{subfigure}{.44\textwidth}
\includegraphics[width=\textwidth]{hammud5_b}
\caption{}\label{fig5b}
\end{subfigure}
\caption{Графическое определение прямой ширины запрещенной (а) и непрямой (б) 
зоны многокомпонентного боро-висмутного стекла}
\captionf{Graphical determination of the direct width of the forbidden (а) and 
indirect (б) zones of multicomponent boron-bismuth glass}
\label{fig5}
\end{figure}

Результаты, приведенные на рис.~\ref{fig5}, указывают на тот факт, что с увеличением 
содержания тербия в матрице ширина прямой запрещенной зоны увеличивалась 
примерно на 0,1~эВ при значении в диапазоне от 3,60 до 3,72~эВ, а непрямой 
от 3,49 до 3,66~эВ. Небольшое изменение ширины запрещенной зоны в 
зависимости от различных концентраций эрбия непосредственно связано с 
небольшим изменением оптической плотности образцов. В статье~\cite{B20} авторы 
сообщали о корреляции между шириной запрещенной зоны и оптической плотностью 
образцов.

Что касается образца Tb5--Bi0, то попытка получить энергию запрещенной зоны с 
использованием спектра отражения и функции Кубелки--Мунка не увенчалась 
успехом. Результаты были неудовлетворительными из-за содержания различных 
фаз внутри образца, который содержит аморфную основу, кристаллы различного 
химического состава и, как следствие, обладающих разными физическими 
свойствами.

\section{Расчет физических параметров многокомпонентного стекла, легированного ионами Tb$^\mathbf{3+}$}

Используя величину ширины запрещенной зоны, показатель преломления образцов 
был рассчитан по уравнению~\cite{B17}
\[
\frac{\left( n^{2}-1 \right)}{\left( n^{2}+2 \right)}=1-\sqrt 
\frac{E_{g}}{20} 
\]
где $n$~--- оптический показатель преломления.

Плотность полученных образцов рассчитывалась методом Архимеда
\[
\rho =\frac{W_{a}}{{(W}_{a}-W_{b})} \rho_{b},
\]
где $W_{a}$~--- вес образца в воздухе, $W_{b}$ --- вес образца в плавучем 
состоянии и $\rho_{b}$~--- плотность плавучего вещества. Для определения 
плотности использовали дистиллированную воду.

Используя плотность и показатель преломления образцов, можно рассчитать 
следующие физические параметры по соотношениям, представленным в работах~\cite{B18,B19}:


-- молярный объем
\[
V_{m} =\frac{\sum\limits_{i=1}^N {X_{i}Z_{i}} }{\rho },
\]
где $V_{m}$ --- молярный объем; $X_{i}$ и $М_{i}$ --- мольная доля и 
молекулярная масса компонент стекла;


--- эффективную концентрацию ионов тербия
\[
N_{{Tb}^{3+}}=\frac{2\rho  N_{a} x_{{Tb}^{3+}}}{M_{m}},
\]
где $N_{{Tb}^{3+}}$ --- эффективная концентрация ионов Tb$^{3+}$; $N_{a}$ 
--- число Авогадро; $\rho $ --- плотность стекла; $M_{m}$ --- средняя 
молекулярная масса стекла; $M_{m}$; $x_{{Tb}^{3+}}$ --- мольная доля оксида 
редкоземельного элемента;

-- радиус полярона
\[
r_{p}(A^{^{\circ}})=\left(\frac{1}{2}\right)\left(\frac{\pi }{6N}\right)^{1/3};
\]


-- межъядерное расстояние
\[
r_{i}\left( A^{^{\circ}} \right)=\left(\frac{1}{N}\right)^{1/3};
\]


-- диэлектрическую постоянную
\[
\varepsilon =n^{2};
\]


-- потери на отражение
\[
R_{L}={\left(\frac{n-1}{n+1}\right)}^{2};
\]


-- молярную рефракцию
\[
R_{M}=\left( \frac{n^{2}-1}{n^{2}+2} \right)\cdot V_{m}.
\]

Результаты расчетов представлены в табл. \ref{tab1}.

\begin{table}[h!]
\caption{Физические параметры многокомпонентного стекла, легированного ионами Tb$^{3+}$}
\captiont{Physical parameters of multicomponent glass doped with Tb$^{3+}$ ions}
\begin{tabularx}{\textwidth}{|Y|Y|Y|Y|Y|Y|P{8mm}|P{8mm}|P{8mm}|}
\hline
Код образца& 
$\rho $ \par (г/см$^{3})$& 
$V_{m}$ \par (см$^{3}$/мол)& 
$N_{{Tb}^{3+}}$ \par (см$^{-3})$& 
$r_{p}$ \par (\AA)& 
$r_{i}$ \par (\AA)& 
$\varepsilon $& 
$R_{L}$ \par $\% $& 
$R_{M}$ \\
\hline
Tb5-Bi0& 
3,40& 
31,383& 
$1,9\cdot 10^{21}$& 
3,25& 
8,07& 
--& 
--& 
-- \\
\hline
Tb3-Bi2& 
3,61& 
30,112& 
$1,19\cdot 10^{21}$& 
3,8& 
9,43& 
4,97& 
14,5& 
17,16 \\
\hline
Tb2-Bi3& 
3,64& 
30,139& 
$0,8\cdot 10^{21}$& 
4,34& 
10,77& 
5,06& 
14,97& 
17,34 \\
\hline
Tb1-Bi4& 
3,67& 
30,165& 
$0,4\cdot 10^{21}$& 
5,46& 
13,57& 
5,10& 
14,93& 
17,43 \\
\hline
Tb0-Bi5& 
3,70& 
30,191& 
--& 
--& 
--& 
5,18& 
15,08& 
17,53 \\
\hline
\end{tabularx}
\label{tab1}
\end{table}

\section{Люминесцентный анализ}


\begin{figure}[b!]
\centerline{\includegraphics[width=.5\textwidth]{hammud6}}
\caption{Спектры возбуждения фотолюминесценции стекол бората висмута, 
легированных ионами Tb$^{3+}$ (контроль излучения при 542 нм)}
\captionf{Photoluminescence excitation spectra of Tb$^{3+}$ ions doped bismuth 
borate glasses (monitoring emission at~542~nm)}
\label{fig6}
\end{figure}

Спектры возбуждения излучения люминофора на длине волны 542~нм демонстрируют 
широкую полосу возбуждения в УФ-области с центром при 274~нм и характерные 
пики в~ближней УФ- и синей областях.~Регистрацию спектров возбуждения 
образцов осуществляли на спектрофлюориметре FLUORAT-02-Panorama 
spectrofluorometer (рис.~\ref{fig6}). Характерные пики возбуждения иона Tb$^{3+}$ 
наблюдаются на 352 нм и 372 нм для квантовых переходов $^{7}$F$_{6}  \to $ 
$^{5}$L$_{10}$ и $^{7}$F$_{6} \to {}^{5}$G$_{6}$ соответственно. Широкая 
полоса возбуждения в стеклянной матрице также наблюдается в области 200--320 
нм. Она обусловлена 4f$^{n}\to $4f$^{n-1}$5d (4f--5d) переходами ионов 
Tb$^{3+}$, эти квантовые переходы описывают спин-орбитальное взаимодействие 
электронов на уровнях 4f--5d в лантаноидах, где (SF) означает запрещенный 
спин с полосами с высоким спином (4f-5d), а (SA) --- разрешенный спин для 
переходов с низким спином (4f--5d)~\cite{B21}.

На интенсивность и поведение этого уникального квантового перехода влияет 
природа взаимодействия между ионами активатора и окружающей их средой. 
Широкие полосы возбуждения (4f--5d) указывают на возможность возбуждения с 
высокой эффективностью ионов тербия через стеклянную матрицу. Максимум этого 
возбуждения приходится на 274 нм. Для переходов (4f--5d) интенсивность 
возбуждения возрастала с увеличением концентрации ионов тербия. Что касается 
образца Tb5--Bi0, то увеличение интенсивности спектра возбуждения носит 
нелинейный характер, что связано с образованием в нем кластеров в 
кристаллических состояниях.

\begin{figure}[b!]
\centerline{\includegraphics[width=.5\textwidth]{hammud7}}
\caption{Спектры фотолюминесценции стекла бората висмута, легированного 
ионами Tb$^{3+}$, возбуждение при 274 нм}
\captionf{Photoluminescence spectra of bismuth borate glass doped Tb$^{3+}$ ions}
\label{fig7}
\end{figure}

Следует также отметить, что интенсивность прямого возбуждения иона 
лантаноида одинакова для образцов с содержанием тербия 3 моль~{\%} и 5 
моль~{\%}, что может быть обусловлено насыщением концентрации тербия и, как 
следствие, активным влиянием процессов реабсорбции в стеклах.

Спектры фотолюминесценции (рис.~\ref{fig7}) образцов, легированных тербием, при 
возбуждении 274 нм содержат характерные узкие пики люминесценции иона 
металла, соответствующие излучательным переходам следующим образом: 
$^{5}$D$_{4} \to {}^{7}$F$_{6}$ (488,5 нм), $^{5}$D$_{4} \to $ 
$^{7}$F$_{5}$ (542,0 нм), $^{5}$D$_{4} \to {}^{7}$F$_{4}$ (583,9 нм), 
$^{5}$D$_{4} \to {}^{7}$F$_{3}$ (622,0 нм). Резкое расщепление пиков в 
спектрах всех образцов одинаково, что указывает на идентичное 
координационное окружение ионов тербия. По мере увеличения содержания тербия 
в стекле интегральная интенсивность люминесценции увеличивается нелинейно.

Интенсивность фотолюминесценции зависит от ряда факторов, остановимся на 
двух из них, непосредственно связанных с образцами, изученными в этом 
исследовании. Это концентрация активированных ионов и структура образцов. 
Концентрационное тушение ионов Tb$^{3+}$ в спектре излучения не наблюдалось, 
поэтому с увеличением концентрации ионов тербия интенсивность радиационных 
переходов увеличивались, а безызлучательные переходы (тепловые фононы) четко 
не наблюдались. Это приводит к увеличению радиационной эффективности 
образцов из-за высокой симметрии координации ионов Tb$^{3+}$ и совпадающих 
атомных связей внутри образцов.

Что касается второго фактора, который обусловлен природой образцов и 
проявляется непосредственно в образце Tb5--Bi0, то наличие нанокристаллов с 
высокой степенью симметрии, образовавшихся внутри стеклянной матрицы, 
привело к увеличению интенсивности спектра излучения.

Кинетика затухания люминесценции является моноэкспоненциальной для всех 
образцов, что указывает на эквивалентность всех центров люминесценции. Время 
жизни люминесценции, определяемое как время уменьшения интенсивности 
люминесценции в e раз, близко для образцов с различным содержанием тербия и 
незначительно уменьшается с увеличением концентрации легированных ионов.

В табл. \ref{tab2} приведены основные оптические характеристики полученных 
образцов.

\begin{table}[h!]
\caption{Оптические параметры многокомпонентного стекла, легированного 
ионами Tb$^{3+}$}
\begin{tabularx}{\textwidth}{|Y|Y|Y|Y|Y|}
\hline
Код образца& 
E$_{g}$ \par прямой \par (еВ)& 
E$_{g}$ \par непрямой \par (еВ)& 
n& 
Время жизни, мкс \\
\hline
Tb5-Bi0& 
--& 
--& 
--& 
2325 \\
\hline
Tb3-Bi2& 
3,73& 
3,66& 
2,23& 
2330 \\
\hline
Tb2-Bi3& 
3,69& 
3,60& 
2,25& 
2325 \\
\hline
Tb1-Bi4& 
3,66& 
3,55& 
2,26& 
2370 \\
\hline
Tb0-Bi5& 
3,60& 
3,49& 
2,27& 
-- \\
\hline
\end{tabularx}
\label{tab2}
\end{table}

Для того чтобы оценить квантовый выход фотолюминесценции полученных образцов 
в~соответствии с доступными нам технологическими возможностями, применен 
относительный метод сравнения с эталонным образцом, квантовая эффективность 
которого известна и составляет около 62~{\%}~\cite{B22}. По технологии работы~\cite{B22} 
был приготовлен образец со следующим составом 16,58Li$_{2}$O ---
82,92B$_{2}$O$_{3}$ --- 0,5Tb$_{4}$O$_{7}$ мол. {\%}, обозначенный как (LBT), 
затем измерялась плотность полученного образца и сравнивалась с плотностью в 
статье. Было установлено, что плотность одинакова и равна 2,21 г/см$^{3}$ 
\textpm 0,03.

Части от каждого из образцов измельчили их в агатовой ступке, приготовили 
таблетки массой 1~г, при этом все таблетки имели одинаковую площадь 
поверхности, затем возбудили все образцы одним и тем же источником (мощность 
1~Вт, GaN, 365 нм) и зарегистрировали люминесценцию. Данные, приведенные в 
табл.~\ref{tab3}, показывают, что образцы обладают сопоставимой фотолюминесценцией 
по сравнению с образцом LBT. Площадь под кривой соответствует 
энергетическому выходу.

\begin{table}[h]
\caption{Сравнение площади области спектра излучения между образцами при 
542~нм}
\begin{tabularx}{\textwidth}{|P{55mm}|Y|Y|Y|Y|Y|}
\hline
Код образца& 
Tb1-Bi4& 
Tb2-Bi3& 
Tb3-Bi2& 
Tb5-Bi0& 
LBT \\
\hline
Интегральная площадь, отн. ед.& 
30& 
62& 
133& 
404& 
53 \\
\hline
\end{tabularx}
\label{tab3}
\end{table}

Результаты табл. \ref{tab3} указывают на тот факт, что квантовая эффективность 
излучения образцов выше по сравнению с таковой у образца LBT. Однако судить 
о величине эффективности на основе используемого этого метода нельзя, 
поскольку изменение технологии приготовления стекла с учетом погрешности 
может привести к иным результатам. Однако образцы при $x = 2$, 3 и 5 можно 
считать эффективными источником зеленого света.

\section*{Заключение}

Методом закалки из расплава получены стекла 70B$_{2}$O$_{3}$\,--\,10Na$_{2}$O\,--\,15PbO\,--\,(5-x)Bi$_{2}$O$_{3}$\,-- xTb$_{2}$O$_{3}$, где $x = 0$; 1; 2; 3; 
5 мол.~{\%}. Инфракрасная спектроскопия с преобразованием Фурье проливают 
свет на различные структурные свойства стекол. Выявлено, что полученные 
образцы имеют три основные инфракрасные полосы около $\sim 1300$~cm$^{-1}$ (BO$_{3}$ тригональные); 800~cm$^{-1}$(BO$_{4}$ 
тетраэдрические) и при 680 cm$^{-1}$ (пентаборатных групп) соответственно, 
эти комплексы характеризуют боратное стекло. Спектр пропускания доказал, что 
образцы имеют высокую прозрачность (340--1100~нм), в ультрафиолетовом 
диапазоне при уменьшении висмута в образцах наблюдалось большее пропускание. 
С увеличением концентрации тербия вместо висмута плотность образцов 
уменьшалась от 3,7 до 3,4~г/см$^{3}$ (Tb0--Bi5--Tb5--Bi0), а уменьшение 
плотности образцов приводит к увеличению ширины запрещенных зон (прямых и 
непрямых). Поглощение Tb$^{3+}$ ионы в видимом диапазоне слишком слабое (при 
460~нм) и увеличивается с~увеличением концентрации. В основном Tb$^{3+}$ 
возбуждается в ультрафиолетовом диапазоне, на что указывает спектр 
возбуждения. Эффект тушения не наблюдается с увеличением концентрации 
Tb$_{2}$O$_{3}$, соответственно, люминесценция образцов зарегистрировала 
максимальное значение при концентрации 5 мол.~{\%} Tb$^{3+}$ при 542 нм. 
Кинетика затухания люминесценции, соответствующая излучательному переходу 
$^{5}$D$_{4} \to {}^{7}$F$_{5}$ (542,0 нм) при возбуждении на 270 нм, 
позволила определить время жизни примерно в 2,3~мс. Для оценки эффективности 
полученных люминофоров провели сравнение с изученным стекла, которое имеет 
высокую эффективность спонтанного излучения на 542 нм. Доказано, что образец 
(Tb5--Bi0), имеющий двойную структурную фазу, светится значительно 
интенсивнее, что указывает на то, что полученные люминофоры имеют высокие 
фотолюминесцентные свойства и могут использоваться в различных оптических 
приложениях.

\begin{thebibliography}{22}
\bibitem{B01}
Alzahrani, J.S., Alrowaili, Z.A., Eke, C., Al-Qaisi, S., Alsufyani, S.J., 
Olarinoye, I.O., Boukhris, I., Al-Buriahi, M.S., Tb$^{3+}$-doped 
GeO$_{2}$-B$_{2}$O$_{3}$--P$_{2}$O$_{5}$--ZnO magneto-optical glasses: 
Potential application as gamma-radiation absorbers. \emph{Radiation Physics and 
Chemistry}, 2023, vol.~208(11), art.~110874. \doi{10.1016/j.radphyschem.2023.110874} 

\bibitem{B02}
Kesavulu, C.R., Kim, H.J., Lee, S.W., Kaewkhao, J., Kaewnuam, E.,  
Wantana, N., Luminescence properties and energy transfer from Gd$^{3+}$ to 
Tb$^{3+}$ ions in gadolinium calcium silicoborate glasses for green laser 
application. \emph{J. Alloys Compd.}, 2017, vol.~704, pp.~557--564. \doi{10.1016/j.jallcom.2017.02.056}

\bibitem{B03}
Swapna, K., Mahamuda, Sk., Srinivasa Rao, A., Jayasimhadri, M., Shakya, 
Suman, Prakash, G. Vijaya, Tb$^{3+}$ doped Zinc Alumino Bismuth Borate 
glasses for green emitting luminescent devices. \emph{Journal of Luminescence}, 
2014, vol.~156, pp.~180--187. \doi{10.1016/j.jlumin.2014.08.019}

\bibitem{B04}
Abbas, B.K., Baki, S.O., Leng, F.W., Abbas, H.K., Al-Sarraj, L., Mahdi, M.A., Investigation of Structural, Thermal Properties and Shielding 
Parameters of Borosilicate Glasses Doped with Dy$^{3+}$/Tb$^{3+}$ Ions for 
Gamma and Neutron Radiation Shielding Applications. \emph{Journal of Advanced 
Research in Fluid Mechanics and Thermal Sciences}, 2021, vol.~80(1), pp.~50--61. \doi{10.37934/arfmts.80.1.5061}

\bibitem{B05}
Linganna, K., Sreedhar, V.B., Jayasankar, C.K., Luminescence properties of Tb$^{3+}$ ions in zinc fluorophosphates glasses for green laser applications. \emph{Mater. Res. Bull.}, 2015, vol.~67, pp.~196--200. \doi{10.1016/j.materresbull.2015.02.062}

\bibitem{B06}
Alrowaili, Z.A., Basha, B., Alwadai, N., Olarinoye, I.O., Hammoud, A., 
Al-Buriahi, M.S., V. Stroganova, E.V., Sriwunkum, C., Experimental design and 
characterization of Eu-doped tellurite matrix glassy composite for medical 
and ionizing-radiation sensing applications. \emph{Ceramics International}, 2023, 
vol.~49, iss.~12, pp.~20772--20783. \doi{10.1016/j.ceramint.2023.03.209}.

\bibitem{B07}
Altowyan, A.S., Hammoud, A., Al-Qaisi, S., Alwadai, N., Lebedev, A.V., 
Klimenko, V.A., Vasileva, L.V., Al-Buriahi, M.S., Synthesis, XRD, UV-Vis 
spectra and photoluminescent properties of TeO$_{2}$-based glasses doped 
with Yb$^{3+}$ and Bi$^{3+}$. \emph{Optik}, 2022, vol.~268, art.~169808. 
\doi{10.1016/j.ijleo.2022.169808}

\bibitem{B08}
Farouk, M., Samir, A., Metawe, F., Elokr, M., Optical absorption and 
structural studies of bismuth borate glasses containing Er$^{3+}$ ions. 
\emph{Journal of Non-Crystalline Solids}, 2013, vol.~371--372, pp.~14--21. 
\doi{10.1016/j.jnoncrysol.2013.04.001}

\bibitem{B09}
Kumar, A., Kaur, R., Sayyed, M.I., Rashad, M., Singh, M., Ali, A.M., 
Physical, structural, optical and gamma ray shielding behavior of 
(20$+$x)PbO-10BaO-10Na$_{2}$O-10MgO-(50-x)B$_{2}$O$_{3}$ glasses. \emph{Phys. B 
Condens. Matter}, 2019, vol.~552, pp.~110--118. \doi{10.1016/j.physb.2018.10.001}

\bibitem{B10}
Vishal, K., Pandey, O.P., Singh, K., Structural and optical properties of 
barium borosilicate glasses. \emph{Phys. B Condens. Matter}, 2010, vol.~405, pp.~204--207. \doi{10.1016/j.physb.2009.08.055}

\bibitem{B11}
Laariedh, F., Sayyed, M.I., Kumar, A., Tekin, H.O., Kaur, R., Badech, 
T.-B., Studies on the structural, optical and radiation shielding properties 
of (5,0 -- x) PbO -- 10WO$_{3}$--10Na$_{2}$O -- 10MgO -- (20 $+$ 
x)B$_{2}$O$_{3}$ glasses. \emph{J. Non-Cryst. Solids}, 2019, vol.~513, pp.~159--166. \doi{10.1016/j.jnoncrysol.2019.03.007}

\bibitem{B12}
Kaur, R., Singh, S., Singh, K., Pandey, O.P., Effect of swift heavy ions 
on structural and optical properties of bismuth based alumina borosilicate 
glasses. \emph{Radiat. Phys. Chem.}, 2013, vol.~86, pp.~23--30. \doi{10.1016/j.radphyschem.2013.01.031}

\bibitem{B13}
Priyanka Goyal, P., Sharma,Y.K., Pal, S., Bind, U.C., Huang, S.C., 
Chung, S.L. The effect of SiO$_{2}$ content on structural, physical and 
spectroscopic properties of Er$^{3+}$ doped B$_{2}$O$_{3}$. J. Non-Cryst. 
Solids, 2017, vol. 463, pp. 118--127. \doi{10.1016/j.jnoncrysol.2017.03.009}

\bibitem{B14}
Alzahrani, J.S., Hammoud, A., Altowyan, A.S., Olarinoyec, I.O. Lebedev, 
A.V., Al-Buriahi, M.S., Influence of Sm/Bi substitution on synthesis, 
structural, and photon interaction properties of TeO single bond MoO$_{3}$ 
single bond BaO single bond Sm$_{2}$O$_{3}$ single bond Bi$_{2}$O$_{3}$ 
glass system. \emph{Optik}, 2023, vol.~274, art.~170507. \doi{10.1016/j.ijleo.2023.170507}

\bibitem{B15}
Tauc, J., Menth, A., Wood, D.L., Optical and Magnetic Investigations 
of the Localized States in Semiconducting Glasses. \emph{Phys. Rev. Letters}, 
1970, vol.~25, pp.~749--752. \doi{10.1103/PhysRevLett.25.749}

\bibitem{B16}
Li, X., Zhu, H., Wei, J., Wang, K., Xu, E., Li, Z., Wu, D., 
Determination of band gaps of self-assembled carbon nanotube films using 
Tauc/Davis--Mott model. \emph{Appl. Phys. A}, 2009, vol.~97, pp.~341--344. 
\doi{10.1007/s00339-009-5330-z}

\bibitem{B17}
Dimitrov, V., Sake, S., Electronic oxide polarizability and optical 
basicity of simple oxides. I. \emph{Journal of Applied Physics}, 1996, vol.~79, pp.~1736--1740. \doi{10.1063/1.360962}

\bibitem{B18}
Mohan, S., Thind, K., Optical and spectroscopic properties of neodymium 
doped cadmium-sodium borate glasses. \emph{Optics and Laser Technology}, 2017, vol.~95, pp.~36--41. \doi{10.1016/j.optlastec.2017.04.016}

\bibitem{B19}
Sayyeda, M.I., Aşkın, A., Ali, A.M., Kumard, A., Rashada, M. Alshehric, A.M., Extensive study of newly developed highly dense transparent 
PbO-WO$_{3}$-BaO-Na$_{2}$O-B$_{2}$O$_{3}$ glasses for radiation shielding 
applications. \emph{Journal of Non-Crystalline Solids}, 2019, vol.~521, art.~119521. 
\doi{10.1016/j.jnoncrysol.2019.119521}

\bibitem{B20}
Tamam, N., Alrowaili, Z.A., Hammoud, A., Lebedev, A.V., Boukhris, I., 
Olarinoye, I.O., Al-Buriahi, M.S., Mechanical, optical, and gamma-attenuation 
properties of a newly developed tellurite glass system. \emph{Optik}, 2022, vol.~266, art.~169355. \doi{10.1016/j.ijleo.2022.169355}

\bibitem{B21}
van Pieterson, L., Reid, M.F., Burdick, G.W., Meijerink, A., 
$4f^{n}\to 4f^{n-1} 5d$ transitions of the heavy lanthanides: Experiment and 
theory. \emph{Physical review B}, 2002, vol.~65, art.~045114. \doi{10.1103/PhysRevB.65.045114}

\bibitem{B22}
Rimbacha, A.C., Steudel, F., Ahrens, B., Schweizer, S., Tb$^{3+}$, Eu$^{3+}$, and Dy$^{3+}$ doped lithium borate and lithium aluminoborate 
glass: Glass properties and photoluminescence quantum efficiency. \emph{Journal of 
Non-Crystalline Solids}, 2018, vol.~499, pp.~380--386. 
\doi{10.1016/j.jnoncrysol.2018.07.029}
\end{thebibliography}
\end{document}
