% !TEX TS-program = pdflatexmk
\documentclass[press]{vestnik}

\draft{3}

\OJS{1060}
\EDN{RANMTJ}

\rubric{Механика}
\begin{document}

\udc{519.634}

\titlerus[Схлопывание осесимметричной каверны при медленных движениях цилиндра в~жидкости\ldots]{Схлопывание осесимметричной каверны при медленных движениях цилиндра в~жидкости после отрывного удара}

\addtotocrus{Схлопывание осесимметричной каверны при медленных движениях цилиндра в~жидкости после отрывного удара}

\titleeng[Collapse of an axisymmetric cavity during slow movements of a cylinder in~a~liquid after a separation\ldots]{Collapse of an axisymmetric cavity during slow movements of a cylinder in a liquid after a separation impact}

\addtotoceng{Collapse of an Axisymmetric Cavity During Slow Movements of a Cylinder in~a~Liquid After a Separation Impact}

\authorrus*{Норкин}{Михаил Викторович}
\authoreng*{Norkin}{Mikhail V.}
\inforus{д-р физ.-мат. наук, доцент, профессор кафедры вычислительной математики и математической физики Южного федерального университета}
\address{Служебный адрес: 344000, г. Ростов-на-Дону, Южный федеральный университет, 
Институт математики, механики и компьютерных наук им. И.И. Воровича. 
Мильчакова, 8а. Домашний адрес: 344010, г. Ростов-на-Дону, ул. Пушкинская, 135, кв.5а; тел. 
89045052384.}
\orcid{0000-0002-9508-5996}
\email{norkinmi@mail.ru}
\spin{1638-4892}

\affilrus{\orgname{Южный федеральный университет}, ул. Большая Садовая 105/42, \city{Ростов-на-Дону}, 344000, \country{Россия}}
\affileng{\orgname{Southern Federal University}, Bolshaya Sadovaya str., 105/42, \city{Rostov-on-Don}, 344000, \country{Russia}}

\reviewer{Маклаков}{Дмитрий Владимирович}
\inforev{д-р физ.-мат. наук, профессор Институт математики и механики им. Н.И. Лобачевского Казанского (Приволжского) федерального университета}
\emailrev{dmaklak@kpfu.ru}
\affilrev{\orgname{Казанский (Приволжский) федеральный университет}, \city{Казань}, \country{Россия}}

\review{В работе исследуется осесимметричная задача о движении полностью погруженного в жидкость цилиндра конечных размеров после вертикального удара. В момент времени непосредственно после удара цилиндр мгновенно приобретает вертикальную скорость V0, и на его поверхности согласно модели удара Л.И. Седова образуется заранее неизвестная зона отрыва с нулевым импульсным давлением. Отметим, что задача об определении этой зоны сама по себе представляет значительные математические трудности. В данной статье основное внимание автора сосредоточено на том, что произойдет после удара. Ранее аналогичная плоская задача исследовалась в работе автора (Норкин, М.В., Асимптотика медленных движений прямоугольного цилиндра в жидкости после отрывного удара. Ученые записки Казанского университета. Серия Физико-математические науки, 2020, т. 162, кн. 4, с. 426–440). С течением времени зона отрыва трансформируется в каверну, внутри которой давление считается заданным и постоянным. Деформируется также изначально горизонтальная свободная поверхность. Применяя весьма тонкие асимптотические методы, автор получает аналитические формулы для формы каверны и приближённо определяет момент схлопывания. Работа интересная и содержит новые результаты. В качестве замечания отмечу, что для наглядности изложения и лучшего понимания результатов, особенно касающихся момента схлопывания, было бы уместно привести графики функции $c_0(\tau)$ для нескольких значений параметров $\chi$, $b$, $H$, $R_b$ и $H_B$. Считаю, что работа может быть опубликована в журнале с учётом сделанного замечания.}

\annotationrus{Рассматривается осесимметричная задача о вертикальном и отрывном ударе цилиндра конечных размеров под свободной поверхностью идеальной несжимаемой тяжелой жидкости. Предполагается, что после удара цилиндр движется вглубь жидкости с постоянной скоростью. Позади тела образуется присоединенная каверна и появляется новая внутренняя свободная граница жидкости. Требуется изучить процесс схлопывания каверны при малых скоростях движения цилиндра, которые соответствуют небольшим числам Фруда. В главном асимптотическом приближении формулируется задача с~односторонними ограничениями, на основе которой определяется динамика круговой линии отрыва и время схлопывания тонкой каверны. Проводится асимптотический анализ формы каверны с~учетом решений типа пограничного слоя вблизи линии отрыва. Показывается, что при малых числах Фруда внутренняя свободная граница подходит к линии отрыва под прямым углом.}

\keywordsrus{отрывной удар цилиндра, динамика зоны отрыва, схлопывание каверны, малые числа Фруда, решения типа пограничного слоя}

\annotationeng{The axisymmetric problem of the vertical and separation impact of a cylinder of finite dimensions under the free surface of an ideal, incompressible, heavy fluid is considered. It is assumed that after the~impact the cylinder moves deep into the liquid at a constant speed. An attached cavity is formed behind the body and a new internal free fluid boundary appears. It is required to study the process of~cavity collapse at low cylinder speeds, which correspond to small Froude numbers. In the main asymptotic approximation, a problem with one-sided constraints is formulated, on the basis of which the dynamics of~the~circular separation line and the collapse time of a thin cavity are determined. An asymptotic analysis of the cavity shape is carried out taking into account solutions such as a boundary layer near the separation line. It is shown that at low Froude numbers the internal free boundary approaches the separation line at~a~right angle.}

\keywordseng{separation impact of a cylinder, dynamics of the separation zone, cavity collapse, small Froude numbers, boundary layer type solutions}

\date{27-05-2024}
\revised{10-06-2024}
\accepted{12-06-2034}

\maketitle


\section*{Введение}

Задача об ударе плавающего тела с учетом отрыва частиц жидкости от его 
поверхности относится к числу классических задач современной гидродинамики~\cite{B01}. Представляя большой самостоятельный интерес для аналитических 
исследований, она дает начальные условия для решения более сложной 
динамической кавитационной задачи. Сразу после удара начинает формироваться 
присоединенная каверна и образуется внутренняя свободная граница жидкости. 
Давление в каверне предполагается постоянным и равным давлению насыщенных 
паров жидкости или газа (или давлению газа при искусственной кавитации). 
Требуется изучить динамику каверны и влияние на нее характерных параметров 
задачи. Одним из направлений исследования таких задач является 
асимптотический анализ на малых временах. Такой подход позволяет определить 
форму каверны на некотором начальном этапе движения тела в жидкости 
(подробная библиография приводится в~\cite{B02}). Изучение дальнейшей динамики, в 
общем случае, остается сложной и малоизученной задачей. В связи с этим 
представляют интерес такие подходы, в которых нет серьезных ограничений на 
продолжительность времени рассматриваемого процесса. В работах~\cite{B03,B04} данные 
задачи изучались при дополнительном предположении о малости скорости 
движения тела после удара (что соответствует небольшим числам Фруда). При 
таком допущении возмущения свободных границ жидкости являются 
незначительными и процесс схлопывания каверны, в основном, сводится к 
изучению динамики точек отрыва. Следует отметить, что несмотря на малость 
поперечных размеров каверны, определение ее формы также представляет 
определенный интерес. Ранее такие задачи исследовались только в плоской 
постановке (круг и прямоугольник~\cite{B03,B04}). В настоящей статье дается обобщение 
этих результатов на пространственную осесимметричную задачу. Показывается, 
что специальные погранслойные решения, построенные для возмущения внутренней 
свободной границы жидкости, также хорошо аппроксимируют внешнее разложение, 
как и аналогичные решения в плоском случае. 

Среди близких направлений исследования отметим задачи проникания твердых тел 
в~жидкость с учетом отрыва частиц жидкости от их поверхностей~\cite{B05}; задачи 
подводного старта ракет кавитационным способом~\cite{B06}; экспериментальное 
изучение ударного воздействия жидкости на твердые стенки в условиях 
кавитации~\cite{B07,B08}. 

Общие принципы кавитационных течений при взаимодействии твердых тел с 
жидкостью изложены в~\cite{B09,B10}. 


\section{Постановка задачи}

\begin{figure}[b]
\centering
\includegraphics[width=.25\textwidth]{NorkinMV(Fig1)}
\caption{Схема течения жидкости в меридиональной плоскости $r$, $z$}
\captionf{Scheme of fluid flow in the meridional plane}
\label{fig1}
\end{figure}

Рассматривается осесимметричная задача об отрывном ударе цилиндра конечных 
размеров под свободной поверхностью идеальной несжимаемой тяжелой жидкости~\cite{B01}. Предполагается, что после удара цилиндр движется вглубь жидкости с 
постоянной скоростью. За телом образуется присоединенная каверна, форма 
которой зависит от физических и геометрических параметров задачи. Требуется 
изучить процесс схлопывания каверны при малых скоростях движения цилиндра, 
которые соответствуют небольшим числам Фруда. Математическая постановка 
задачи, записанная в безразмерных переменных в подвижной системе координат, 
связанной с цилиндром, имеет вид (рис.~\ref{fig1})

\begin{equation}\label{1.1}
\Delta \varphi =0, \quad  R\in \Omega (t),
\end{equation}
\begin{equation}\label{1.2}
\frac{\partial \varphi }{\partial n}=-n_{z} ,
\quad
R\in S_{11} (t),
\end{equation}
\begin{equation}\label{1.3}
\frac{\partial \varphi }{\partial \tau }+\varepsilon^{2}\frac{\partial \varphi }{\partial z}+0,5\varepsilon^{2}\left( {\nabla \varphi } \right)^{2}+b+\zeta -\varepsilon^{2}\tau -H-0,5\chi =0,
\quad
R\in S_{12} (t),
\end{equation}
\begin{equation}\label{1.4}
\frac{\partial \varphi }{\partial z}+1=\frac{\partial \zeta }{\partial r}\frac{\partial \varphi }{\partial r}+\varepsilon^{-2}\frac{\partial \zeta }{\partial \tau },
\quad
R\in S_{12} (t),
\end{equation}
\begin{equation}\label{1.5}
\frac{\partial \varphi }{\partial \tau }+\varepsilon^{2}\frac{\partial \varphi }{\partial z}+0,5\varepsilon^{2}\left( {\nabla \varphi } \right)^{2}+\psi =0,
\quad
R\in S_{2} (t),
\end{equation}
\begin{equation}\label{1.6}
\frac{\partial \varphi }{\partial z}=\frac{\partial \psi }{\partial r}\frac{\partial \varphi }{\partial r}+\varepsilon^{-2}\frac{\partial \psi }{\partial \tau },
\quad
R\in S_{2} (t),
\end{equation}
\begin{equation}\label{1.7}
\frac{\partial \varphi }{\partial z}=0,
\quad
z=-H_{b} +\varepsilon^{2}\tau ;
\quad
\frac{\partial \varphi }{\partial r}=0,
\quad
r=R_{b},
\end{equation}
\begin{equation}\label{1.8}
\varphi (r,z,0)=\varphi_{0} (r,z),
\quad
\psi (r,0)=0,
\quad
\zeta (r,0)=0.
\end{equation}

Потенциал скоростей $\varphi_{0} =\varphi_{0} (r,z)$, приобретенных частицами 
жидкости в момент, непосредственно следующий после удара (в начальный момент 
времени), а также первоначальная зона отрыва находятся на основе решения 
классической модели удара с отрывом~\cite{B01}: 
\begin{equation}\label{1.9}
\Delta \varphi_{0} =0,
\quad
R\in \Omega (0);
\quad
\varphi_{0} =0,
\quad
z=H,
\end{equation}
\begin{equation}\label{1.10}
\frac{\partial \varphi_{0} }{\partial n}=-n_{z} ,
\quad
\varphi_{0} \leqslant 0,
\quad
R\in S_{11} (0),
\end{equation}
\begin{equation}\label{1.11}
\frac{\partial \varphi_{0} }{\partial n}\geqslant -n_{z} ,
\quad
\varphi_{0} =0,
\quad
R\in S_{12} (0),
\end{equation}
\begin{equation}\label{1.12}
\frac{\partial \varphi_{0} }{\partial z}=0,
\quad
z=-H_{b} ;
\quad
\frac{\partial \varphi_{0} }{\partial r}=0,
\quad
r=R_{b}.
\end{equation}

В поставленной задаче используется растянутое время $\tau $, связанное с 
безразмерным временем $t$ равенством $t=\varepsilon \tau $. Целесообразность 
его введения обьясняется быстротой процесса схлопывания каверны после удара 
(обычно он составляет сотые или десятые доли секунды), а также 
необходимостью согласования порядков рассматриваемых величин в главном 
асимптотическом приближении. 

Основными безразмерными параметрами задачи являются числа Фруда и 
кавитации
\[
\varepsilon =Fr=\frac{V_{0} }{\sqrt {ga} },
\quad
\chi =2\frac{p_{a} -p_{c} }{\rho ga},
\]
где $p_{a} $~--- атмосферное давление; $p_{c} =\const$~--- давление в 
каверне; $g$~--- ускорение свободного падения; $\rho $~--- плотность жидкости. 

Скорость цилиндра $V_{0} $ предполагается малой величиной по сравнению с 
характерной скоростью $V=\sqrt {ga} $. Малый параметр $\varepsilon $, по 
которому строится асимптотическое разложение, фактически совпадает с числом 
Фруда.

Безразмерные переменные вводятся с помощью равенств
\[
{t}'=\frac{a}{V}t,
\quad
{x}'=ax,
\quad
{y}'=ay,
\quad
{z}'=az,
\quad
{\varphi }'=aV\varphi ,
\quad
{p}'=\rho V^{2}p,
\]
где штрихами помечаются размерные величины.

Неподвижные координаты $X$, $Y$, $Z$ связаны с подвижными $x,y,z$ соотношениями
$X=x$, $Y=y$, $Z=z+h(t)$, где $h(t)$~--- закон движения цилиндра. 
Предполагается, что ось $z$ направлена против вектора силы тяжести, начало 
координат находится в центре цилиндра. 

Функции $\varphi$, $\zeta$, $\psi $ выражаются через потенциал скоростей $\Phi $, а 
также возмущения внутренней и внешней свободных границ жидкости $\eta $ и 
$\xi $ по формулам
\[
\varphi (r,z,\tau )=\varepsilon^{-1}\Phi (r,z,\varepsilon \tau ),
\quad
\zeta (r,\tau )=\eta (r,\varepsilon \tau ),
\quad
\psi (r,\tau )=\xi (r,\varepsilon \tau ).
\]

В статье также используются следующие обозначения: $\Omega (t)$~--- область 
течения жидкости; $S_{11} (t)=\left\{ {z=b,c(t)\leqslant r\leqslant 1} \right\}\cup 
\left\{ {z=-b,\ 0\leqslant r\leqslant 1} \right\}\dm\cup \left\{ {r=1,\ -b\leqslant z\leqslant b} \right\}$~--- часть поверхности цилиндра, на которой не происходит отрыва частиц 
жидкости; $S_{12} (t)=\{ z=b+\eta (r,t)$, $0< r < c(t) \}$~--- 
внутренняя свободная граница жидкости (граница каверны); $z\dm=H+\xi 
(r,t)+\varepsilon t$~--- уравнение внешней свободной границы; $\bm V=(0, 0, -V_{0} )$~--- скорость, приобретенная цилиндром в результате 
удара ($V_{0} >0)$; $h(t)=-\varepsilon t$~--- безразмерный закон движения 
цилиндра; $a$, $2b$~--- радиус и высота цилиндра (после обезразмеривания для 
$b$ сохраняется прежнее обозначение); $R_{b} $~--- радиус цилиндрического 
бассейна, в котором находится плавающий цилиндр; $z=-H_{b} $~--- его дно; 
$R$~--- радиус-вектор с цилиндрическими координатами $(r,z)$.

На линии пересечения внутренней свободной границы жидкости с поверхностью 
цилиндра (на круговой линии отрыва) ставится условие Кутты--Жуковского, 
означающее, что скорость жидкости на ней должна быть конечной.


\section{Асимптотика медленных движений}

Решение задачи (\ref{1.1})--(\ref{1.12}) будем искать в виде следующих асимптотических 
разложений: 
\begin{equation}
\label{eq1}
\varphi (r,z,\tau )=\varphi_{1} (r,z,\tau )+\ldots,
\quad
c(t)=c(\varepsilon \tau )=c_{0} (\tau )+\ldots,
\end{equation}
\begin{equation}
\label{eq2}
\zeta (r,\tau )=\varepsilon^{2}\zeta_{1} (r,\tau )+\ldots,
\quad
\psi (r,\tau )=\varepsilon^{2}\psi_{1} (r,\tau )+\ldots,
\end{equation}
где многоточием обозначены члены более высокого порядка малости по 
$\varepsilon $.

Подставляя разложения (\ref{eq1})--(\ref{eq2}) в уравнение и граничные условия задачи 
(\ref{1.1})--(\ref{1.8}),  с~помощью формулы Тейлора осуществляя перенос краевых 
условий с возмущенных участков границы области $\Omega (t)$ на первоначально 
невозмущенный уровень, а затем приравнивая коэффициенты при одинаковых 
степенях $\varepsilon $, придем в главном приближении к смешанной краевой 
задаче теории потенциала в области $\Omega (0)$. В предположении, что 
круговая линия отрыва монотонно стягивается в точку, проинтегрируем 
полученное динамическое условие в зоне отрыва по времени от $0$ до $\tau $ 
при фиксированном $r\in (0,c_{0} (\tau ))$. Интегрируя еще динамическое 
условие на внешней свободной границе, придем к следующей задаче:
\begin{equation}
\label{eq3}
\Delta \varphi_{1} =0,
\quad
R\in \Omega (0);
\quad
\varphi_{1} =0,
\quad
z=H,
\end{equation}
\begin{equation}
\label{eq4}
\varphi_{1} =(0,5\chi +H-b)\tau ,
\quad
z=b,
\quad
0<r<c_{0} (\tau ),
\end{equation}
\begin{equation}
\label{eq5}
\frac{\partial \varphi_{1} }{\partial z}=-1,
\quad
z=b,
\quad
c_{0} (\tau )<r<1,
\end{equation}
\begin{equation}
\label{eq6}
\frac{\partial \varphi_{1} }{\partial z}=-1,
\quad
z=-b,
\quad
0<r<1;
\quad
\frac{\partial \varphi_{1} }{\partial r}=0,
\quad
r=1,
\quad
-b<z<b,
\end{equation}
\begin{equation}
\label{eq7}
\frac{\partial \varphi_{1} }{\partial z}=0,
\quad
z=-H_{b} ;
\quad
\frac{\partial \varphi_{1} }{\partial r}=0,
\quad
r=R_{b} .
\end{equation}
В силу неизвестности $c_{0} (\tau )$ задача (\ref{eq3})--(\ref{eq7}) является 
нелинейной и относится к классу задач со свободными границами. Радиус 
круговой линии отрыва $c_{0} (\tau )$ в каждый момент времени определяется 
из условия Кутты--Жуковского, которое равносильно системе неравенств
\begin{equation}
\label{eq8}
(0,5\chi +H-b)\tau -\varphi_{1} +\varphi_{0} \geqslant 0,
\quad
z=b,
\quad
c_{0} (\tau )<r<1,
\end{equation}
\begin{equation}
\label{eq9}
\frac{\partial \varphi_{1} }{\partial z}\geqslant -1,
\quad
z=b,
\quad
0<r<c_{0} (\tau ).
\end{equation}
Неравенство (\ref{eq8}) является следствием динамического условия $p\geqslant p_{c} $, 
означающего что давление в зоне контакта не может опускаться ниже давления в 
каверне. Кинематическое условие (\ref{eq9}) говорит о том, что жидкие частицы не 
входят внутрь твердого тела. Начальное условие $\varphi_{1} (r,z,0)=\varphi_{0} 
(r,z)$ выполняется, поскольку при $\tau =0$ получается задача, совпадающая с 
классической моделью удара с отрывом (\ref{1.9})--(\ref{1.12}).

\begin{figure}
\centering
\includegraphics[width=.35\textwidth]{NorkinMV(Fig2)(new)}
\caption{График функции $C_0(\tau)$}
\captionf{Function graph  $C_0(\tau)$}
\label{fig2}
\end{figure}

Задача с односторонними ограничениями (\ref{eq3})--(\ref{eq9}) решается численно с 
помощью специального итерационного метода, применявшегося ранее при 
исследовании плоских задач~\cite{B02,B03,B04}. Приведем численные значения величины 
$c_{0} (\tau )$ для некоторых $\tau $ ($\chi =0$, $b=0,1$, $H=1$, $R_{b} 
=4$, $H_{b} =3)$: $c_{0} (0)=0,951$; $c_{0} (0,2)=0,841$; $c_{0} 
(0,4)=0,65$; $c_{0} (0,5)=0,502$; $c_{0} (0,6)=0,261$; $c_{0} (0,63)=0,117$; 
$c_{0} (0,635)=0,067$. При $\tau =0,64$ зона отрыва не видна (происходит 
схлопывание тонкой каверны). График функции $C_0(\tau)$ изображен на рис.~\ref{fig2}. 

На основании решения задачи (\ref{eq3})--(\ref{eq9}) находятся главные приближения для 
возмущений внутренней и внешней свободных границ жидкости. Для функций 
$\zeta_{1} (r,\tau )$, $\psi_{1} (r,\tau )$ справедливы следующие 
представления:
\begin{equation}
\label{eq10}
\zeta_{1} (r,\tau )=\int\limits_0^\tau {\left( {\frac{\partial \varphi_{1} }{\partial z}+1} 
\right)} d\tau ,
\quad
z=b;
\quad
\psi_{1} (r,\tau )=\int\limits_0^\tau {\frac{\partial \varphi_{1} }{\partial z}d\tau } ,
\quad
z=H.
\quad
\end{equation}
Далее вкратце остановимся на построении двух специальных погранслойных 
решений для функции $\zeta (r,\tau )$. Первое из них дает хорошую 
аппроксимацию внешнего разложения (\ref{eq2}) практически во всей зоне отрыва, а 
второе поправляет это разложение в маленькой окрестности линии отрыва 
(условие $\zeta_{1} (c_{0} (\tau ),\tau )=0$ не выполняется).

Учитывая регулярность решения задачи с односторонними ограничениями (\ref{1.9})--(\ref{1.12}) и тот факт, что производная функции $\varphi_{1} $ по $r$ обращается в 
ноль при $z=b$, $0<r<c_{0} (\tau )$, заменим производные функции $\varphi_{1} $ 
по $z$ и $r$ в кинематическом уравнении (\ref{1.4}) асимптотическими формулами
\begin{equation}
\label{eq11}
\frac{\partial \varphi_{1} }{\partial z}+1\sim \beta (\tau )\sqrt {c_{0} (\tau 
)-r} ,
\quad
r\to c_{0} (\tau )-0,
\end{equation}
\[
\frac{\partial \varphi_{1} }{\partial r}=\frac{\partial^{2}\varphi_{1} 
}{\partial r\partial z}(r,b,\tau ) \zeta +\ldots \sim \frac{-\beta (\tau 
)}{2\sqrt {c_{0} (\tau )-r} }\zeta ,
\quad
r\to c_{0} (\tau )-0,
\]
где коэффициент $\beta (\tau )$ находится численно.

В результате придем к уравнению, содержащему только одну неизвестную функцию 
$\zeta (r,\tau )$. Вначале, следуя классическому подходу, представим искомую 
функцию в виде
\begin{equation}
\label{eq12}
\zeta (r,\tau )=\varepsilon^{7/2}\frac{\beta (\tau )}{{c}'_{0} (\tau 
)}F(\tilde{{r}})+\ldots,
\quad
F(\tilde{{r}})=\frac{2}{3}\tilde{{r}}^{3/2},
\quad
\tilde{{r}}=
\frac{с_{0} (\tau )-r}{\varepsilon }.
\end{equation}
Подставляя (\ref{eq12}) в дифференциальное уравнение внутренней свободной границы, 
осуществляя переход к погранслойной переменной $\tilde{{r}}$ и, используя 
приведенные выше асимптотические формулы, видим, что функция $\zeta (r,\tau 
)$ удовлетворяет этому уравнению в главном асимптотическом приближении (левая 
часть дифференциального уравнения и производная по времени имеют порядок 
$\sqrt \varepsilon $, а нелинейный член имеет более высокий порядок 
малости). Также можно убедиться в том, что выполняется условие сращивания 
для производной $\zeta_{\tau } (r,\tau )$. В самом деле, из (\ref{eq12}) следует 
асимптотическая формула 
\[
\zeta_{\tau } (r,\tau )=\varepsilon^{5/2}\beta (\tau )\sqrt {\tilde{{r}}} 
+\ldots,
\]
которая при больших $\tilde{{r}}$ должна переходить в соответствующее внешнее 
разложение. Последнее вытекает из того, что главный член асимптотики 
производной внешнего разложения (\ref{eq2}) при $r\to c_{0} (\tau )-0$ после 
перехода к внутренней переменной $\tilde{{r}}$ будет иметь такой же вид 
(получается на основании формул (\ref{eq10}), (\ref{eq11})). Не останавливаясь на 
подробном анализе остальных условий заметим, что функция, определяемая 
формулами (\ref{eq12}), не может быть использована в качестве погранслойной. Это 
обьясняется тем, что в силу отрицательности коэффициента ($\beta (\tau )>0$, 
${c}'_{0} (\tau )<0)$ соответствующая осесимметричная поверхность лежит 
внутри цилиндра. Отметим также, что функция (\ref{eq12}) не удовлетворяет 
однородному начальному условию и~условию сращивания с внешним разложением 
(хотя однородное условие на линии отрыва выполняется). Таким образом, эту 
функцию необходимо подправить. Поскольку условие сращивания справедливо для 
производной $\zeta_{\tau } (r,\tau )$, то можно попытаться восстановить 
функцию $\zeta (r,\tau )$с помощью интегрирования по времени. В силу того, 
что сама зона сращивания зависит от времени, неправильно будет выполнить 
интегрирование по времени с произвольной постоянной при фиксированном $r$ 
(то есть, добавить к (\ref{eq12}) произвольную постоянную). Если все-таки так 
сделать и определить эту постоянную из начального условия, то полученное 
решение будет плохо согласовываться с внешним разложением. Однако можно 
провести локальное интегрирование за весьма маленький временной промежуток в 
окрестности фиксированного $\tau $ (при этом $r$ считаем фиксированным и 
принадлежащем зоне сращивания). При интегрировании возникает произвольная 
постоянная, которая для каждого бесконечно малого временного промежутка 
будет своя. В результате получим функцию от времени, которую представим в~виде произведения временного множителя в (\ref{eq12}) на некоторую величину 
$\gamma $. Определяя $\gamma$ из начального условия, придем к 
следующему решению типа пограничного слоя
\begin{equation}
\label{eq13}
\zeta (r,\tau )=\varepsilon^{7/2}\frac{\beta (\tau )}{{c}'_{0} (\tau 
)}\left[ {F(\tilde{{r}})-F(\tilde{{r}}_{0} )} \right] ,
\quad
\tilde{{r}}=\frac{c_{0} (\tau )-r}{\varepsilon } ,
\quad
\tilde{{r}}_{0} =\frac{c_{0} (0)-r}{\varepsilon }.
\end{equation}
Теперь проведем другие рассуждения, подтверждающие справедливость формулы 
(\ref{eq13}). \mbox{Поскольку} решение типа пограничного слоя строится при фиксированном 
$\tau $, то можно с самого начала искать решение в форме (\ref{eq13}) с 
замороженным коэффициентом $\beta (\tau )[{c}'_{0} (\tau )]^{-1}$ (при этом 
зависимость от времени будет осуществляться только через погранслойную 
переменную). В этом случае функция (\ref{eq13}) будет удовлетворять в главном 
приближении кинематическому дифференциальному уравнению (\ref{1.4}) для данного 
фиксированного момента $\tau $ и, кроме этого, будет выполнено условие 
сращивания для производной $\zeta_{\tau } (r,\tau )$ (также для данного 
$\tau$). \mbox{Последние} рассуждения теряют силу в случае, если 
$F(\tilde{{r}}_{0} )$ заменить на величину, зависящую от времени.

Справедливость условия сращивания для самой функции $\zeta $ проверяется 
численно на различных примерах. Отметим, что начальное условие при $\tau =0$ 
будет выполнено точно, а условие на линии отрыва по-прежнему не выполняется 
(хотя соответствующая кривая подходит к~твердой границе ближе, чем внешнее 
разложение).

На рис.~\ref{fig3} показано согласование формулы (\ref{eq13}) с внешним разложением при 
$\varepsilon =0,4$, $\tau =0,4$, $\chi =0$, $b=0,1$, $H=1$, $R_{b} =4$, 
$H_{b} =3$. Рис.~\ref{fig4} соответствует случаю $\varepsilon =0,3$ (остальные 
параметры не меняются). Из приведенных рисунков видно, что при уменьшении 
параметра $\varepsilon $ происходит сближение соответствующих кривых вблизи 
точки отрыва. Последнее полностью соответствует условию сращивания в главном 
приближении. Заметим, что хорошее согласование получается не только вблизи 
точки отрыва, но также и во всей зоне отрыва. Таким образом, для определения 
формы внутренней свободной границы жидкости при малых $\varepsilon $ можно 
использовать формулу (\ref{eq13}). 

\begin{figure}
\centering
\includegraphics[width=.7\textwidth]{NorkinMV(Fig2)}
\caption{Согласование погранслойного решения (сплошная линия) и внешнего 
разложения (пунктирная линия) при $\varepsilon =0,4$; $\tau =0,4$ в 
меридиональной плоскости $r$, $z$}
\captionf{Agreement between the boundary layer solution (solid line) and the external expansion (dashed line) at~$\varepsilon =0,4$; $\tau =0,4$ in the meridional plane $r$, $z$}
\label{fig3}
\end{figure}

\begin{figure}
\centering
\includegraphics[width=.7\textwidth]{NorkinMV(Fig3)}
\caption{Согласование погранслойного решения (сплошная линия) и внешнего 
разложения (пунктирная линия) при $\varepsilon =0,3$; $\tau =0,4$ в 
меридиональной плоскости $r$, $z$}
\captionf{Agreement between the boundary layer solution (solid line) and the external expansion (dashed line) at~$\varepsilon =0,3$; $\tau =0,4$ in the meridional plane $r$, $z$}
\label{fig4}
\end{figure}

Теперь построим второе решение типа пограничного слоя, которое будет 
удовлетворять необходимому условию при $r=c_{0} (\tau )$ и согласовываться с 
первым решением. Искомую функцию будем искать в виде
\[
\zeta (r,\tau )=\varepsilon^{\alpha }F(\tilde{{r}})+\ldots,
\quad
\tilde{{r}}=\frac{c_{0} (\tau )-r}{\varepsilon^{\lambda }},
\quad
\alpha >0, \quad \lambda >0.
\]
Подставляя этот проект решения в кинематическое уравнение (\ref{1.4}) и учитывая 
асимптотические формулы (\ref{eq11}), придем к равенству, в которое входят 
слагаемые, имеющие следующие порядки малости по $\varepsilon $: $\varepsilon 
^{0,5\lambda }$, $\varepsilon^{2\alpha -1,5\lambda }$, $\varepsilon 
^{\alpha -2-\lambda }$. Если все показатели равны, то величина $\lambda =-4$. 
Поэтому рассмотрим случаи, когда равными оказываются только два показателя 
степени а третье слагаемое имеет более высокий порядок малости по 
$\varepsilon $. Возможны три случая: $0,5\lambda =2\alpha -1,5\lambda $, 
$0,5\lambda =\alpha -2-\lambda $, $2\alpha -1,5\lambda =\alpha -2-\lambda $. 
В первом случае $\lambda <-4$. Во~втором~--- получается решение с отрицательным 
коэффициентом, уходящее внутрь твердого тела. Поэтому остается третий 
случай, который приводит к следующему выражению для функции $F(\tilde{{r}})$ 
($\lambda =2\alpha +4)$:
\[
F(\tilde{{r}})=-\frac{2{c}'_{0} (\tau )}{\beta (\tau )}\sqrt {\tilde{{r}}} .
\]
Возвращаясь к исходной переменной, получим для возмущения свободной границы 
вблизи линии отрыва асимптотическую формулу
\[
\zeta (r,\tau )\sim -\varepsilon^{-2}\frac{2{c}'_{0} (\tau )}{\beta (\tau 
)}\sqrt {c_{0} (\tau )-r} .
\]
Отметим, что при малых $\varepsilon $ область применимости этого решения 
будет очень маленькой. Соответствующая кривая выходит из точки отрыва под 
прямым углом и практически не отличается от вертикального отрезка прямой. 
Сращивание двух решений типа пограничного слоя в промежуточном пределе может 
быть выполнено при надлежащем выборе соответствующих параметров. 

\section*{Заключение}

В работе представлен численно-аналитический метод определения формы 
присоединенной каверны при медленных движениях цилиндра в жидкости после 
отрывного удара. В главном асимптотическом приближении сформулирована задача 
с односторонними ограничениями, на основе которой находится динамика 
круговой линии отрыва и время схлопывания тонкой каверны. Для определения 
формы внутренней свободной границы жидкости построены специальные 
погранслойные решения и показано их хорошее согласование с внешним 
разложением. 

\begin{thebibliography}{10}
\bibitem{B01}
Седов, Л.И., \textit{Плоские задачи гидродинамики и аэродинамики}. Москва, Наука, 1980. \altbib{Sedov, L.I., \textit{Ploskie zadachi gidrodinamiki i aerodynamiki = Plane problems of hydrodynamics and aerodynamics.} Moscow, Nauka, 1980. (in Russian)}

\bibitem{B02}
Норкин, М.В., Образование каверны при наклонном отрывном ударе кругового цилиндра под свободной поверхностью тяжелой жидкости. \textit{Сибирский журнал индустриальной математики}, 2016, т.~19, №~4, с.~81--92. \altbib{Norkin, M.V., Cavity formation at the inclined separated impact on a circular sylinder under a free surface of a heavy liquid. \textit{Sibirskiy zhurnal industrial'noy matematiki = Journal of Applied and Industrial Mathematics}, 2016, vol. 10, no. 4, pp.~538--548. (in Russian)} \doi{10.1134/S1990478916040104}

\bibitem{B03}
Норкин, М.В., Динамика точек отрыва при ударе плавающего кругового цилиндра. \textit{Прикладная механика и техническая физика}, 2019, т.~60, №~5, с.19--27. \altbib{Norkin, M.V., Dynamics of separation points upon impact of a floating circular cylinder. \textit{Prikladnaya mekhanika i tekhnicheskaya fizika = Journal of Applied Mechanics and Technical Physics}, 2019, vol.~60, no.~5, pp.~798--804. (in Russian)} \doi{10.15372/PMTF20190503}

\bibitem{B04}
Норкин, М.В., Асимптотика медленных движений прямоугольного цилиндра в жидкости после отрывного удара. \textit{Ученые записки Казанского университета. Серия Физико-математические науки}, 2020, т.~162, кн.~4, с.~426--440. \altbib{Norkin, M.V., Asymptotics of slow motions of a rectangular cylinder in a liquid after a separation impact. \textit{Uchenye zapiski Kazanskogo universiteta. Seriya fiziko-matematicheskie nauki = Scientific Notes of Kazan University. Series Physical and Mathematical Sciences}, 2020, vol.~162, no.~4, pp.~426--440. (in Russian)} \doi{10.26907/2541-7746.2020.4.426-440}

\bibitem{B05}
Reinhard, M, Korobkin, A.A., Cooker, M.J., Cavity formation on the surface of a body entering water with deceleration. \textit{Journal of Engineering Mathematics}, 2016, vol.~96(1), pp.~155--174. \doi{10.1007/s10665-015-9788-8}

\bibitem{B06}
Пегов, В.И., Мошкин, И.Ю., Расчет гидродинамики кавитационного способа старта ракет. \textit{Челябинский физико-математический журнал}, 2018, т.~3, №~4, с.~476--485. \altbib{Pegov, V.I., Moshkin, I.Yu, Analysis of Fluid Dynamics of Cavitational Launch Technique. \textit{Chelyabinskiy fiziko-matematicheskiy zhurnal = Chelyabinsk Physical and Mathematical Journal}, 2018, vol.~3, no.~4, pp.~476--485. (in Russian)} \doi{10.24411/2500-0101-2018-13408}

\bibitem{B07}
\textit{Bergant, A., Simpson, A.R., Tijsseling, A.S.}, Water hammer with column separation: A~historical review. \textit{Journal of Fluids and Structures}, 2006, vol.~22, no.~2, pp.~135--171. \doi{10.1016/j.jfluidstructs.2005.08.008}

\bibitem{B08}
Аганин, А.А., Ильгамов, М.А., Мустафин, И.Н., Ударная кавитация жидкости в цилиндрической емкости. \textit{Ученые записки Казанского университета. Серия физико-математические науки}, 2020, т.~162, кн.~1. c.~27--37. \altbib{Aganin, A.A., Ilgamov, M.A., Mustafin, I.N., Impact-induced cavitation in a cylindrical container with liquid. \textit{Uchenye zapiski Kazanskogo universiteta. Seriya fiziko-matematicheskie nauki = Scientific Notes of Kazan University. Series Physical and Mathematical Sciences}, 2020, vol.~162, no.~1, pp.~27--37. (in Russian)} \doi{10.26907/2541-7746.2020.1.27-37}

\bibitem{B09}
Гуревич, М.И., \textit{Теория струй идеальной жидкости}. Москва, Наука, 1979. \altbib{Gurevich, M.I., \textit{Teoriya struy ideal'noy zhidkosti = Theory of jets of an ideal fluid}. Moscow, Nauka, 1979. (in Russian)}

\bibitem{B10}
Иванов, А.Н., \textit{Гидродинамика развитых кавитационных течений}. Ленинград, Судостроение, 1980. \altbib{Ivanov, A.N, \textit{Gidrodinamika razvitykh kavitatsionnykh techeniy = Hydrodynamics of developed cavitations flows}. Leningrad, Shipbuilding, 1980. (in Russian)}
\end{thebibliography}

\end{document}
